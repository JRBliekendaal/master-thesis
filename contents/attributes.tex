\chapter{Attributes}
\label{ch:attributes}
\setcounter{footnote}{0}
Before using \glspl{attribute} for validation, we have to define the them. \acrlong{eaal} \parencite[p.~7]{Botjes2021} was selected for \gls{antifragile} \glspl{attribute} (\cref{sec:tbantifragile}) and the \acrlong{ea} school of thought of \gls{enterpriseecologicaladaptation} was selected for the \acrlong{ea} \glspl{attribute} (\cref{sec:tbenterprisearchitecture}). The \glspl{attribute} will be used as input for interviews to determine the current state of \gls{antifragility} and \acrlong{ea} in the \gls{ps}.
\section{Attributes of antifragile}
\label{sec:attributesofantifragile}
The \acrlong{eaal} (\cref{fig:eaalbw}) from \textcite[p.~7]{Botjes2021} is selected as a source for \gls{antifragile} attributes. The \acrlong{eaal} contains \glspl{attribute} for \gls{attenuatevariety}, \gls{amplifyvariety}, and the learning organisation. Essential attributes for an \gls{antifragile} system are \gls{optionality}, \gls{resourcestoinvest}, \gls{senecabarbell}, \gls{insertrandomness}, \gls{reducenaiveintervention}, and \gls{skininthegame} \parencite[p.~64]{Botjes2020}.

\textcite[p.~64]{Botjes2020} explains that \gls{optionality} is excluded because of the overlap with \gls{diversity}. \textcite[p.~66]{Botjes2020} merges both attributes into \gls{diversity}. They are both much alike. But \textcites{Taleb2012}{Gorgeon2015} both use the term \gls{optionality}. \Gls{optionality} is the availability of options \parencites[pp.~176--177]{Taleb2012}[p.~9]{Gorgeon2015}. \Gls{optionality} is an idea advanced by \textcite{Taleb2012}. At the most basic level, \gls{optionality} means having lots of options. E.g. if you develop a skill with many possible job opportunities, you have more \gls{optionality} than someone who develops a skill that only has one or two job opportunities. For \gls{diversity} \textcite{Botjes2020} defined internally A not being a mono-culture and externally having options. E.g. having two different coffee suppliers or having a diverse team. \textcite[Table II]{Botjes2021} refined the definition by adding optionality ''Diversity is the ability to solve a problem in more than one way with different components. Optionality, the availability of options, is a specialisation of diversity. An example is that you want diverse co-workers in a team since other types of people to come up with other types of solutions.''

The difference between \gls{optionality} and \gls{diversity} is very subtle. \Gls{optionality} is when you have the right to do something, but you do not have an obligation to do it, where diversity is something that is there or not. It is an option, ''the right but not the obligation'' for the buyer and, of course, ''the obligation but not the right'' for the other party, called the seller \parencite[p.~174]{Taleb2012}. The option is an agent of \gls{antifragility} \parencite[p.~174]{Taleb2012}. \Gls{optionality} allows the buyer to retain the upper bound and be unaffected by adverse outcomes which makes the buyer \gls{antifragile}\footnote{\label{foot:nesslabs}\url{https://nesslabs.com/optionality-fallacy}}. Since \gls{optionality} is seen as essential for \gls{antifragile} and it has a slightly different definition \gls{optionality} is taken back into the set of possible attributes (\cref{tab:attributesofantifragile}). \Gls{optionality} can be an important differentiator in the \gls{ps}.

\textcite[p.~4]{Botjes2021} used three categories to sort attributes based on behaviour: \gls{attenuatevariety}, \gls{amplifyvariety}, and learning organisation. The same categories are used to sort the \gls{antifragile} \glspl{attribute}. The reinstated \gls{attribute} of \gls{optionality} is placed in the same category as \gls{diversity}, \gls{amplifyvariety}. For the \glspl{attribute} of \gls{antifragile} see \cref{tab:attributesofantifragile}.
\begin{longtable}{@{}p{.4\textwidth}p{.4\textwidth}@{}}
	\toprule%
	\textbf{Attribute} & \textbf{Category} \\%
	\midrule%
	\endhead%
	\hline
	\endfoot%
	\caption[Attributes of antifragile]{Attributes of antifragile}
	\label{tab:attributesofantifragile}
	\endlastfoot%
	\Gls{topdowncc} & \Gls{attenuatevariety} \\%
	\Gls{micromanagement} & \Gls{attenuatevariety} \\%
	\Gls{redundancy} & \Gls{attenuatevariety} \\%
	\Gls{modularity} & \Gls{attenuatevariety} \\%
	\Gls{looselycoupled} & \Gls{attenuatevariety} \\%
	\Gls{diversity} & \Gls{amplifyvariety} \\%
	\Gls{nonmonotonicity} & \Gls{amplifyvariety} \\%
	\Gls{emergence} & \Gls{amplifyvariety} \\%
	\Gls{selforganisation} & \Gls{amplifyvariety} \\%
	\Gls{insertlowlevelstress} & \Gls{amplifyvariety} \\%
	\Gls{networkconnections}  & \Gls{amplifyvariety} \\%
	\Gls{failfast} & \Gls{amplifyvariety} \\%
	\Gls{resourcestoinvest} & \Gls{amplifyvariety} \\%
	\Gls{senecabarbell} & \Gls{amplifyvariety} \\%
	\Gls{insertrandomness} & \Gls{amplifyvariety} \\%
	\Gls{reducenaiveintervention} & \Gls{amplifyvariety} \\%
	\Gls{skininthegame} & \Gls{amplifyvariety} \\%
	\Gls{optionality} &  \Gls{amplifyvariety} \\%
	\Gls{personalmastery} &  \Gls{learningorganisation} \\%
	\Gls{sharedmentalmodel} &  \Gls{learningorganisation} \\%
	\Gls{buildingsharedvision} &  \Gls{learningorganisation} \\%
	\Gls{teamlearning} &  \Gls{learningorganisation} \\%
	\Gls{systemsthinking} &  \Gls{learningorganisation} \\%
	\bottomrule%
\end{longtable}
\newpage
\section{Attributes of Enterprise Architecture}
\label{sec:attributesonea}
The \acrlong{ea} school of thought of \gls{enterpriseecologicaladaptation} was selected as the school of thought that aligns best with \gls{antifragile} (\cref{sec:tbenterprisearchitecture}). The \glspl{attribute} used for Enterprise Architecture will be those of the school of \gls{enterpriseecologicaladaptation} of \textcite[pp.~40--41]{Lapalme2012}. For the \glspl{attribute} of \acrlong{ea} see \cref{tab:attributesofeea}.
\begin{longtable}{@{}p{.4\textwidth}p{.4\textwidth}@{}}
	\toprule%
	\textbf{Attribute} & \textbf{Category} \\%
	\midrule%
	\endhead%
	\hline
	\endfoot%
	\caption[Attributes of Enterprise Architecture]{Attributes of Enterprise Architecture}
	\label{tab:attributesofeea}
	\endlastfoot%
	\Gls{systeminenvironment} & \Gls{enterpriseecologicaladaptation} \\%
	\Gls{holisticsystemicstance} & \Gls{enterpriseecologicaladaptation} \\%
	\Gls{intraorganisationalcoherency} & \Gls{enterpriseecologicaladaptation} \\%
	\Gls{organisationallearning} & \Gls{enterpriseecologicaladaptation} \\%
	\Gls{environmentallearning} & \Gls{enterpriseecologicaladaptation} \\%
	\Gls{systeminenvironmentcoevolutionlearning} & \Gls{enterpriseecologicaladaptation} \\%
	\bottomrule%
\end{longtable}