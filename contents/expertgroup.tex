\chapter{Expert group}
\label{ch:expertgroup}
\setcounter{footnote}{0}
The selected \glspl{attribute} (\cref{sec:attributeslikelysf}) are most likely to have a positive influence on \acrshort{ea} in achieving \gls{antifragility} in the \gls{ps}. These attributes are the result of literature study (\cref{ch:attributes}) and the \acrfull{qda} of interviews (\cref{ch:interviews}). The \glspl{attribute} still needed to be validated for \gls{triangulation} to work. An expert group validated these \glspl{attribute}. All of the expert group participants have experience with \acrshort{ea} and the \gls{ps}. A survey on experience was part of the expert group session (\cref{tab:experiencevalidationgroup}). The expert group participants were selected to balance the group. There was a balanced mix of participants from the central government, the local government, independent software vendors, service providers, and universities. Ten experts participated in the session.

The duration of the expert group session was two hours. The expert group session was online with Microsoft Teams and Meeting Wizard (a group support system). The session was recorded and transcribed. All participants gave their consent. The recording and transcription are not publicly available because they cannot be anonymised\footnote{The Antwerp Management School can request the recordings and transcriptions only for (re)accreditations and visitations to enable the Antwerp Management School to comply with statutory obligations. The recordings and transcriptions are kept for seven years after graduation before they are deleted.}.
\begin{longtable}{@{}p{.55\textwidth}ccc@{}}
	\toprule%
	\textbf{Question} & \textbf{Years} & \textbf{Variability} & \textbf{Abstains} \\
	\midrule%
	\endhead%
	\hline
	\endfoot%
	\caption[Average experience of expert group participants]{Average experience of expert group participants}
	\label{tab:experiencevalidationgroup}%
	\endlastfoot%
	How many years of experience do you have in the field of \acrlong{ea}? & 9,8 & 8\% & 0 \\%
	How many years have you worked as an (Enterprise) Architect? & 10,6 & 12\% & 0 \\%
	How many years of experience do you have in the field of complexity sciences (like \gls{antifragile})? & 7,4 & 16\% & 0 \\%
	How many years of experience do you have with the \gls{ps}? & 12,2 & 17\% & 0 \\%
	How many years of experience do you have with working in publicly-held organisations? & 10 & 16\% & 0 \\%
	How many years of experience do you have with working in privately-held organisations? & 17,2 & 21\% & 0 \\%
	\bottomrule%
\end{longtable}
All the participants received information beforehand by email. This information contained the invite, the goal of the session, the agenda and all relevant definitions. Three recorded seminars given by Nassim Nicholas Taleb\footnote{\url{https://youtu.be/B2-QCv-hChY}}\footnote{\url{https://youtu.be/1NXaafTpVjM}}\footnote{\url{https://youtu.be/C40zwpdc_yo}} were shared to ensure that the participants had a basic understanding of \gls{antifragility}. All participants confirmed that they did see at least one of the videos. The book of \textcite{Taleb2012} was read by multiple participants. For the session the following agenda was used:
\begin{enumerate}
	\item{Introduction}
	\item{Survey on the experience of the participants}
	\item{Presentation on the results of the research up to now\footnote{\url{https://github.com/JRBliekendaal/master-thesis/blob/3666f93bb95308572722082393e684ba40caa5cb/datasets/expertgroup/validationsession.pdf}}}
	\item{Validation of attributes and EA schools of thought}
	\item{Survey on the relevance of the research}
\end{enumerate}
Meeting Wizard supported the surveys and validations. The data set of the surveys and validations is publicly available as a Microsoft Excel file in the public GitHub repository\footnote{\url{https://github.com/JRBliekendaal/master-thesis/blob/3666f93bb95308572722082393e684ba40caa5cb/datasets/expertgroup/dataset_expertgroup.xlsx}} of this research.
\section{Validation of attributes}
\label{sec:validationofattributes}
The two newly found attributes \textit{\gls{adapttobusinesslanguage}} and \textit{\gls{safeworkingenvironment}} (\cref{sec:attributeslikelysf}) were moved to the \gls{antifragile} \glspl{attribute} and \acrshort{ea} \glspl{attribute}. \textit{\Gls{safeworkingenvironment}} was moved to the \glspl{attribute} of \gls{antifragile} and \textit{\gls{adapttobusinesslanguage}} was moved to the \glspl{attribute} of \acrshort{ea}. 

A specific approach was taken to validate the \glspl{attribute}. Through brainstorming as a group, the participants could add new \glspl{attribute}. The new \glspl{attribute} were explained, discussed, combined and sorted. The participants rated the \gls{attribute} on a scale from one to ten, with a one for least likely and a ten for most likely. There was a rating for the \glspl{attribute} of \gls{antifragility} and \acrshort{ea} and there was a rating for the \acrlong{ea} schools of thought. The participants rated the sets based on the following questions:
\begin{enumerate}
	\item{For the \glspl{attribute}: ''To what extent is the \gls{attribute} a success factor for \gls{antifragility} in the \gls{ps}?''}
	\item{For the \acrlong{ea} schools of thought: ''To what extent is the \acrlong{ea} school of thought a success factor for \gls{antifragility} in the \gls{ps}?''}
\end{enumerate}
\subsection{Validation of antifragile attributes}
\label{sub:validationofantifragileattributes}
While brainstorming, the participants came up with twelve possible new \glspl{attribute}. After discussion, only two remained. These two were \textit{'\gls{outsideincollaboration}'} and \textit{'\gls{datagovernanceplanes}'}. The others were the same as another \gls{attribute} or were a child of one of the other \glspl{attribute}. 
The participants rated a total of nine \glspl{attribute}. As \cref{tab:validationofantifragileattributes} shows there are three \glspl{attribute} that had a variability of more than 50\%, and only two \glspl{attribute} scored less than a rating of six. The new \glspl{attribute} were among those. There was one abstain on three \glspl{attribute}.
\begin{table}[H]
	\centering
	\begin{tabular}{p{.55\textwidth}ccc}
		\toprule
		\textbf{Attribute} & \textbf{Rating} & \textbf{Variability} & \textbf{Abstains} \\
		\midrule
		\Gls{optionality} & 6,9 & 32\% & 0 \\%
		\Gls{nonmonotonicity} & 7 & 51\% & 0 \\%
		\Gls{selforganisation} & 8,2 & 23\% & 0 \\%
		\Gls{failfast} & 7,8 & 35\% & 0 \\%
		\Gls{resourcestoinvest} & 6,7 & 36\% & 1 \\%
		\Gls{senecabarbell} & 5,8 & 37\% & 1 \\%
		\Gls{safeworkingenvironment} & 7,4 & 31\% & 0 \\%
		\Gls{outsideincollaboration} & 6,2 & 55\% & 0 \\%
		\Gls{datagovernanceplanes} & 4,4 & 56\% & 1 \\%
		\bottomrule
	\end{tabular}%
	\caption{Validation of antifragile attributes}
	\label{tab:validationofantifragileattributes}%
\end{table}%

\subsection{Validation of Enterprise Architecture schools of thought}
\label{sub:validationofeaschools}
Validating the \acrlong{ea} schools of thought needed a different approach. The presentation\footnote{\url{https://github.com/JRBliekendaal/master-thesis/blob/3666f93bb95308572722082393e684ba40caa5cb/datasets/expertgroup/validationsession.pdf}} introduced the attributes of \gls{antifragile} and \acrshort{ea}. The expert group could extend the list of \glspl{attribute} with new \glspl{attribute}. There is a high chance of influencing the expert group when presenting the most likely \acrlong{ea} school of thought. The approach was to use the expert group to validate the findings in a non-biased way. The validation used all three schools of thought. The expert group could rate the likelihood of the school of thought positively influencing \acrlong{ea} in achieving \gls{antifragility} in the \gls{ps}. 
\begin{table}[H]
	\centering
	\begin{tabular}{p{.55\textwidth}ccc}
		\toprule
		\textbf{School of thought} & \textbf{Rating} & \textbf{Variability} & \textbf{Abstains} \\
		\midrule%
		\Gls{enterpriseitarchitecting} & 5,6   & 34\%  & 0 \\
		\Gls{enterpriseintegrating} & 7,2   & 16\%  & 0 \\
		\Gls{enterpriseecologicaladaptation} & 8,8   & 27\%  & 0 \\
		\bottomrule%
	\end{tabular}%
	\caption[Rating of Enterprise Architecture schools of thought]{Rating of Enterprise Architecture schools of thought}
	\label{tab:validationgscoreeaschools}
\end{table}%
The rating had low variability, and no abstains (\cref{tab:validationgscoreeaschools}). \textit{\textit{\gls{enterpriseitarchitecting}}} had the lowest rating and \textit{\textit{\gls{enterpriseecologicaladaptation}}} had the highest with \textit{\textit{\gls{enterpriseintegrating}}} in between. It confirmed the results of the literature research (\cref{sub:eaapproaches}) and the interviews (\cref{sec:resultsofqda}).
\subsection{Validation of Enterprise Architecture attributes}
\label{sub:validationofenterprisearchitectureattributes}
The validation of the \acrshort{ea} \glspl{attribute} used the same approach like that of the \gls{antifragile} \glspl{attribute}. The validation contained the \glspl{attribute} of the \acrlong{ea} school of thought of \gls{enterpriseecologicaladaptation}. Brainstorming resulted in nine new identified \glspl{attribute}. After discussion, five remained. These five were \textit{\gls{agileenterprise}}, \textit{\gls{realtimetrust}}, \textit{\gls{fosterdialogue}}, \textit{\gls{architecturevalidation}} and \textit{\gls{alwaysfittingarchitecture}}. The participants rated the the attributes.
\begin{table}[H]
	\centering
	\begin{tabular}{p{.55\textwidth}ccc}
		\toprule
		\textbf{Attribute} & \textbf{Rating} & \textbf{Variability} & \textbf{Abstains} \\
		\midrule
		\Gls{systeminenvironment} & 7,7   & 28\%  & 0 \\
		\Gls{holisticsystemicstance} & 7     & 47\%  & 0 \\
		\Gls{organisationallearning} & 7,3   & 44\%  & 0 \\
		\Gls{environmentallearning} & 7,7   & 29\%  & 0 \\
		\Gls{intraorganisationalcoherency} & 6,4   & 31\%  & 0 \\
		\Gls{systeminenvironmentcoevolutionlearning} & 6,6   & 36\%  & 0 \\
		\Gls{adapttobusinesslanguage} & 7,1   & 35\%  & 0 \\
		\Gls{agileenterprise} & 6,4   & 50\%  & 0 \\
		\Gls{realtimetrust} & 5,6   & 54\%  & 1 \\
		\Gls{fosterdialogue} & 6,9   & 32\%  & 0 \\
		\Gls{architecturevalidation} & 7,4   & 24\%  & 0 \\
		\Gls{alwaysfittingarchitecture} & 5,8   & 46\%  & 1 \\
		\bottomrule
	\end{tabular}%
	\caption[Rating of Enterprise Architecture attributes]{Rating of Enterprise Architecture attributes}
	\label{tab:validationofeaattributes}%
\end{table}%
The rating shows that two \glspl{attribute} have a variability of 50\% or higher, and only one \gls{attribute} got a rating of less than six (\cref{tab:validationofeaattributes}). All three were from the five new \glspl{attribute} from the expert group. There were only two abstains on a total of two attributes.
\section{Relevance of the research}
\label{sec:relevanceofresearch}
The final part of the expert group session was about the relevance of the research. A survey determined the research's relevance. The expert group rated the research on different areas of application. These areas of relevance were, \textit{in general}, \textit{for themselves}, \textit{for the public sector} and \textit{for the organisation of the expert}. The last question asked was if the expert group session fulfilled their expectations.

The variability of the ratings was low. There was only one abstain on the relevance of the research for the public sector. The question that scored the least was about the relevance for the organisation of the expert. The expert group finds the research relevant. They rated it with a rating of 8,2 (\cref{tab:relevanceofresearch}).
\begin{table}[H]
	\centering
	\begin{tabular}{p{.55\textwidth}ccc}
		\toprule
		\textbf{Question} & \textbf{Rating} & \textbf{Variability} & \textbf{Abstains} \\
		\midrule
		To what extent do you find the research relevant? & 8,2 & 23\% & 0 \\%
		To what extent did this session fulfil your expectations? & 8 & 24\% & 0 \\%
		To what extent do you think that the research can be used by yourself? & 7,7 & 10\% & 0 \\%
		To what extent do you think that the research can be used in the public sector? & 7,2 & 32\% & 1 \\%
		To what extent do you think that the research can be used by your organisation? & 6,6 & 33\% & 0 \\%
		\bottomrule
	\end{tabular}%
	\caption[Rating of the relevance of the research]{Rating of the relevance of the research}
	\label{tab:relevanceofresearch}%
\end{table}%
\section{Attributes most likely to be a success factor}
\label{sec:expertattributessf}
Threshold of six out of ten. the higher the higher the quality and more valid.
Variability cannot be used. It is not xxx because of n=10


\begin{longtable}{@{}ll@{}}
	\toprule%
	\textbf{Attribute} & \textbf{Category}  \\%
	\midrule%
	\endhead%
	\hline
	\endfoot%
	\caption[Possible success factors identified by the expert group]{Possible success factors identified by the expert group}
	\label{tab:expertgrouppossiblesf}
	\endlastfoot%
	\Gls{optionality} & \Gls{antifragile} \\%
	\Gls{nonmonotonicity} & \Gls{antifragile} \\%
	\Gls{selforganisation} & \Gls{antifragile} \\%
	\Gls{failfast} & \Gls{antifragile} \\%
	\Gls{resourcestoinvest} & \Gls{antifragile} \\%
	\Gls{safeworkingenvironment} & New \Gls{antifragile} \\%
	\Gls{outsideincollaboration} & New \Gls{antifragile} \\%
	\Gls{systeminenvironment} & \acrlong{ea} \\%
	\Gls{holisticsystemicstance} & \acrlong{ea} \\%
	\Gls{organisationallearning} & \acrlong{ea} \\%
	\Gls{environmentallearning} & \acrlong{ea} \\%
	\Gls{intraorganisationalcoherency} & \acrlong{ea} \\%
	\Gls{systeminenvironmentcoevolutionlearning} & \acrlong{ea} \\%
	\Gls{adapttobusinesslanguage} & \acrlong{ea} \\%
	\Gls{agileenterprise} & New \acrlong{ea} \\%
	\Gls{fosterdialogue} & New \acrlong{ea} \\%
	\Gls{architecturevalidation} & New \acrlong{ea} \\%
	\bottomrule%
\end{longtable}%
