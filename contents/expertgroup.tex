\chapter{Expert group}
\label{ch:expertgroup}
\setcounter{footnote}{0}
In \cref{ch:interviews}, the \glspl{attribute} were selected. The selected \glspl{attribute} are most likely to have a positive influence on \acrshort{ea} in achieving \gls{antifragility} in the \gls{ps}. These attributes are the result of literature study (\cref{ch:attributes}) and the qualitative analysis of interviews (\cref{ch:interviews}). The \glspl{attribute} still needed to be validated for \gls{triangulation} to work. An expert group validated these \glspl{attribute}. All of the expert group participants have experience with \acrshort{ea} and the \gls{ps}. A survey on experience was part of the expert group session (\cref{tab:experiencevalidationgroup}). The expert group participants were selected to balance the group. There was a balanced mix of participants from the central government, the local government, independent software vendors, service providers, and universities. Ten experts participated in the session.

The expert group session had a duration of 2 hours. The expert group session was online with Microsoft Teams and Meeting Wizard (a group support system). There is a recording and a transcription available of this session. All participants gave their consent. The transcription and recording are not publicly available because they cannot be anonymised\footnote{The Antwerp Management School can request the recordings and transcriptions only for (re)accreditations and visitations to enable the Antwerp Management School to comply with statutory obligations. The recordings and transcriptions are kept for seven years after graduation before being deleted.}.
\begin{table}[H]
	\centering
	\begin{tabular}{p{.55\textwidth}ccc}
		\toprule
		\textbf{Question} & \textbf{Years} & \textbf{Variability} & \textbf{Abstains} \\
		\midrule
		How many years of experience do you have in the field of enterprise architecture? & 9,8 & 8\% & 0 \\%
		How many years have you worked as an (enterprise) architect? & 10,6 & 12\% & 0 \\%
		How many years of experience do you have in the field of complexity sciences (like antifragile)? & 7,4 & 16\% & 0 \\%
		How many years of experience do you have with the public sector? & 12,2 & 17\% & 0 \\%
		How many years of experience do you have with working in publicly-held organisations? & 10 & 16\% & 0 \\%
		How many years of experience do you have with working in privately-held organisations? & 17,2 & 21\% & 0 \\%
		\bottomrule
	\end{tabular}%
	\caption[The average experience of expert group participants]{The average experience of expert group participants}
	\label{tab:experiencevalidationgroup}%
\end{table}%
All the participants received information beforehand by email. This information contained the invite, the goal of the session, the agenda and all relevant definitions. Three youtube videos\footnote{\url{https://www.youtube.com/watch?v=B2-QCv-hChY}, \url{https://www.youtube.com/watch?v=1NXaafTpVjM}, and \url{https://www.youtube.com/watch?v=C40zwpdc_yo}} were shared to ensure that the participants had a basic understanding of \gls{antifragility}. All participants confirmed that they did see at least one of the videos. There were also multiple participants who read the book of \textcite{Taleb2012}. For the session the following agenda was used:
\begin{enumerate}
	\item{Introduction}
	\item{Survey on experience\footnote{for the outcome of this survey see \cref{tab:experiencevalidationgroup}}}
	\item{Presentation on the results of the research up to now\footnote{\url{https://github.com/JRBliekendaal/master-thesis/blob/main/datasets/expertgroup/validationsession.pdf}}}
	\item{Validation of attributes and ea school of thought}
	\item{Survey on the relevance of the research}
\end{enumerate}
Meeting Wizard supported the survey's and the validations. The output from these 




\section{Expert group validation}
\label{sec:expertgroupvalidation}

the two newly found attributes \gls{adapttobusinesslanguage} and \gls{safeworkingenvironment} were logically placed with the categories antifragile and Enterprise Architecture attributes. \Gls{adapttobusinesslanguage} had something to do with enterprise architecture and safe working environment had more to do with the antifragile attributes.


\subsection{Validation of antifragile attributes}
\label{sub:validationofantifragileattributes}

\begin{table}[H]
	\centering
	\begin{tabular}{p{.55\textwidth}ccc}
		\toprule
		\textbf{Attribute} & \textbf{Rating} & \textbf{Variability} & \textbf{Abstains} \\
		\midrule
		Optionality & 6,9 & 32\% & 0 \\%
		Mono-Monotonicity & 7 & 51\% & 0 \\%
		Self-Organisation & 8,2 & 23\% & 0 \\%
		Fail-Fast & 7,8 & 35\% & 0 \\%
		Resources to Invest & 6,7 & 36\% & 1 \\%
		Seneca's Barbell & 5,8 & 37\% & 1 \\%
		Safe working environment & 7,4 & 31\% & 0 \\%
		Naar buiten kijken, samenwerking zoeken & 6,2 & 55\% & 0 \\%
		Data Governance planes (tbv infrastructure-as-code / Software Defined Anything) & 4,4 & 56\% & 1 \\%
		\bottomrule
	\end{tabular}%
	\caption{Validation of antifragile attributes}
	\label{tab:validationofantifragileattributes}%
\end{table}%


\subsection{Validation of Enterprise Architecture attributes}
\label{sub:validationofenterprisearchitectureattributes}




\section{title}

