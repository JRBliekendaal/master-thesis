\chapter{Expert group}
\label{ch:expertgroup}
\setcounter{footnote}{0}
The \glspl{attribute}, as a result of the interviews (\cref{sec:attributeslikelysf}), probably have a positive influence on \gls{ea} in achieving \gls{antifragility} in the Dutch \gls{ps}. We used these \glspl{attribute} for validation by an expert group. All the expert group participants were selected to have experience with \gls{ea} and the Dutch \gls{ps}. The expert group was composed of experts from different organisations of the Dutch \gls{ps}. A survey on experience with \gls{ea}, \gls{antifragility}, and the \gls{ps} was part of the expert group session (\cref{tab:experiencevalidationgroup}). Central governments, local governments, independent software vendors, service providers, and universities all had delegates in the expert group. A total of ten experts participated in an online session. 
 
The duration of the expert group session was two hours. The expert group session was online with support of Microsoft Teams and Meetingwizard. The session was recorded and automatically transcribed. All participants gave their consent for the recording and transcription.
\begin{table}[H]
	\centering
	\resizebox{\textwidth}{!}{%
		\begin{tabular}{p{.5\textwidth}ccc}
			\toprule
			\textbf{Question} & \textbf{Average years} & \textbf{Variability} & \textbf{Abstains} \\
			\midrule
			How many years of experience do you have in the field of \gls{ea}? & 9,8 & 8\% & 0 \\%
			How many years have you worked as an (Enterprise) Architect? & 10,6 & 12\% & 0 \\%
			How many years of experience do you have in the field of complexity sciences (like \gls{antifragile})? & 7,4 & 16\% & 0 \\%
			How many years of experience do you have with the \gls{ps}? & 12,2 & 17\% & 0 \\%
			How many years of experience do you have with working in publicly-held organisations? & 10 & 16\% & 0 \\%
			How many years of experience do you have with working in privately-held organisations? & 17,2 & 21\% & 0 \\%
			\bottomrule
		\end{tabular}%
	}%
	\caption[Average experience of expert group participants]{Average experience of expert group participants}
	\label{tab:experiencevalidationgroup}%
\end{table}%

All the participants received information beforehand. This information contained the invite, the goal of the session, the agenda and all relevant definitions. Three recorded seminars given by Nassim Nicholas Taleb\footnote{\url{https://youtu.be/B2-QCv-hChY}}\footnote{\url{https://youtu.be/1NXaafTpVjM}}\footnote{\url{https://youtu.be/C40zwpdc_yo}} were shared to ensure that the participants had a basic understanding of \gls{antifragility}. All participants confirmed that they did see at least one of the videos. The book of \textcite{Taleb2012} was read by multiple participants.
\newpage
We used the following agenda for the session:
\begin{enumerate}
	\item{Introduction}
	\item{Survey on the experience of the participants}
	\item{Presentation on the results of the research up to now\footnote{\url{https://github.com/JRBliekendaal/master-thesis/blob/3666f93bb95308572722082393e684ba40caa5cb/datasets/expertgroup/validationsession.pdf}}}
	\item{Validation of \gls{antifragile} \glspl{attribute}}
	\item{Validation of the \gls{ea} schools of thought and the \glspl{attribute} of \gls{ea}}
	\item{Survey on the relevance of the research}
\end{enumerate}
Meetingwizard supported the surveys and validations. The data set of the surveys and validations is publicly available as a Microsoft Excel file in the public GitHub repository\footnote{\url{https://github.com/JRBliekendaal/master-thesis/blob/3666f93bb95308572722082393e684ba40caa5cb/datasets/expertgroup/dataset_expertgroup.xlsx}} of this research.
\section{Validation of attributes}
\label{sec:validationofattributes}
The two newly found attributes \textit{\gls{safeworkingenvironment}} and \textit{\gls{adapttobusinesslanguage}} (\cref{sec:attributeslikelysf}) were moved to the \gls{antifragile} \glspl{attribute} and \gls{ea} \glspl{attribute}. We used brainstorming to make sure that we did not miss possible \glspl{attribute} that are specific to the Dutch \gls{ps}. Through brainstorming as a group, the participants could add new \glspl{attribute}. The expert group explained, discussed, combined and sorted the added \glspl{attribute}. The participants validated the \glspl{attribute} one by one. They used a scale from one to ten, one for least and ten for most probable. There was a validation for the \glspl{attribute} of \gls{antifragility}, \gls{ea} and for the \gls{ea} schools of thought. The participants validated the \glspl{attribute} by answering the following questions:
\begin{enumerate}
	\item{For the \glspl{attribute}: ''To what extent is the \gls{attribute} a success factor for \gls{antifragility} in the \gls{ps}?''}
	\item{For the \gls{ea} schools of thought: ''To what extent is the \gls{ea} school of thought a success factor for \gls{antifragility} in the \gls{ps}?''}
\end{enumerate}
\subsection{Validation of antifragile attributes}
\label{sub:validationofantifragileattributes}
While brainstorming, the participants came up with twelve new \glspl{attribute}. After discussion, only two remained. These two were \textit{'\gls{outsideincollaboration}'} and \textit{'\gls{datagovernanceplanes}'}. The others were the same as another \gls{attribute} or were a child of one of the other \glspl{attribute}. The participants validated a total of nine \glspl{attribute}. There are three \glspl{attribute} that have a variability of more than 40\%, and only two \glspl{attribute} had an average rating of six (\cref{tab:validationofantifragileattributes}). The new \glspl{attribute} were among those. There was one abstain on three \glspl{attribute}. \Cref{sec:validationofafattributes} contains the details of the validation per \gls{attribute}.
\begin{table}[H]
	\centering
	\resizebox{\textwidth}{!}{%
	\begin{tabular}{p{.5\textwidth}ccc}
		\toprule
		\textbf{Attribute} & \textbf{Average rating} & \textbf{Variability} & \textbf{Abstains} \\
		\midrule
		\Gls{optionality} & 6,9 & 32\% & 0 \\%
		\Gls{nonmonotonicity} & 7 & 51\% & 0 \\%
		\Gls{selforganisation} & 8,2 & 23\% & 0 \\%
		\Gls{failfast} & 7,8 & 35\% & 0 \\%
		\Gls{resourcestoinvest} & 6,7 & 36\% & 1 \\%
		\Gls{senecabarbell} & 5,8 & 37\% & 1 \\%
		\Gls{safeworkingenvironment} & 7,4 & 31\% & 0 \\%
		\Gls{outsideincollaboration} & 6,2 & 55\% & 0 \\%
		\Gls{datagovernanceplanes} & 4,4 & 56\% & 1 \\%
		\bottomrule
	\end{tabular}%
	}%
	\caption{Validation of antifragile attributes}
	\label{tab:validationofantifragileattributes}%
\end{table}%

\subsection{Validation of Enterprise Architecture schools of thought}
\label{sub:validationofeaschools}
Validating the \gls{ea} schools of thought needed a somewhat different approach. The presentation\footnote{\url{https://github.com/JRBliekendaal/master-thesis/blob/3666f93bb95308572722082393e684ba40caa5cb/datasets/expertgroup/validationsession.pdf}} given to the expert group introduced the \glspl{attribute} of \gls{antifragile} and \gls{ea}. The expert group could extend the list of \glspl{attribute} with new \glspl{attribute} by brainstorming. There is a high chance of influencing the expert group when presenting the \gls{ea} school of thought with the probability of being a success factor. The approach was to use the expert group to validate the findings in a non-biased way. Because of this, the validation used all three schools of thought. The expert group could rate the probability of the school of thought positively influencing \gls{ea} in achieving \gls{antifragility} in the Dutch \gls{ps}. 

The validation had low variability, and no abstains (\cref{tab:validationgscoreeaschools}). \textit{\gls{enterpriseitarchitecting}} had the lowest average rating and \textit{\gls{enterpriseecologicaladaptation}} had the highest with \textit{\gls{enterpriseintegrating}} in between. The validation confirmed the results of the literature research (\cref{sub:definingea}) and the interviews (\cref{sec:dataprep}). \Cref{sec:validationofeaschools} contains the details of the validation per \gls{ea} school of thought.
\begin{table}[H]
	\centering
	\resizebox{\textwidth}{!}{%
	\begin{tabular}{p{.5\textwidth}ccc}
		\toprule
		\textbf{School of thought} & \textbf{Average rating} & \textbf{Variability} & \textbf{Abstains} \\
		\midrule%
		\Gls{enterpriseitarchitecting} & 5,6   & 34\%  & 0 \\
		\Gls{enterpriseintegrating} & 7,2   & 16\%  & 0 \\
		\Gls{enterpriseecologicaladaptation} & 8,8   & 27\%  & 0 \\
		\bottomrule%
	\end{tabular}%
	}%
	\caption[Validation of Enterprise Architecture schools of thought]{Validation of Enterprise Architecture schools of thought}
	\label{tab:validationgscoreeaschools}
\end{table}%
\subsection{Validation of Enterprise Architecture attributes}
\label{sub:validationofenterprisearchitectureattributes}
The validation of the \gls{ea} \glspl{attribute} used the same approach like that of the \gls{antifragile} \glspl{attribute}. The validation contained the \glspl{attribute} of the \gls{ea} school of thought of \gls{enterpriseecologicaladaptation}. Brainstorming resulted in nine new identified \glspl{attribute}. After discussion, five remained. These five were \textit{\gls{agileenterprise}}, \textit{\gls{realtimetrust}}, \textit{\gls{fosterdialogue}}, \textit{\gls{architecturevalidation}} and \textit{\gls{alwaysfittingarchitecture}}. The participants rated all the \glspl{attribute}.

The validation shows that five \glspl{attribute} have a variability of 40\% or higher, and only one \gls{attribute} got a rating of less than six (\cref{tab:validationofeaattributes}). Three of the five new attributes had a variability of 40\% or higher. There were only two abstains on a total of two attributes. \Cref{sec:validationofeaattributes} contains the details of the rating per \gls{attribute}.
\begin{table}[H]
	\centering
	\resizebox{\textwidth}{!}{%
	\begin{tabular}{p{.5\textwidth}ccc}
		\toprule
		\textbf{Attribute} & \textbf{Average rating} & \textbf{Variability} & \textbf{Abstains} \\
		\midrule
		\Gls{systeminenvironment} & 7,7   & 28\%  & 0 \\
		\Gls{holisticsystemicstance} & 7     & 47\%  & 0 \\
		\Gls{organisationallearning} & 7,3   & 44\%  & 0 \\
		\Gls{environmentallearning} & 7,7   & 29\%  & 0 \\
		\Gls{intraorganisationalcoherency} & 6,4   & 31\%  & 0 \\
		\Gls{systeminenvironmentcoevolutionlearning} & 6,6   & 36\%  & 0 \\
		\Gls{adapttobusinesslanguage} & 7,1   & 35\%  & 0 \\
		\Gls{agileenterprise} & 6,4   & 50\%  & 0 \\
		\Gls{realtimetrust} & 5,6   & 54\%  & 1 \\
		\Gls{fosterdialogue} & 6,9   & 32\%  & 0 \\
		\Gls{architecturevalidation} & 7,4   & 24\%  & 0 \\
		\Gls{alwaysfittingarchitecture} & 5,8   & 46\%  & 1 \\
		\bottomrule
	\end{tabular}%
	}%
	\caption[Validation of Enterprise Architecture attributes]{Validation of Enterprise Architecture attributes}
	\label{tab:validationofeaattributes}%
\end{table}%

\section{Relevance of the research}
\label{sec:relevanceofresearch}
The final part of the expert group session was about the relevance of the research. A survey determined the research's relevance. The expert group rated the research on different areas of application. These areas of relevance were, \textit{in general}, \textit{for themselves}, \textit{for the public sector} and \textit{for the organisation of the expert}. The last question asked was if the expert group session fulfilled the expectations.

The variability of the survey was low. There was only one abstain on the relevance of the research for the public sector. The question that scored the least was about the relevance for the organisation of the expert. The expert group finds the research relevant. They rated it with a rating of 8,2 (\cref{tab:relevanceofresearch}). \Cref{sec:relevanceofresearchandexpectations} contains the details of the survey per question.

\begin{table}[H]
	\centering
	\resizebox{\textwidth}{!}{%
	\begin{tabular}{p{.55\textwidth}ccc}
		\toprule
		\textbf{Question} & \textbf{Average rating} & \textbf{Variability} & \textbf{Abstains} \\
		\midrule
		To what extent do you find the research relevant? & 8,2 & 23\% & 0 \\%
		To what extent did this session fulfil your expectations? & 8 & 24\% & 0 \\%
		To what extent do you think that the research can be used by yourself? & 7,7 & 10\% & 0 \\%
		To what extent do you think that the research can be used in the public sector? & 7,2 & 32\% & 1 \\%
		To what extent do you think that the research can be used by your organisation? & 6,6 & 33\% & 0 \\%
		\bottomrule
	\end{tabular}%
	}%
	\caption[Validation on the relevance of the research]{Validation on the relevance of the research}
	\label{tab:relevanceofresearch}%
\end{table}%
\section{Potential success factors}
\label{sec:expertattributessf}
The used research approach is a qualitative method. The number of participants was ten. We believe that n=10 is too small of a sample size to use pure quantitative tools. We decided to use variability and the average rating as discriminators. We did not use the abstains as discriminator. If there was an abstain it was with a maximum of one. We decided to use the following rules for selection.

\begin{enumerate}
	\item{\textbf{Variability.} We decided that an \gls{attribute} must have a variability of 40\% or less. Exceptions are possible when the expert group decided on it after discussion. If there is an exception it will be noted.}
	\item{\textbf{Average rating.} The attributes left after applying the first rule must have a average rating of 6 or higher to be noted as an attribute with potential.}
\end{enumerate}

\subsection{Selected antifragile attributes}
\label{sub:validationselectedafattributes}
Applying the rules for selection resulted in four dropped and five accepted \glspl{attribute} (\cref{tab:expertgrouppossiblesf}). The four dropped \glspl{attribute} are \gls{nonmonotonicity}, \gls{senecabarbell}, \gls{outsideincollaboration}, and \gls{datagovernanceplanes}. The accepted \glspl{attribute} are \textit{\gls{optionality}}, \textit{\gls{selforganisation}}, \textit{\gls{failfast}}, \textit{\gls{resourcestoinvest}}, and \textit{\gls{safeworkingenvironment}}. None of the expert group's proposed \glspl{attribute} were selected. The rule dropped only two \gls{attribute} from the literature study, \textit{\gls{nonmonotonicity}} and \textit{\gls{senecabarbell}}.
\begin{table}[H]
	\centering
	%\resizebox{\textwidth}{!}{%
		\begin{tabular}{@{}lccc@{}}
			\toprule
			\textbf{Attribute} & \textbf{Variability} & \textbf{Average rating} & \textbf{Selected} \\%
			\midrule
			\Gls{optionality} & 32\% & 6,9 & \checkmark \\%
			\Gls{nonmonotonicity} & 51\% & 7 &  \\%
			\Gls{selforganisation} & 23\% & 8,2 & \checkmark  \\%
			\Gls{failfast} & 35\% & 7,8 & \checkmark  \\%
			\Gls{resourcestoinvest} & 36\% & 6,7 & \checkmark \\%
			\Gls{senecabarbell} & 37\% & 5,8 &  \\%
			\Gls{safeworkingenvironment} & 31\% & 7,4 & \checkmark  \\%
			\Gls{outsideincollaboration} & 55\% & 6,2 &  \\%
			\Gls{datagovernanceplanes} & 56\% & 4,4 &  \\%
			\bottomrule
		\end{tabular}%
%	}%
	\caption[Identified probable antifragile attributes by the expert group]{Identified probable antifragile attributes by the expert group}
	\label{tab:expertgrouppossiblesf}%
\end{table}%

\subsection{Selected Enterprise Architecture attributes}
\label{sub:validationselectedeaattributes}
Applying the rules for selection resulted in five dropped \glspl{attribute} and seven selected \glspl{attribute} (\cref{tab:expertgrouppossiblesfea}). The five dropped \glspl{attribute} are \textit{\gls{holisticsystemicstance}}, \textit{\gls{organisationallearning}}, \textit{\gls{agileenterprise}}, \textit{\gls{realtimetrust}}, and \textit{\gls{alwaysfittingarchitecture}}. The seven selected attributes are \textit{\gls{systeminenvironment}}, \textit{\gls{environmentallearning}}, \textit{\gls{intraorganisationalcoherency}}, \textit{\gls{systeminenvironmentcoevolutionlearning}}, \textit{\gls{adapttobusinesslanguage}}, \textit{\gls{fosterdialogue}}, and \textit{\gls{architecturevalidation}}. Two out of five attributes that were proposed by the expert group are selected. \textit{\Gls{agileenterprise}}, \textit{\gls{realtimetrust}}, and \textit{\gls{alwaysfittingarchitecture}} were dropped.
\begin{table}[H]
	\centering
	\resizebox{\textwidth}{!}{%
		\begin{tabular}{@{}lccc@{}}
			\toprule
			\textbf{Attribute} & \textbf{Variability} & \textbf{Average rating} & \textbf{Selected} \\%
			\midrule
			\Gls{systeminenvironment} & 28\% & 7,2 & \checkmark  \\%
			\Gls{holisticsystemicstance} & 47\% & 7 &  \\%
			\Gls{organisationallearning}  & 44\% & 7,3 & \\%
			\Gls{environmentallearning} & 29\% & 7,7 & \checkmark  \\%
			\Gls{intraorganisationalcoherency} & 31\% & 6,4 & \checkmark  \\%
			\Gls{systeminenvironmentcoevolutionlearning} & 36\% & 6,6 & \checkmark \\%
			\Gls{adapttobusinesslanguage} & 35\% & 7,1 & \checkmark  \\%
			\Gls{agileenterprise} & 50\% & 6,4 &  \\%
			\Gls{realtimetrust} & 54\% & 5,6 &  \\%
			\Gls{fosterdialogue} & 32\% & 6,9 & \checkmark \\%
			\Gls{architecturevalidation} & 24\% & 7,4 & \checkmark \\%
			\Gls{alwaysfittingarchitecture} & 46\% & 5,8 &  \\%
			\bottomrule
		\end{tabular}%
		}%
	\caption[Identified Enterprise Architecture attributes by the expert group]{Identified Enterprise Architecture attributes by the expert group}
	\label{tab:expertgrouppossiblesfea}%
\end{table}%

\subsection{Selected attributes as probable success factors}
\label{sub:validationselectedattributesassf}
Combining both sets (\cref{tab:expertgrouppossiblesf,tab:expertgrouppossiblesfea}) gives an overview. This overview summarises the \glspl{attribute} that are rated best by the expert group (\cref{tab:expertgroupp2ossiblesf}). These \glspl{attribute} can be of significance in achieving \gls{antifragility} with \gls{ea} in the Dutch \gls{ps}. The set contains fifteen \gls{attribute}. Six \gls{antifragile} \glspl{attribute} from which one new discovered \gls{attribute}. Furthermore, nine \gls{ea} \glspl{attribute} from which two new discovered \glspl{attribute}.
\begin{table}[H]
	\centering
		\begin{tabular}{@{}ll@{}}
				\toprule
				\textbf{Attribute} & \textbf{Category}  \\%
				\midrule
				\Gls{optionality} & \Gls{antifragile} \\%
				\Gls{nonmonotonicity} & \Gls{antifragile} \\%
				\Gls{selforganisation} & \Gls{antifragile} \\%
				\Gls{failfast} & \Gls{antifragile} \\%
				\Gls{resourcestoinvest} & \Gls{antifragile} \\%
				\Gls{safeworkingenvironment} & New \Gls{antifragile} \\%
				\Gls{systeminenvironment} & \gls{ea} \\%
				\Gls{holisticsystemicstance} & \gls{ea} \\%
				\Gls{organisationallearning} & \gls{ea} \\%
				\Gls{environmentallearning} & \gls{ea} \\%
				\Gls{intraorganisationalcoherency} & \gls{ea} \\%
				\Gls{systeminenvironmentcoevolutionlearning} & \gls{ea} \\%
				\Gls{adapttobusinesslanguage} & \gls{ea} \\%
				\Gls{fosterdialogue} & New \gls{ea} \\%
				\Gls{architecturevalidation} & New \gls{ea} \\%
				\bottomrule
			\end{tabular}%
		\caption[Probable success factors identified by the expert group]{Probable success factors identified by the expert group}
		\label{tab:expertgroupp2ossiblesf}%
	\end{table}%
