\phantomsection
\addcontentsline{toc}{chapter}{Acknowledgements}
\chapter*{Acknowledgements}
%\begin{comment}
{\small%
	I do believe you should never stop learning. When you stop learning, you will stop developing yourself while your life and profession are in flux. That is why I am always following an education, professional training or reading books on various subjects. A couple of years ago, I decided to pursue a bachelor's degree in Business \& IT. At that time, I had classes on Enterprise Engineering. A study of eighteen weeks on Design \& Engineering Methodology for Organisations. At the end of the first day, we had a flash visit from Hans Mulder, who drove by after a lecture at the Nyenrode Business University. Hans told us that this particular bachelor was the perfect preparation for the Executive Master of Enterprise IT Architecture (MEITA) at the Antwerp Management School (AMS). At that moment, I had never heard of the AMS, and I even was surprised that there was an educational track for architecture. I visited an introduction evening at AMS for more information. After my visit, I knew it for sure. I am privileged that my directors Dieneke Schouten and Maarten Hillenaar supported me to pursue the MEITA at the AMS. Not only for making it possible but also for supporting me during my research with all means necessary.

	I still remember what Steven de Haes (the dean of the AMS) told us during the opening seminar at Corsendonk (BE). Studying at the AMS will be a life-changing experience. It did not feel that way in the first year. COVID-19 happened, and classes on-site were impossible. Everything was online after Corsendonk. There was a noticeable distance between the students and the Antwerp Management School. Gladly this changed the second and last year of the MEITA.
	
	At least once a month to AMS for two days of masterclasses and some fun with my fellow students. We have grown from a group of casual students to a group with strong ties in social and business. We endured a lot together in that last year. I will never forget our Thursday nights with 'Bollekes'. Always with the same group of the MEITA with Ingrid, Stefan, Didier, Marc, Cole, Maarten, and Gijs.
	
	I started thinking of a subject for my research and thesis right after we started the master's. I started researching antifragility combined with enterprise architecture in the Dutch public sector. I did this with Hans's support as my promotor and Edzo Botjes as my co-promotor. Edzo already researched antifragility during his master's at the AMS. Thank you both for this life-changing experience. Edzo I will never forget our collaboration and the philosophical discussions we had. But all good things come to an end. I finished my research and thesis, and so has my journey at the AMS. If you ask me what the most important lesson is that I have learned during the past two years, I can only give the following answer:
	\begin{center}
	\textbf{\textit{''I discovered that I have found out how little I actually know.''}}
	\end{center}
	I want to stand still with the two most important people in my life. My wife Krista and my son Declan. Thank you both for supporting me and enduring me in the time of following classes, conducting research, and writing this thesis. I could not have done this without you. Finally you have your husband and father back. Love you both.
	
	\vspace{\baselineskip}
	\noindent 13 May 2022, René Bliekendaal

}%
%\end{comment}