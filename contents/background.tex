\chapter{Theoretical background}
\label{ch:theoreticalbackground}

\section{What is a system?}
\label{sec:tbsystem}

\subsection{Open vs Closed systems}

Complex adaptive system (CAS)\\

Quote from AMS011: \parencite{Turner2019}\\
''The whole is different from the sum of its parts and their interactions'' [61] (p.77) Though emergence, the whole cannog be reduced to the original parts, the whole is considered a new entity or unit. The whole is ''qualitativly different from their parts ... The cannot be meaningfully compared-they are different'' [61] (system holism)\\
CAS is going against the second law of thermodynamics.\\

\subsection{Linear and non-linear systems}

\subsection{Complexity Theory}
Quote from AMS011:\\
The interactions within organisations are complex and can be explained better through the lens of complexity theory and CAS than by the other theoretical system approaches \parencite[p. 15]{Turner2019}.

\section{Organisation}
\label{sec:tborganisation}

\subsection{Independent Software Vendor}
\label{sub:tbisv}

\section{Antifragile}
\label{sec:tbantifragile}

\begin{itemize}
	\item{Randomness}
	\item{Variability}
	\item{Hormesis / Mithridatisation (by taleb) / Antidotum Mithridatium}
\end{itemize}

\subsection{Relation between antifragile, fragile, robust, resilient, and agile}
\label{sub:tbrelatedtoantifragile}

\gls{antifragile} with \gls{fragile}, \gls{robust}, \gls{resilient}, and \gls{agile}.

\subsection{Volatile, uncertain, complex, and ambiguous}
\label{seb:tbvuca}

\Gls{volatile}, \gls{uncertain}, \gls{complex}, and \gls{ambiguous}.

\section{Enterprise Architecture}
\label{sec:tbea}

\subsection{Steering mechanisms}
\label{sub:tbeasteering}

\section{Public Sector market}
\label{sec:tbpsmarket}

\subsection{Differences with the Private Sector Market}
\label{sub:tbdifferenceprivatesector}

\section{What is a stressor?}
\label{sec:stressor}

