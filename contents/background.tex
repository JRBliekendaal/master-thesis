\chapter{Theoretical background}
\label{ch:theoreticalbackground}

\section{What is a system?}
\label{sec:tbsystem}

\subsection{Open vs Closed systems}

Complex adaptive system (CAS)\\

Quote from AMS011: \parencite{Turner2019}\\
''The whole is different from the sum of its parts and their interactions'' [61] (p.77) Though emergence, the whole cannog be reduced to the original parts, the whole is considered a new entity or unit. The whole is ''qualitativly different from their parts ... The cannot be meaningfully compared-they are different'' [61] (system holism)\\
CAS is going against the second law of thermodynamics.\\

\subsection{Linear and non-linear systems}

\subsection{Complexity Theory}
Quote from AMS011:\\
The interactions within organisations are complex and can be explained better through the lens of complexity theory and CAS than by the other theoretical system approaches \parencite[p. 15]{Turner2019}.

\section{Organisation}
\label{sec:tborganisation}

\subsection{Independent Software Vendor}
\label{sub:tbisv}

\section{Antifragile}
\label{sec:tbantifragile}

\begin{itemize}
	\item{Randomness}
	\item{Variability}
	\item{Hormesis / Mithridatisation (by taleb) / Antidotum Mithridatium}
\end{itemize}

\subsection{Relation between antifragile, fragile, robust, resilient, and agile}
\label{sub:tbrelatedtoantifragile}

\gls{antifragile} with \gls{fragile}, \gls{robust}, \gls{resilient}, and \gls{agile}.

\subsection{Resilience}
\textcite[p. 5-7]{MartinBreen2011} distinguishes three types of resilience:
\begin{itemize}
	\item{\textbf{Engineering Resilience.} Bounce back faster after stress, enduring greater stresses, and being disturbed less by a given amount of stress.}
	\item{\textbf{Systems Resilience.} Maintaining system function in the event of a disturbance.}
	\item{\textbf{Resilience in Complex Adaptive Systems.} The ability to withstand, recover from, and reorganise in repsonse to crisis. The function is maintained by the system structure may not be. The main differentiator is the adaptive capacity or adaptability of the system.}
\end{itemize}
Three key sytems properties contribute to its resilience \parencite[p. 9]{MartinBreen2011}:
\begin{itemize}
	\item{Diversity and Redundancy}
	\item{Modular Networks}
	\item{Responsive, regulatory feedbacks.}
\end{itemize}
For resilience one not only needs to answer the questions ''Resilience of what?'' and ''Resilience to what?'', but also ''Resilience for whom?'' \parencite[p. 21]{Lebel2006}. One can apply basic critical systems design principles to spot ways to maintain any system's function in the event of a crisis \parencite[p. 10]{MartinBreen2011}:
\begin{itemize}
	\item{Maintain a diversity of mechanisms to provide identical functions.}
	\item{Make sure networks (social or otherwise) are modular enough so damange or ''infection'' of one portion does not immediately propogate to all others.}
	\item{Maintain or establish feedbacks to, in the simplest case, establish fail0safe mechanisms in case of malfunction.}
\end{itemize}
One can maximize efficiency over all of these variables; however, such optimisation assumes full working knowledge of the system.
\begin{remark}
Enterprise architecture can be used to give this full working knowledge of the system.
\end{remark}

\subsection{Volatile, uncertain, complex, and ambiguous}
\label{seb:tbvuca}

\Gls{volatile}, \gls{uncertain}, \gls{complex}, and \gls{ambiguous}.

\section{Enterprise Architecture}
\label{sec:tbea}

\subsection{Three schools of Enterprise Architecture}

There are three schools of Enterprise Architecture \parencite{Lapalme2012}:
\begin{itemize}
	\item{\textbf{Enterprise IT Architecting.} Inputs are business strategy and objectives.}
	\item{\textbf{Enterprise Integrating.} It is grounded in systems thinking. It has a holistic view. The link between strategy and execution. Inputs are business strategy and objectives.}
	\item{\textbf{Enterprise Ecological Adaption.} Fostering organisational learning by designing all facets of the enterprise, including the relation to its environment.}
\end{itemize}

\subsection{Steering mechanisms}
\label{sub:tbeasteering}

\section{Public Sector market}
\label{sec:tbpsmarket}

\subsection{Differences with the Private Sector Market}
\label{sub:tbdifferenceprivatesector}

\section{What is a stressor?}
\label{sec:stressor}

