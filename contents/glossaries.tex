% Glossary
\newglossaryentry{ps}
{
	name={public sector},
	description={The Public Sector is comprised of organisations that are owned and operated by the government and exist to provide services for its citizens.}
}
\newglossaryentry{artefact}
{
	name={artefact},
	plural={artefacts},
	description={An artefact is used to describe something artificial, or constructed by humans, as opposed to something that occurs naturally}
}
\newglossaryentry{agile}
{
	name=agile,
	plural={agility},
	description={The ability to adjust before failure happens}
}
\newglossaryentry{agility}
{
	name={agility},
	description={The state of being agile}
}
\newglossaryentry{antifragile}
{
	name={antifragile},
	plural={antifragility},
	description={The ability to strive for and evolve under stress}
}
\newglossaryentry{antifragility}
{
	name={antifragility},
	description={The state of being antifragile}
}
\newglossaryentry{fragile}
{
	name=fragile,
	plural={fragility},
	description={The quality of being easily broken or destroyed}
}
\newglossaryentry{fragility}
{
	name={fragility},
	description={The state of being fragile}
}
\newglossaryentry{resilient}
{
	name=resilient,
	plural={resiliency},
	description={The ability to recover from failure}
}
\newglossaryentry{resiliency}
{
	name={resiliency},
	description={The state of being resilient}
}
\newglossaryentry{robust}
{
	name=robust,
	plural={robustness},
	description={The ability to resist failure}
}
\newglossaryentry{robustness}
{
	name={robustness},
	description={The state of being robust}
}
\newglossaryentry{volatile}
{
	name={volatile},
	description={Likely to change in a very sudden or extreme way}
}
\newglossaryentry{volatility}{
	name={volatility},
	description={The state of being volatile},
}
\newglossaryentry{uncertain}
{
	name=uncertain,
	description={Not known beyond doubt}
}
\newglossaryentry{uncertainty}{
	name={uncertainty},
	description={the state of being uncertain}
}
\newglossaryentry{complex}
{
	name=complex,
	description={A whole made up of complicated or interrelated parts}
}
\newglossaryentry{ambiguous}
{
	name=ambiguous,
	description={Not expressed or understood clearly}
}
\newglossaryentry{stressor}
{
	name={stressor},
	plural={stressors},
	description={When systems are performing effectively, they are in a predetermined condition and conversely when they are not functioning correctly, they are in an unintended state. An unintended condition can be known or unknown. Stressors are forces that threaten to transfer a system from an intended to an unintended condition. In short you can also say that a stressor is an event from outside the system that causes stress}
}
\newglossaryentry{glos_fmea}
{
	name={failure mode effects analysis},
	text={Failure Mode Effects Analysis},
	description={Is a Six Sigma technique that helps manage quality in a system by investigating how the system will cope with failure}
}
\newglossaryentry{entropy}
{
	name={entropy},
	description={The entropy of the universe increases in all natural processes. Isolated systems tend towards greater disorder and entropy is a measure of that disorder}
}
\newglossaryentry{digitaltransformation}{
	name={digital transformation},
	description={Digital Transformation is the application of digital capabilities to processes, products, and assets to improve efficiency, enhance customer value, manage risk, and uncover new monetisation opportunities.}
}
\newglossaryentry{archframework}{
	name={architecture framework},
	description={An enterprise architecture framework (EA framework) defines how to create and use an enterprise architecture. An architecture framework provides principles and practices for creating and using the architecture description of a system. It structures architects' thinking by dividing the architecture description into domains, layers, or views.}
}
\newglossaryentry{immemorial}{
	name={immemorial},
	description={Reaching beyond the limits of memory, tradition, or recorded history}
}
\newglossaryentry{disintermediation}{
	name={disintermediation},
	description={Disintermediation is the process of cutting out one or more middlemen from a transaction, supply chain, or decision-making process}
}
\newglossaryentry{arduous}{
	name={arduous},
	description={Hard to accomplish or achieve}
}
\newglossaryentry{domain}{
	name={domain},
	plural={domains},
	description={A field of action, thought, influence, etc.: The Domain of Science}
}
\newglossaryentry{convex}{
	name={convex},
	description={Being a continuous function or part of a continuous function with the property that a line joining any two points on its graph lies on or above the graph. In the antifragile theory more upside than downside}
}
\newglossaryentry{concave}{
	name={cavex},
	description={Being a continuous function or part of a continuous function with the property that a line joining any two points on its graph lies below the graph. In the antifragile theory more downside than upside}
}
\newglossaryentry{personalmastery}{
	name={personal mastery},
	description={Personal mastery is a discipline of continually clarifying and deepening our personal vision, of focusing our energies, of developing patience, and of seeing reality objectively}
}
\newglossaryentry{sharedmentalmodel}{
	name={shared mental model},
	plural={shared mental models},
	description={Mental models are deeply ingrained assumptions, generalizations, or even pictures of images that influence how we understand the world and how we take action}
}
\newglossaryentry{buildingsharedvision}{
	name={building shared vision},
	description={A practice of unearthing shared pictures of the future that foster genuine commitment and enrollment rather than compliance}
}
\newglossaryentry{teamlearning}{
	name={team learning},
	description={Team learning starts with 'dialogue', the capacity of members of a team to suspend assumptions and enter into genuine 'thinking together'}
}
\newglossaryentry{systemsthinking}{
	name={systems thinking},
	description={a discipline for seeing wholes. It is a framework for seeing inter-relationships rather than things, for seeing patterns of change rather than static snapshots. The fifth discipline of Senge states that it must contain personal mastery, shared mental models, building shared vision, and team learning for a learning organisation.}
}
\newglossaryentry{attenuatevariety}{
	name={attenuate variety},
	description={Dampening or reducing the possible outcomes / states. A light that can be turned on and off has the variety of 2. Your hand during Rock, paper, scissors has the variety or 3}
}
\newglossaryentry{amplifyvariety}{
	name={amplify variety},
	description={Amplifying or increasing the possible outcomes / states. A light that can be turned on and off has the variety of 2. Introducing the possibility of setting the light intensity increases the possible states}
}
\newglossaryentry{resourcestoinvest}{
	name={resources to invest},	
	description={Oportonies can only be seized when there are resources free to do see. This can be money but also time and labour. To Survive a black swan investment should be possible}
}
\newglossaryentry{senecabarbell}{
	name={seneca's barbell},
	text={Seneca's barbell},	
	description={To be antifragile you need a robust sub-system to which 80\%/90\% predictable value with low risk is situated. The 20\%/10\% should be used for high return on investment activities}
}
\newglossaryentry{insertrandomness}{
	name={insert randomness},	
	description={When insert-low-level stress and fail fails delivers no issues the next step is to insert randomness into the systems. A great example of this is Chaos Engineering by Netflix or the HackerOne bug-bounty system}
}
\newglossaryentry{reducenaiveintervention}{
	name={reduce naive intervention},	
	description={Intervention based on a model and reductionistic logic and ignoring the experience. An example is not listening to the experienced but not so articulate employee, or by ignoring the balance nature has found in a ecosystem}
}
\newglossaryentry{skininthegame}{
	name={skin in the game},	
	description={Make certain that the one making the decision and doing the work has a pain and gain relation with the outcome. This goes beyond having a feedback system in place. This is goed beyond having KPI’s in place. An example is that when working Agile scrum, the product owner should be a co-worker in the team for whom the solution is being build}
}
\newglossaryentry{topdowncc}{
	name={top-down dommand \& control},
	text={Top-Down Command \& Control},
	description={Top-Down command and control is in an organisation that a employee is not free to decide to go left or right but has to follow orders. The careful design of an iPhone or a good pen is also an example of limited freedom of movement in the product itself}
}
\newglossaryentry{micromanagement}{
	name={micro-management},
	text={Micro-Management},
	description={Micro-management is about the freedom in the use of the product. When there are minitious working instructions available in a business process the employee has no freedom in the execution of the job. Another great example is a lego building block. It is engineered and fabricated into the greatest detail creating a building block that is almost completely robust. Lego has a very small resilience behaviour through engineering}
}
\newglossaryentry{redundancy}{
	name={redundancy},
	description={Redundancy is about having not a single point of failure by making use of duplication. An example is a backup electricity generators. Another example is local government as backup system of the central government}
}
\newglossaryentry{modularity}{
	name={modularity},
	description={Modularity is the degree that components may be separated and recombined, often with the benefit of flexibility. For example the finance team and the marketing team. Another example is the user-interface module and the data storage module}
}
\newglossaryentry{looselycoupled}{
	name={loosely coupled},
	description={Loosely coupled is the degree of dependency on the exact working of another module. For example when the color-schema of a website is changed it is preferred that this does not impact the functioning of the website. Another example is that when there are new employees introduced at the finance department the taste of the coffee changes. It is important to understand that there is always some degree of coupling}
}
\newglossaryentry{diversity}{
	name={diversity},
	description={Diversity is internally not being a mono-culture and externally having options. For example having two different coffee suppliers. Or having a diverse team}
}
\newglossaryentry{optionality}{
	name={optionality},
	description={Optionality is an idea advanced by Nassim Taleb in his book Antifragile. At the most basic level, optionality just means having lots of options. If you develop a skill with many possible job opportunities, you have more optionality than someone who develops a skill that only has one or two job opportunities}
}
\newglossaryentry{nonmonotonicity}{
	name={non-monotonicity},
	description={Non-monotonicity is about not only learning from the good but als from the bad. For example the lessons learned during a retrospective session}
}
\newglossaryentry{emergence}{
	name={emergence},
	description={Emergence refers to the existence or formation of collective behaviors, what parts of a system do together that they would not do alone}
}
\newglossaryentry{selforganisation}{
	name={self-organisation},
	description={Self-Organisation is a process where some form of overall order arises from local interactions between parts of an initially disordered system. For example students sitting together in the school cafeteria}
}
\newglossaryentry{insertlowlevelstress}{
	name={insert low-level stress},
	description={Continuous Improvement is achieved by inserting low-level of stress continuously into a learning system. This will keep the system sharp all the time}
}
\newglossaryentry{networkconnections}{
	name={network-connections},
	description={A network is created by connections to other nodes. More connections increases potential for optionality for new constructions and also new functionalities}
}
\newglossaryentry{failfast}{
	name={fail-fast},
	text={Fail-Fast},
	description={The attributes ''diversity'', ''non-monotonicity'', ''emergence'', ''self-organisation'', ''insert low-level stress'', and ''network-connections'' combined enables the possibility to execute the strategy to embrace the adagium ''Fail Fast''.}
}
\newglossaryentry{foster}{
	name={foster},
	description={To promote the growth or development of}
}
\newglossaryentry{parliamentaryinquiry}{
	name={parliamentary inquiry},
	plural={parliamentary inquiries},
	description={The parliamentary committee of inquiry is a particular type of temporary committee of the House. The parliamentary inquiry is the most powerful instrument the Dutch parliament has at its disposal to carry out its duty to scrutinize the work of the government}
}
\newglossaryentry{houseofthorbecke}{
	name={the house of thorbecke},
	text={the House of Thorbecke},
	description={In 1848, as minister, Thorbecke laid the foundations for the current administrative division and task demarcation. In 1850 and 1851 he established the Provinces Act and the Municipalities Act. We therefore also speak of 'the House of Thorbecke'}
}
\newglossaryentry{jv}{
	name={joint venture},
	description={A joint venture is a business entity created by two or more parties, generally characterized by shared ownership, shared returns and risks, and shared governance}
}
\newglossaryentry{safeworkingenvironment}{
	name={safe working environment},
	description={When you create a safe work environment for employees, you set yourself up for business success, by reducing problem avoidance, accelerating trouble shooting, and increasing innovation. Taking this approach to errors demonstrates a leader’s acceptance that people need to make mistakes in order to improve so that your business can achieve ever-greater goals}
}
\newglossaryentry{enterpriseitarchitecting}{
	name={enterprise it architecting},
	text={Enterprise IT Architecting},
	description={Enterprise Architecture is the glue between business and IT}
}
\newglossaryentry{enterpriseintegrating}{
	name={enterprise integrating},
	text={Enterprise Integrating},
	description={Enterprise Architecture is the link between strategy and execution}
}
\newglossaryentry{enterpriseecologicaladaptation}{
	name={enterprise ecological adaptation},
	text={Enterprise Ecological Adaptation},
	description={Enterprise Architecture is the means for organizational innovation and sustainability}
}
\newglossaryentry{systeminenvironment}{
	name={systems-in-environment thinking},
	text={Systems-in-Environment thinking},
	description={A system (enterprise) in its environment, including not only the enterprise but also its environment and the bidirectional relationship and transactions between the enterprise and its environment}
}
\newglossaryentry{holisticsystemicstance}{
	name={holistic (systemic) stance},
	description={The EA process must not only think of a single domain but about the combination of domains (IT domains and business domains) together. Addressing any IT and business architecture sub-domains separately and trying to adapt the other sub-domains accordingly will probably produce an ineffective and unsustainable outcome}
}
\newglossaryentry{organisationallearning}{
	name={organisational learning},
	description={To enable innovation and system-in-environment adaptation, Enterprise Architecture is about organisational learning. Designing all facets of the enterprise, including its relationship to the environment, will foster organisational learning}
}
\newglossaryentry{environmentallearning}{
	name={environmental learning},
	description={Use environmental learning to adapt the enterprise desired goals to be more compatible with the environment}
}
\newglossaryentry{intraorganisationalcoherency}{
	name={intra-organisational coherency},
	description={Its possible to make the organisation conducive to ecological learning, environmental influencing, and coherent strategy execution by reinforce wanted intra-dynamics and attenuate unwanted ones}
}
\newglossaryentry{systeminenvironmentcoevolutionlearning}{
	name={system-in-environment coevolution learning},
	description={System-in-environment coevolution is the combination of environmental learning, intra-organisational coherency and attenuating unwanted forces}
}
\newglossaryentry{adapttobusinesslanguage}{
	name={adapt to business language},
	description={Speak the language of your stakeholders such as Directors, Politicians, Public Administrators, and others}
}
\newglossaryentry{complexityscience}{
	name={complexity science},
	text={Complexity Science},
	description={Complexity science is concerned with complex systems and problems that are dynamic, unpredictable and multi-dimensional, consisting of a collection of interconnected relationships and parts. Unlike traditional ''cause and effect'' or linear thinking, complexity science is characterized by non-linearity}
}
\newglossaryentry{blackswan}{
	name={black swan event},
	text={Black Swan event},
	plural={Black Swan events},
	description={A black swan is an unpredictable event that is beyond what is normally expected of a situation and has potentially severe consequences. Black swan events are characterized by their extreme rarity, severe impact, and the widespread insistence they were obvious in hindsight}
}
\newglossaryentry{chiefarchitect}{
	name={chief architect},
	text={Chief Architect},
	description={In information technology, a Chief Architect is a c-level executive whose job is to look closely at how IT functions can be centralized so that departments across the company can work together seamlessly. The chief architect may also be called the Enterprise Architect}
}
\newglossaryentry{fair}{
	name={fair guiding principles},
	text={FAIR Guiding Principles},
	description={The FAIR Guiding Principles provide guidelines to improve the \textbf{F}indability, \textbf{A}ccessibility, \textbf{I}nteroperability, and \textbf{R}euse of digital assets}
}
\newglossaryentry{socialresponsbility}{
	name={social responsibility},
	description={Social responsibility is an ethical framework in which an individual is obligated to work and cooperate with other individuals and organizations for the benefit of the community that will inherit the world that individual leaves behind}
}
\newglossaryentry{openscience}{
	name={open science},
	text={Open Science},
	description={Open science is the movement to make scientific research (including publications, data, physical samples, and software) and its dissemination accessible to all levels of society, amateur or professional}
}
\newglossaryentry{conceptmap}{
	name={concept map},
	description={A concept map or conceptual diagram is a diagram that depicts suggested relationships between concepts. Concept maps may be used by instructional designers, engineers, technical writers, and others to organize and structure knowledge}
}
\newglossaryentry{triangulation}{
	name={triangulation},
	plural={triangulations},
	description={Triangulation means you are seeking convergence and corroboration of results from different methods and designs studying the same phenomenon}
}
\newglossaryentry{attribute}{
	name={attribute},
	plural={attributes},
	description={A quality or characteristic that someone or something has}
}
\newglossaryentry{subsidiarityprinciple}{
	name={subsidiarity principle},
	description={The principle that a central authority should have a subsidiary function, performing only those tasks which cannot be performed at a more local level}
}
\newglossaryentry{outsideincollaboration}{
	name={outside-in and collaboration},
	text={Outside-In and Collaboration},
	description={to be written}
}
\newglossaryentry{datagovernanceplanes}{
	name={data governance planes},
	text={Data Governance Planes},
	description={to be written}
}
\newglossaryentry{agileenterprise}{
	name={agile enterprise},
	text={Agile Enterprise},
	description={to be written}
}
\newglossaryentry{realtimetrust}{
	name={real-time trust},
	text={Real-Time Trust},
	description={to be written}
}
\newglossaryentry{fosterdialogue}{
	name={foster dialogue},
	description={to be written}
}
\newglossaryentry{architecturevalidation}{
	name={architecture validation},
	description={to be written}
}
\newglossaryentry{alwaysfittingarchitecture}{
	name={always fitting enterprise architecture},
	text={Always Fitting Enterprise Architecture},
	description={to be written}
}
\newglossaryentry{specialisation}{
	name={specialisation},
	plural={specialisations},
	description={An element that is a particular kind of another element. E.g. a travel insurance is a specialisation of insurance}
}
\newglossaryentry{engineeringresilience}{
	name={engineering resilience},
	description={Prevent disruption and changes and to bounceback to the fixed function/basis}
}
\newglossaryentry{systemsresilience}{
	name={systems resilience},
	description={The system is able to withstand the impact of any interruption and recuperate while resuming its operations,the function of the system stays the same over time}
}
\newglossaryentry{casresilience}{
	name={complex adaptive systems resilience},
	description={The system is able to become more resilient and to generate new system relationships by reorganisation. The function is maintained, but the system's structure may change. A continuously evolving system}
}
\newglossaryentry{learningorganisation}{
	name={learning organisation},
	description={the learning organisation is a way to create resilient organizations which let them cope with unknown and unpredictable events}
}
\newglossaryentry{cxo}{
	name={cxo},
	text={CxO},
	plural={CxOs},
	description={The generalisation of c-level officers}
}
\newglossaryentry{triad}{
	name={triad},
	description={a group or set of three related people or things},
}