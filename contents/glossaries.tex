% Glossary
\newglossaryentry{ps}
{
	name={public sector},
	description={The Public Sector is comprised of organisations that are owned and operated by the government and exist to provide services for its citizens.}
}
\newglossaryentry{artefact}
{
	name={artefact},
	plural={artefacts},
	description={description}
}
\newglossaryentry{agile}
{
	name=agile,
	plural={agility},
	description={The ability to adjust before failure happens}
}
\newglossaryentry{agility}
{
	name={agility},
	description={The state of being agile}
}
\newglossaryentry{antifragile}
{
	name=antifragile,
	plural={antifragility},
	description={The ability to strive for and evolve under stress}
}
\newglossaryentry{antifragility}
{
	name={antifragility},
	description={The state of being antifragile}
}
\newglossaryentry{fragile}
{
	name=fragile,
	plural={fragility},
	description={The quality of being easily broken or destroyed}
}
\newglossaryentry{fragility}
{
	name={fragility},
	description={The state of being fragile}
}
\newglossaryentry{resilient}
{
	name=resilient,
	plural={resiliency},
	description={The ability to recover from failure}
}
\newglossaryentry{resiliency}
{
	name={resiliency},
	description={The state of being resilient}
}
\newglossaryentry{robust}
{
	name=robust,
	plural={robustness},
	description={The ability to resist failure}
}
\newglossaryentry{robustness}
{
	name={robustness},
	description={The state of being robust}
}
\newglossaryentry{volatile}
{
	name=volatile,
	description={Likely to change in a very sudden or extreme way}
}
\newglossaryentry{uncertain}
{
	name=uncertain,
	description={Not known beyond doubt}
}
\newglossaryentry{complex}
{
	name=complex,
	description={A whole made up of complicated or interrelated parts}
}
\newglossaryentry{ambiguous}
{
	name=ambiguous,
	description={Not expressed or understood clearly}
}
\newglossaryentry{stressor}
{
	name={stressor},
	description={An event from outside the system that causes stress}
}
\newglossaryentry{glos_fmea}
{
	name={Failure Mode Effects Analysis},
	description={Is a Six Sigma technique that helps manage quality in a system by investigating how the system will cope with failure}
}
\newglossaryentry{entropy}
{
	name={entropy},
	description={The entropy of the universe increases in all natural processes. Isolated systems tend towards greater disorder and entropy is a measure of that disorder}
}
\newglossaryentry{digitaltransformation}{
	name={digital transformation},
	description={Digital Transformation is the application of digital capabilities to processes, products, and assets to improve efficiency, enhance customer value, manage risk, and uncover new monetisation opportunities.}
}
\newglossaryentry{archframework}{
	name={architecture framework},
	description={An enterprise architecture framework (EA framework) defines how to create and use an enterprise architecture. An architecture framework provides principles and practices for creating and using the architecture description of a system. It structures architects' thinking by dividing the architecture description into domains, layers, or views.}
}
\newglossaryentry{immemorial}{
	name={immemorial},
	description={Reaching beyond the limits of memory, tradition, or recorded history}
}
\newglossaryentry{disintermediation}{
	name={disintermediation},
	description={Disintermediation is the process of cutting out one or more middlemen from a transaction, supply chain, or decision-making process}
}
\newglossaryentry{arduous}{
	name={arduous},
	description={Hard to accomplish or achieve}
}
\newglossaryentry{domain}{
	name={domain},
	plural={domains},
	description={A field of action, thought, influence, etc.: The Domain of Science}
}
\newglossaryentry{convex}{
	name={convex},
	description={Being a continuous function or part of a continuous function with the property that a line joining any two points on its graph lies on or above the graph}
}
\newglossaryentry{concave}{
	name={cavex},
	description={Being a continuous function or part of a continuous function with the property that a line joining any two points on its graph lies below the graph}
}
