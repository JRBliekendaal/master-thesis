\chapter{Overview of Laws}
\setcounter{footnote}{0}
The research references to several laws. This appendice gives a small explainatory overview of these laws.

\begin{itemize}
	\item{2\textsuperscript{nd} Law of Thermodynamics}
	\item{Conways Law}
	\item{Metcalfe's Law}
	\item{Law of Municipalities}
	\item{Lehmans Law of Increasing Complexity}
\end{itemize}

\section{2\textsuperscript{nd} Law of Thermodynamics}
\label{sec:appendix2ndlawthermodynamics}
The ‘2\textsuperscript{nd} Law’ was formulated after nineteenth century engineers noticed that heat cannot pass from a colder body to a warmer body by itself. It states that in any closed system the amount of order can never increase, only decrease over time. Another way of saying this is that entropy always increases.

\section{Conway's Law}
Any organization that designs a system (defined broadly) will produce a design whose structure is a copy of the organization's communication structure.

\section{Metcalfe's Law}
Metcalfe's Law\footnote{\url{https://www.techopedia.com/definition/29066/metcalfes-law}} states that a network's impact is the square of the number of nodes in the network. For example, if a network has 10 nodes, its inherent value is 100 (10 * 10). The end nodes can be computers, servers and/or connecting users.

\section{Thorbecke's Law}

\section{Lehman's Law of Increasing Complexity}
As an evolving program is continually changed, its complexity, reflecting deteriorating structure, increases unless work is done to maintain or reduce it.

