\chapter{Research Methodology}
\label{ch:research-methodology}

\section{Research Model}
\label{sec:research-model}
The method of \textcite{Verschuren2016} is used for the research model.
	\begin{figure}[h]
		\centering
		\includegraphics[width=12cm]{images/research-model.png}
		\caption[Research Model]{Research Model}
		\label{fig:research-model}
	\end{figure}

In the first phase of research (a), the researcher executes preliminary research and studies different theories and definitions of the involved concepts. The output of the first phase is the definitions and theories relevant to this research, such as \gls{antifragile}, \acrlong{ea}, the Public Sector market, and \acrshort{vuca}. In the second phase of research (b), the researcher confronts \gls{antifragile} with \acrlong{ea} and the Public Sector market with \acrshort{vuca}. The researcher uses interviews to validate the confrontation between the Public Sector Market with \acrshort{vuca}. The outcome of the second phase is the initiation of analysis on success factors of \acrlong{ea} relevant for contribution to \gls{antifragile} and analysis on attributes of the public sector market influenced by \acrshort{vuca} (c). In the fourth phase (d), the researcher uses the output of the analysis to confront the success factors with validation artefact through the Delphi Method to conclude and discuss his research (e).
\section{Research quality}
\label{sec:researchquality}
The researcher uses three frameworks to increase the rigorousness of the research as much as possible.
\begin{itemize}
	\item{Quality Principles of \textcite{Recker2013} (subsection \ref{sub:recker}).}
	\item{The FAIR Principles from Scientific Data (subsection \ref{sub:fair}).}
	\item{The Open Science Framework (subsection \ref{sub:osf}).}
\end{itemize}
\subsection{Quality Principles of Recker}
\label{sub:recker}
The first framework is that of \textcite[p. 16-17]{Recker2013} who uses four important principles:
\begin{itemize}
	\item{\textbf{Replicability} is a term that characterises the extent to which research procedures are repeatable. The principle states that the procedures by which research outputs are created should be conducted and documented in a manner that allows others outside the research team to independently repeat the procedures and obtain similar, if not identical, results.}
	\item{\textbf{Independence} is closely related to reliability. It concerns the extent to which the research conduct is impartial and freed from any subjective judgment or other bias stemming from the researcher or research team itself.}
	\item{\textbf{Precision} states that in all scientific research the concepts, constructs, and measurements should be as carefully and precisely defined as possible to allow others to use, apply, and challenge the definitions, concepts, and results in their own work.}
	\item{\textbf{Falsification} describes the logical possibility than an assertion, hypothesis, or theory can be contradicted by an observation or other outcome of a scientific study or experiment.}
\end{itemize}
\begin{remark}
	Howto falsify? 
\end{remark}
\subsection{Fair Principles}
\label{sub:fair}
In 2016, the 'FAIR Guiding Principles for scientific data management and stewardship' were published in Scientific Data. The authors intended to provide guidelines to improve the Findability, Accessibility, Interoperability, and Reuse of digital assets. The research is using the FAIR Principles\footnote{\url{https://www.go-fair.org/fair-principles/}} to increase the quality of the published thesis.
\begin{itemize}
	\item{\textbf{Findable.} The first step in (re)using data is to find them. Metadata and data should be easy to find for both humans and computers. Machine-readable metadata are essential for automatic discovery of datasets and services. The thesis, research and used datasets are containing keywords, links, and structures that can be indexed.}
	\item{\textbf{Accesible.} Once the user finds the required data, she/he/they need to know how can they be accessed. The thesis, research and used datasets are published on GitHub, Zenodo, and Researchgate based on Open Access. The researcher created objects containing a location on where the data can be acquired if it cannot be published because of author rights.}
	\item{\textbf{Interoperable.} The data usually need to be integrated with other data. In addition, the data need to interoperate with applications or workflows for analysis, storage, and processing. This principle is not relevant for this research. The data are qualitative data sets based on literature, interviews, and questionnaires.}
	\item{\textbf{Reusable.} The ultimate goal of FAIR is to optimise the reuse of data. To achieve this, metadata and data should be well-described so that they can be replicated and/or combined in different settings. The thesis, research and used datasets are published under the \href{https://creativecommons.org/licenses/by-sa/4.0/}{\ccbysa\ CC-BY-SA 4.0 license.} It is allowed that the thesis, research, and datasets are shared and are adapted (even commercially) as long as the original author is attributed and the possible derivate is published under the same license.}
\end{itemize}
\subsection{The Open Science Framework}
\label{sub:osf}
One of the starting points of the research is Open Science. The idea behind Open Science is to allow scientific information, data and outputs to be more widely accessible (Open Access) and more reliably harnessed (Open Data) with the active engagement of all the stakeholders (Open to Society) \parencite{UNESCO2020}. The Center for Open Science\footnote{{\url{https://www.cos.io/}}} supports this way of research by supplying guidelines and even a toolkit. For this research the toolkit is used to support Open Access, Open Data and Open to Society. One of the tools in the toolkit is a reference model to select tools for the four main phases of research: Search and Discover, Design Study, Collect and Analyse Data, and Publish Reports. The researcher uses this reference model in section \ref{sec:researchinfraandtooling}. Using this framework will help in achieving replicability, precision, and reusability.
\section{Research approach}
\label{sec:researchapproach}
In this section, the researcher describes the approach of the research. This description helps to increase replicability, independence, and reusability. For this research approach, the researcher follows the research model (figure \ref{fig:research-model}) and the research (sub)questions (section \ref{sec:researchsubject}). The research model contains five phases in the research. The five phases are used to describe the research approach. The five phases are (a) Desk research, (b) Confrontation, (c) Analysis, (d) Validation, and (e) Conclusion and discussions.

\subsection{Desk research}
\label{sub:deskresearchphase}
The first phase of the research model emphasises desk research on the relevant concepts, theories and definitions. Desk research is conducted based on a literature study. The main concepts of \gls{antifragile}, \acrshort{ea}, \acrshort{vuca}, and the public sector are studied. This first phase (a) will answer the sub-questions of:
\begin{itemize}
	\item{What is literature saying about \gls{antifragile}?}
	\item{What is literature saying about the Public Sector Market?}
	\item{What is literature saying about \acrlong{ea}?}
	\item{What is literature saying about the success factors of Enterprise Architecture?}
\end{itemize}

\subsubsection{Literature research}
For the literature research two primary methods are used. The first method is (foward and backward) snowballing of already acquired literature. The second method is the use of online scientific libraries.

\begin{figure}[H]
	\centering
	\includegraphics[width=0.6\linewidth]{images/snowball}
	\caption[Snowballing literature]{Snowballing literature \parencite{Botjes2021a}}
	\label{fig:snowball}
\end{figure}

For finding relevant literature that was created after the start of the thesis of \citeauthor{Botjes2020} online scientific libraries are used. The online scientific libraries are Web of Science, Research Gate, and Google Scholar. The full concept name is used and the known abbreviations of the concept (e.g. Enterprise Architecture and EA). The list of abbreviations contains the used abbreviations. Literature is only accepted if the literature complies with quality attributes. These attributes are accuracy, authority, objectivity, currency, and coverage\footnote{\url{https://libguides.library.cityu.edu.hk/litreview/evaluating-sources/}}. All found literature is administrated for replicability, independence, precision, accessibility, and reusability. Section \ref{sub:tbresearchexecution} describes how literature registration and administration is executed.

\subsubsection{Antifragile}
\label{subsub:antifragile}
The literature study on \gls{antifragile} makes use of four primary sources. The first primary source is the book ''\Gls{antifragile}: Things that gain from disorder'' \parencite{Taleb2012}. \textcite{Taleb2012} is the progenitor of the \gls{antifragile} theory. The second primary source is the master thesis ''Defining \Gls{antifragility} and the application on Organisation Design" \parencite{Botjes2020}. \citeauthor{Botjes2020} studied the literature, extensively, in the field of \gls{antifragile} and the application in the context of an organisation. By using the thesis of \citeauthor{Botjes2020} the literature study of this study concentrates on the literature after 2018. The last two primary resources are the articles ''No More Snake Oil: Architecting \Gls{agility} through \Gls{antifragility}'' and ''The Philosophy of Residuality Theory'' \parencite{OReilly2019,OReilly2021}. \textcite{Botjes2020} did not use the articles of \citeauthor{OReilly2019}. The theories of \citeauthor{OReilly2019} were less of interest for the subject of \citeauthor{Botjes2020}. While for this research the Residuality Theory of \textcite{OReilly2021} has added value since it targets system architecture.

\begin{remark}
	Need to add second book from Taleb (Black Swan) since Antifragile is an answer to black swan events.\\
	Need to add book of Hole as it is one of the sources referenced by many.
\end{remark}

The first method for literature study is snowballing. Snowballing of these sources is used to determine other important literature on \gls{antifragile}. Forward snowballing is used for the source of \citeauthor{Taleb2012}. Since \citeauthor{Taleb2012} is the progenitor, it is not necessary to do a backward snowballing. Backward snowballing is used for the sources from \citeauthor{Botjes2020} and \citeauthor{OReilly2019}.

The second method for literature study is the use of online scientific libraries. For these libraries the following set of keywords or key sentences are used.
\bigskip

\begin{table}[H]
	\centering
\begin{tabular}{p{0.4\textwidth}p{0.4\textwidth}}
	\toprule
	\gls{antifragile}	& \gls{antifragile} \gls{robust} \gls{resilient} \gls{agile}\\%
	\gls{antifragile} \acrlong{ea}	& \gls{antifragile} public sector\\%
	\gls{antifragile} success factors & residuality theory\\%
	\gls{antifragile} residuality theory & \acrlong{vuca} \\%
	\gls{antifragile} system & \\%
	\bottomrule
\end{tabular}
	\caption{Antifragile keywords}
\end{table}

\subsubsection{Enterprise Architecture}
\label{subsub:enterprisearchitecture}
The literature study on \acrshort{ea} makes use of three primary sources. \textcite{Greefhorst2011} is a book on the theory on steering mechanisms of \acrshort{ea} with the emphasis on the use of principles. \textcite{Ross2014} is the most referenced book on the subject \acrshort{ea} and Business Strategy. The last one is the article of \textcite{Lapalme2012} who researched the different manifestations of \acrshort{ea}.

The first method is snowballing. All three sources will be used for forward and backward snowballing. The second method for literature study is the use of online scientific libraries. For these libraries the following set of keywords and key sentences are used:
\bigskip

\begin{table}[H]
	\centering
\begin{tabular}{p{0.4\textwidth}p{0.4\textwidth}}
	\toprule
	\acrlong{ea} & \acrlong{ea} sucess factors\\%
	\acrlong{ea} \gls{antifragile} system	& \acrlong{ea} steering mechanism\\%
	intentional emergent \acrlong{ea} & \acrlong{ea} Business Strategy\\%
	\bottomrule
\end{tabular}
	\caption{Enterprise Architecture keywords}
\end{table}


\subsubsection{Public Sector}
\label{subsub:publicsector}
The literature study on public sector makes use of one primary source. \textcite{Wal2008} is an article on the differences between the public and private sector based on the core values of these sectors. This article is used for forward and backward snowballing. The last method for literature study is the use of online scientific libraries. For these libraries the following set of keywords and key sentences are used:
\bigskip

\begin{table}[H]
	\centering
	\begin{tabular}{p{0.4\textwidth}p{0.4\textwidth}}
		\toprule
		Difference public and private sector &	Public Sector \gls{antifragile}\\%
		Collaboration public and private sector & Public Sector \gls{resilient}\\%
		Public Sector \acrshort{vuca} & \\%
		\bottomrule
	\end{tabular}
	\caption{Public sector keywords}
\end{table}

\begin{remark}
	The preliminary research on the topic public sector is not started yet. Maybe some primary sources will emerge.
\end{remark}

\subsection{Confrontation}
\label{sub:confrontationphase}

\begin{remark}
	What is the model for confrontation?
	I have to dtermine the lens I am going to use.
\end{remark}


The second phase (b) 

\subsection{Analysis}
\label{sub:analysisphase}

\begin{remark}
	What is the model for Analysis?
	I have to determine the lens I am going to use.
\end{remark}

The third phase (c)

How can the success factors of \acrlong{ea} contribute to becoming antifragile?

\subsection{Validation}
\label{sub:validatinphase}
The fourth phase (d) analyses the outcome of the analysis phase. This outcome is the answer to the sub-question ''How can the success factors of \acrlong{ea} contribute to becoming antifragile?'' This answer needs to be validated. The answer is validated by the means of an artefact. The artefact is validated with the Delphi Method. \needsref

\subsubsection{Delphi Method}
\label{subsub:delphimethod}
The Delphi method is an iterative process to collect and distil the anonymous judgments of experts using a series of data collection and analysis techniques interspersed with feedback. The Delphi method is well suited as a research instrument when incomplete knowledge about a problem or phenomenon. The Delphi method evolved into a flexible research method appropriate for many \acrfull{is} research projects, such as determining the criteria for \acrshort{is} prototyping decisions, ranking technology management issues in new product development projects, and developing a descriptive framework of knowledge manipulation
activities. The Delphi method is a flexible, effective and efficient research method that can be successfully used by \acrshort{is} graduate students to answer research questions in \acrshort{is} and to advance the \acrshort{is} Body of Knowledge rigorously. \parencite{Skulmoski2007}

The group participants are mutually unknown, only the researcher knows the participants. When it cannot be proven that the artefact is incorrect, it must be correct. This method is the principle of falsification. To reach a consensus, the researcher uses questionnaires. To reach a consensus, the researcher works iteratively and adjusts the artefact. The researcher expects consensus on the artefact after two to six rounds of questionnaires. The goal of the Delphi Rounds is that it cannot be proven that the artefact is incorrect. This method is the principle of falsification (subsection \ref{sub:recker}). However, when is there a consensus? \textcite[p. 404]{Diamond2014} concludes in his research for over more than 100 cases that the median of the percentage of consensus 75\% is. The researcher states, as a result of the research of \textcite{Diamond2014}, that consensus is reached with the threshold of 75\%. The researcher states with some degree of certainty that the artefact is correct with a consensus of 75\%.

The researcher defined domains for the group composition based on the context of the research. These domains are \acrfull{isv}, Municipality, National Government, VNG-Realisatie (the association of Dutch municipalities), and Academics. Participants are members of one or more of these domains and have an affinity with Enterprise Architecture and the public sector. The researcher invited at least three participants per domain (n=3). The result is a total population of at least fifteen (n=15). The approach followed \textcite{Denzin2017} multiple triangulation approach, which encourages several methods to collect data and multiple investigators with varied expertise.

For the Delphi Group composition domains are defined based on the context of the research. These domains are \acrfull{isv}, Municipality, National Government, VNG-Realisatie (the association of Dutch municipalities), and Academics. The participants have affinity with \acrshort{ea}. The participants validate the artefact their context and domain.

Meeting Wizard is the service for sending out the questionnaires and execute the analysis of the outcome of the questionnaires. The participants get an invite by email to fill in the questionnaires. The researcher analyses the results after every round and communicates the outcome as soon as a consensus is reached.

\subsection{Conclusion and discussion phase}
\label{sub:conclusionanddiscussinophase}
The fifth phase (e)

What are the success factors of \acrlong{ea} for \gls{antifragility} in the Public Sector?

\section{Research type}

\begin{remark}
	Qualitative vs Quantitative! (use \parencite{Recker2013})
\end{remark}


\section{Research infrastructure and tooling}
\label{sec:researchinfraandtooling}
\begin{wrapfigure}{R}{0.5\textwidth}
	\begin{center}
		\includegraphics[width=0.5\linewidth]{images/osfframework}
		\caption[Open Science Framework]{Open Science Framework}
		\label{fig:osfframework}
	\end{center}
\end{wrapfigure}
For selecting the suitable instruments for the research, the Open Science Framework\footnote{\url{https://www.cos.io/products/osf}} is used. The Open Science Framework consists out of 4 stages in a research project. Those stages are: ''Search and Discover, Design Study, Collect and Analyse, and Publish Reports.'' The Open Science Framework proposes specific infrastructure and tools per stage. The transparency in the used infrastructure and tools increases the quality of the research. It increases the replication factor, findability, accessibility, interoperability, and reusability.
\subsection{Thesis creation}
\label{sub:tbresearchcreation}
The researcher used his corporate laptop (Dell Latitude 7200 2-in-1\footnote{\url{https://www.dell.com/en-us/work/shop/dell-laptops-and-notebooks/latitude-7200-2-in-1-laptop/spd/latitude-12-7200-2-in-1-laptop}}) with Windows 10 Professional installed for creating the thesis. The thesis is created with the markup language \LaTeX\footnote{\url{https://www.latex-project.org/}}. The used typesetting environment is TexLive\footnote{\url{https://www.tug.org/texlive/}} with the document type of ''Report'' from KOMA-Script\footnote{\url{https://ctan.org/pkg/koma-script}}. TexStudio\footnote{\url{https://www.texstudio.org/}} is the used \LaTeX\ Editor. It supports syntax-highlighting, has an integrated viewer, reference checking and numerous wizards. For the creation and administration of references Bib\LaTeX\footnote{\url{https://ctan.org/pkg/biblatex/}} is used with the reference manager JabRef\footnote{\url{https://www.jabref.org/}} with the citation style of APA 7th Edition\footnote{\url{https://apastyle.apa.org/}} and with web browser integration. The files are stored on a personal Dropbox\footnote{\url{https://www.dropbox.com/}} that is used by GitHub Desktop\footnote{\url{https://desktop.github.com/}} to synchronise with a public GitHub repository\footnote{\url{https://github.com/JRBliekendaal/master-thesis}}. GitHub\footnote{\url{https://github.com/}} is used for source control but also for reviewing and discussing the topics with the (Co-)Promotor and the planning of the master thesis project. The thesis source files are copied to an Amazon S3 Blob\footnote{\url{https://aws.amazon.com/s3/}} for backup. The backup rotation is seven versions. Cloudberry Explorer Freeware for Amazon S3\footnote{\url{https://www.msp360.com/explorer/windows/amazon-s3.aspx}} is used for backup. Grammarly\footnote{\url{https://www.grammarly.com}}, with the paid subscription service, checks the thesis for spelling, grammar,  style, and plagiarism. The used goals for Grammarly are audience=knowledgeable, formality=formal, and domain=academic. Microsoft Visio Professional\footnote{\url{https://www.microsoft.com/en-ww/microsoft-365/visio/}} is used to create figures. The GitHub repository contains all the sources.
\subsection{Research administration}
\label{sub:tbresearchadministration}
The research administration, which includes documentation containing privacy-sensitive information, like the name and contact information of the Delphi Group participants, is stored on a non-public GitHub Repository\footnote{\url{https://github.com/JRBliekendaal/master-thesis-administration}}. The private GitHub Repository is also for staging thesis parts that still need to be anonymised. For taking notes Leuchtturm1917\footnote{\url{https://www.leuchtturm1917.us/notebook-classic.html}} Notebooks are used with mechanical pencils of Faber-Castell\footnote{\url{https://www.fabercastell.com/products/tk-fine-vario-l-mechanical-pencil-10mm-135900}} and pens from Sakura\footnote{\url{https://www.sakuraofamerica.com/product/pigma-micron/}} with long-lasting ink.
\subsection{Research execution}
\label{sub:tbresearchexecution}
For the execution of the research, Microsoft Excel\footnote{\url{https://www.microsoft.com/en-us/microsoft-365/excel}} is used for the administration of the literature research. For the administration of the literature research, the following headers are used: ID (for a unique ID per item), search terms used, scope, title, subtitle, author(s), year, type, Bib\LaTeX\ citation key, title relevance, abstract relevance, content relevance, found at, doi/isbn, url, date found, duplicate, date used, use for, and notes. Researchgate\footnote{\url{https://www.researchgate.net/}}, Web of Science\footnote{\url{https://app.webofknowledge.com/}}, and Google Scholar\footnote{\url{https://scholar.google.com/}} are the main sources for searching for literature. PaperPanda\footnote{\url{https://paperpanda.app/}} is used for hard to find literature. The literature administration is, together with the publicly available literature, stored in the repository of the master thesis. For non-public available literature, the administration contains the location where the literature is retrievable. All the literature is added to a bib\LaTeX\ file for future reference. For traceability the entries in the bib\LaTeX\ file contain the Unique ID in the notes field. JabRef is used to sort the references by using subgroups to support the workflow. The subgroups used are: ''evaluate, rejected, and used.'' Only the literature in the subgroup used are transferred to the bibliography file of the thesis. This prevents cluttering. For working as paperless as possible all the literature, where possible, is in pdf or in ebook format. For reading Acrobat Reader DC\footnote{\url{https://get.adobe.com/reader/}} is used for reading the PDF, and an Amazon Kindle Oasis\footnote{\url{https://www.amazon.com/dp/B07L5GJD99}} for eBooks. With the Amazon Kindle the highlight feature is used. This is not stored on GitHub since the highlights are under copyright of the author(s).\par
For the execution of the Delphi Method, Meetingwizard\footnote{\url{https://www.meetingwizard.nl/}} is used for questionnaires and the analysis of the questionnaires. The license for using Meeting Wizard is supplied by the Antwerp Management School.

\subsection{Summary of used infrastructure and tooling}

\begin{table}[!h]
	\begin{center}
		\begin{tabular}{@{}cccc@{}}
			\toprule
			Search \& Discover & Design Study & Collect \& Analyse Data & Publish Reports\\ \midrule
			Web of Science & 1    & JabRef   & \LaTeX \\%
			ResearchGate   &      &           & TeXstudio \\%
			Google Scholar & 2    & PaperPanda  & ORCID \\%
			Z	 & 0    & bib\LaTeX   & ResearchGate \\%
			Z	 &	x	& Meetingwizard	  & Zenodo \\%
			Z	 &	x	& Microsoft Excel  & Grammarly \\%
			Y    & 2    & GitHub  & Microsoft Visio \\%
			Y	 & 2	& Cloud Berry Explorer for S3 & \\%
			\bottomrule
		\end{tabular}
		\caption{Used infrastructure \& tooling}
		\label{tab:usedinfrastructuretooling}
	\end{center}
\end{table}

\begin{remark}
	This section needs a tabel to summarise the used tools in relation to the OSF framework.
\end{remark}