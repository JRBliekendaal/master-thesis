%\subsection{Interview CIO of central government}
\subsection{Interview central government}
\label{sub:interviewciocentral}
% March 18th, 2022 12:00 - 12:30 with Cisco WebEx recording not allowed. Direct Quotes also not allowed. Anonymous.
\subsubsection{Question 1 / Enterprise Architecture}
\acrshort{ea} is not used and we are not \gls{agile}. \acrshort{ea} is too difficult for the public administrators. In addition, we are also responsible for other sectors. There is not one architecture. We have multiple reference architectures. What we have to do in the public sector depends on the political decision making within the period of governing (four years until new elections). \acrshort{ea} is at the end of the chain of administrative decision-making.
\subsubsection{Question 2 / Agility of the public sector}
It is hard to be \gls{agile} within the \gls{ps}. Everything needs to be predefined and planned. \Gls{agile} working is very difficult within the government. The end goal is not very clear with \gls{agile} working. It is unclear how the public money is spent on precisely what. 
\subsubsection{Question 3 / Dealing with uncertainty and unexpected events}
The \gls{ps} cannot deal with \gls{uncertainty}. Everything must be predefined and planned. There must be accountability for how public money is spent. All missteps are magnified. There is a quick result in crises, but with possible consequences later on because of \acrfull{bit} audits or \glspl{parliamentaryinquiry}.
\subsubsection{Question 5 / The risk appetite of the public sector}
There is no risk appetite. Everything must be known and explainable in advance. If it is found that the procedures are not used, it can result in political consequences later on. Afterwards, positive lessons learned are not used to make adjustments within the public sector. Experimentation is (almost) not possible.
%(note: blaming culture)
\subsubsection{Question 6 / Using diversity and optionality in the public sector}
It would be nice to work with optionality and smaller units within the \gls{ps} and \acrshort{ea} to make it easier to adapt. Think about in \acrshort{ea} disposable microservices. Nevertheless, it remains that \acrshort{ea} is not important. It is at the end of the chain and is not used in administrative decision-making. Enterprise architecture is confronted with decision-making.
\subsubsection{Closing statements}
\Gls{antifragile} is not directly applicable for the central government, but it can have a lot of benefits for suppliers in the \gls{ps}. Do not advertise it but exploit it to become better. In the case of an \acrfull{isv} think about many disposable microservices so it will be easier to deal with the \gls{ps}.