\appendix
\chapter{Properties of the Enterprise Architecture schools of thought}
\label{app:easchoolsproperties}
This appendix describes the Enterprise Architecture school of thoughts in more details. It will help the reader with detailed understanding of the three schools of thought.
\section{The properties of Enterprise IT Architecting}
\label{sec:propenterpriseitarchitecting}
The school of thought \gls{enterpriseitarchitecting} \parencite[p. 39]{Lapalme2012} is summarised in the following table.
\begin{table}[!h]
	\begin{center}
		\resizebox{\textwidth}{!}{%
			\begin{tabular}{@{}p{.2\textwidth}p{.8\textwidth}@{}}
				\toprule%
				& \textbf{\gls{enterpriseitarchitecting} school of thought} \\%
				\midrule%
				Motto    		& Enterprise architecture is the glue between business \& IT \\
				Objectives and 	& Effectively enable the enterprise strategy \\
				concerns		& Support IT planning and reduce cost \\
				& Enable business \\
				Principles and  & Apply reductionist (mechanistic) stance \\
				assumptions		& Don't question business strategies  \\
				& Design organisational dimensions independently \\
				& Don't worry about non-IT dimensions; they are not your concerns \\
				Skills 			& Have technical competence and engineering knowledge \\
				Challenges		& Convince the organisation to accept the designed plans \\
				Insights		& Permits the design of robust and complex technological solutions \\
				& Fosters the creation of high-quality models and planning scenarios \\
				Limitation		& Can produce inadequate or unfeasible solutions for the larger organizational context \\
				& Struggles with solution acceptance and implementation barriers \\
				& Susceptible to “perfect” designs 	that support unsustainable strategies \\
				\bottomrule%
			\end{tabular}
		}
		\caption{Properties of \gls{enterpriseitarchitecting}}
		\label{tab:enterpriseitarchitecting}	
	\end{center}
\end{table}
\newpage
\section{The properties of Enterprise Integrating}
\label{sec:propenterpriseintegrating}
The school of thought \gls{enterpriseintegrating} is summarised in the following table.
\begin{table}[!h]
	\begin{center}
		\resizebox{\textwidth}{!}{%
			\begin{tabular}{@{}p{.2\textwidth}p{.8\textwidth}@{}}
				\toprule%
				& \textbf{\gls{enterpriseintegrating} school of thought} \\%
				\midrule%
				Motto    		& Enterprise Architecture is the link between strategy and execution \\
				Objectives and 	& Effectively implement the enterprise strategy \\
				concerns		& Support organizational coherence \\
				Principles and  & Apply a holist (systemic) stance \\
				assumptions		& Don’t question business strategies and objectives  \\
				& Manage the environment \\
				& Jointly design all organisational dimensions \\
				Skills 			& Facilitate small-group collaboration \\
				& Apply systems thinking \\
				Challenges		& Understand organizational systemic dynamics \\
				& Collaborate across the organization \\
				& Encourage systems thinking and paradigm shifts \\
				Insights		& Permits the design of comprehensive solutions \\
				& Enables significant organizational efficiency by eliminating unnecessary contradictions and paradoxes \\
				Limitation		& Susceptible to “perfect” designs that support unsustainable strategies \\
				& Requires a paradigm shift from reductionism to holism \\
				\bottomrule%
			\end{tabular}
		}
		\caption{Properties of \gls{enterpriseintegrating}}
		\label{tab:interpriseintegrating}	
	\end{center}
\end{table}
\newpage
\section{The properties of Enterprise Ecological Adaptation}
\label{sec:propeea}
The school of thought \acrlong{eea} is sumarised in the following table.
\begin{table}[!h]
	\begin{center}
		\resizebox{\textwidth}{!}{%
			\begin{tabular}{@{}p{.2\textwidth}p{.8\textwidth}@{}}
				\toprule%
				& \textbf{\gls{enterpriseecologicaladaptation} school of thought} \\%
				\midrule%
				Motto    		& \acrshort{ea} is the means for organizational innovation and sustainability \\
				Objectives and 	& Innovate and adapt    \\
				concerns		& Support organizational coherence \\
				& Encourage \acrlong{sie} coevolution \\
				Principles and  & Apply a holist (systemic) stance \\
				assumptions		& \acrlong{sie} coevolution  \\
				& Environment can be changed \\
				& Jointly design all organisational dimensions \\
				Skills 			& Foster dialogue \\
				& Apply system and \acrlong{sie} thinking \\
				& Facilitate larger-group collaboration \\
				Challenges		& Foster sensemaking \\
				& Encourage systems thinking and \acrlong{sie} paradigm shifts \\
				& Collaborate across the organisation \\
				Insights		& Fosters \acrlong{sie} coevolution and enterprise choherency \\
				& Fosters organisational innovation and sustainability \\
				Limitation		& Requires many organisational preconditions for management and strategy creation \\
				\bottomrule%
			\end{tabular}
		}
		\caption{Properties of \gls{enterpriseecologicaladaptation}}
		\label{tab:eaeea}	
	\end{center}
\end{table}