%\appendix
\chapter{Interview summaries}
\label{app:interviewsummaries}
This appendix contains summaries per interview. This appendix gives the reader of this thesis more details on the answers given by the interviewees. These summaries are created from the recorded interviews and transcriptions belonging to these recordings.

\section{Interview central government}
\label{sec:interviewcentralgovernment}
\subsubsection{Question 1 / Enterprise Architecture}
\acrshort{ea} is not used and we are not \gls{agile}. \acrshort{ea} is too difficult for the public administrators. In addition, we are also responsible for other sectors. There is not one architecture. We have multiple reference architectures. What we have to do in the public sector depends on the political decision making within the period of governing (four years until new elections). \acrshort{ea} is at the end of the chain of administrative decision-making.
\subsubsection{Question 2 / Agility of the public sector}
It is hard to be \gls{agile} within the \gls{ps}. Everything needs to be predefined and planned. \Gls{agile} working is very difficult within the government. The end goal is not very clear with \gls{agile} working. It is unclear how the public money is spent on precisely what. 
\subsubsection{Question 3 / Dealing with uncertainty and unexpected events}
The \gls{ps} cannot deal with \gls{uncertainty}. Everything must be predefined and planned. There must be accountability for how public money is spent. All missteps are magnified. There is a quick result in crises, but with possible consequences later on because of \acrfull{bit} audits or  \glspl{parliamentaryinquiry}.
\subsubsection{Question 5 / The risk appetite of the public sector}
There is no risk appetite. Everything must be known and explainable in advance. If it is found that the procedures are not used, it can result in political consequences later on. Afterwards, positive lessons learned are not used to make adjustments within the public sector. Experimentation is (almost) not possible.
(note: blaming culture)
\subsubsection{Question 6 / The use of diversity and optionality in the public sector}
It would be nice to work with optionality and smaller units within the \gls{ps} and \acrshort{ea} to make it easier to adapt. Think about in \acrshort{ea} disposable microservices. Nevertheless, it remains that \acrshort{ea} is not important. It is at the end of the chain and is not used in administrative decision-making. Enterprise architecture is confronted with decision-making.
\subsubsection{Closing statements}
\Gls{antifragile} is not directly applicable for the central government, but it can have a lot of benefits for suppliers in the \gls{ps}. Do not advertise it but exploit it to become better. In the case of an \acrfull{isv} think about many disposable microservices so it will be easier to deal with the \gls{ps}.
\section{Interview local government}
\label{sec:interviewlocalgovernment}
\subsubsection{Question 1 / Enterprise Architecture}
There is somewhat of an \acrshort{ea}, but we are not using it that broadly. As an organisation, we do not have a real \acrshort{ea}. Our organisation is best compared to that of a consultancy firm. Our core task is lobbying advocacy. We guide the things we do, and then again, it concerns the things we do for municipalities or on behalf of municipalities. We have a multi-year vision. We use guiding principles for the things we do.
Nevertheless, there are processes in the making for portfolio management. What do we do, what don't we do, and how do they relate to each other. There is not one responsibility on the \acrshort{ea}. It is a stepped responsibility that lies with committees and the services board. When it comes to IT, the responsibility lies with the Directory of Information Society. \acrshort{ea} is used for assay the request for subsidy. When there is a request for a subsidy that is not in line with the goals, it is not requested by the (European) government.
The case of how \acrshort{ea} contributes to the agility of the public sector is complicated. The public administrators are not architects and vice versa. \acrshort{ea} is hard to understand. \acrshort{ea} does not provide answers for the problems of today. Ultimately \acrshort{ea} should deliver this so the change can start tomorrow. It is essential to clarify the problems that public administrators are having. Both the facts as well the underlying causes. \acrshort{ea} should clarify the differences. Make it clear where we need to go and map out a path to get there, based on little steps that ultimately lead to the goal. Moreover, in the language public administrators understand. The architect must use the language of the stakeholders to make \acrshort{ea} successful (note: success factor).
\subsubsection{Question 2 / Agility of the public sector}
The \gls{ps} is more about risk aversion. Legality is about holding on to what is known. So, it is exactly known what the municipalities do, and we know precisely what the Land Registry does. All the subsystems of the \gls{ps} have a defined assignment. Moreover, it would be best if it stayed between the lines. Think, for example, about purpose limitation. Purpose Limitation will hold it back when \gls{ps} wants to be agile. The \gls{ps} cannot experiment that easily with rules like these. It will put experimentation at the edge.
The operating model of the \gls{ps} does not offer the freedom to do so that easily. The \gls{ps} can not experiment, discover and then say this was a nice experiment; let us go further. It is impossible to take decisions on the whole, such as within the social domain. If you have the right to confiscate a car, you cannot decide that there need to be a taxi to drive kids to school. It is a responsibility of a different part of the \gls{ps}. This all has to do with \gls{houseofthorbecke}\footnote{\url{https://www.denederlandsegrondwet.nl/id/vieqcpdzf0gw/bestuurlijke_indeling}} together with the current legislation and regulations. It does not mean that the \gls{ps} does not want to be more agile. It is not about separate parts of the government anymore. There is more and more collaboration based on federation. The \gls{ps} wants to be more agile. It is often a subject of conversation. Nevertheless, it gets stuck in the administrative decision-making processes.
\subsubsection{Question 3 / Dealing with uncertainty}
The reflex on \gls{uncertainty} of the \gls{ps} is that the \gls{ps} gets very insecure from \gls{uncertainty}. So the \gls{ps} does not know how to deal with \gls{uncertainty}. The common reflex is to push the uncertainty back to robust/resilient, so it is under control again. Robust \& resilient is back to its previous state but then sturdier, more robust (note: Risk avoidance). However, the \gls{ps} claims that they can deal with it. See, for example, the energy transition. The \gls{ps} defined the framework for this transition. It can contribute to the economy with many new jobs and a new knowledge model. We did see that also in the past with road and waterway engineering. Nevertheless, then they forget about the mechanisms needed to accomplish it. So the \gls{ps} does want to deal with \gls{uncertainty}, but the \gls{ps} is not creating the right conditions or the freedom of acting to be able to do so.
The available \acrshort{ea}'s within the \gls{ps} do not help either. It does not contribute to accept \gls{uncertainty}. At the most, our new vision on Information, Common Ground\footnote{\url{https://commonground.nl/}}, is contributing to this. An important principle in that vision is the ''community'' principle that could help with this \gls{uncertainty} (note: shared mental models). This principle states that municipalities, chain partners, market parties and the \acrshort{vngr} work together as a community in realisation. It is the certainty that it is uncertain. There are always new issues, and organising collaboration will help us to better deal with this \gls{uncertainty}, especially in the \gls{ps}.
\subsubsection{Question 4 / Dealing with unexpected events}
We, fortunately, live in a country where the \gls{ps} is staffed with good people who understand what citizens need or what is needed in a disaster area. So help is available pretty quickly. For example the fires at Moerdijk\footnote{\url{https://nl.wikipedia.org/wiki/Brand_Moerdijk_5_januari_2011}}, near Rotterdam. Or the plane crash of Turkish Airlines\footnote{\url{https://en.wikipedia.org/wiki/Turkish_Airlines_Flight_1951}} at Schiphol. Before we knew it, the fires were already distinguished. However, the \gls{ps} is not able to ascertain if it is an incident or a structural problem. If the problem takes too long, we see the reflex to control it again, so we fall back to our past habits. Adopting newly learned patterns is hard with the current legislation and regulations. Maybe we must be in a continuous crisis, so we have the freedom to do what we have to do. It is easier to be more agile in a crisis. With the local governments, there is only one who can decide what to do in a crisis, and this is the mayor as part of the triangle\footnote{\url{https://nl.wikipedia.org/wiki/Driehoek_(overheid)}}. 
It seems that the local governments have two organisation \& operational models. A model for running the municipality in a normal state and one in a state of crisis. (note: Seneca's Barbell?)
\subsubsection{Question 5 / The risk appetite of the public sector}
Drive and urge for innovation and change, which could perhaps be another interpretation of the word risk. People in the \gls{ps} want to find out if it can be done differently, but it is very dependent on the person. So actually, that depends on people in the \gls{ps}. Not on the \gls{ps} as a whole, based on intrinsic motivation to make things better for citizens and entrepreneurs. We have even set up our organisation in such a way that we can support this. Nowadays we have a department for research and innovation. But after something new is thought of it will go to the department to create it and finally to to a department to maintain it. Sometimes we are limited because it influences the standing legislation and regulations.
\subsubsection{Question 6 / The use of diversity and optionality in the public sector}
The \gls{ps} is not supporting diversity and optionality, but on the other hand the \gls{ps} is based on the absolute premise that all organisations are autonomous. (note: with a clear goal per organisation so no options). The implementation of the policies is extremely diverse. For example, there are various approaches known for people who are unemployed. One municipality forces people to perform work before they receive benefits, while another municipality supports them to be financially healthy again. Both achieve the obligation to provide care to a citizen so that the citizen has an income again. Diversity and optionality are less important. The local authorities simply have to implement the policies. It is all set down in the law and regulations. There is quite a diversity in how municipalities organize things. One municipality collects the household waste itself, while another has outsourced it. In both cases, the collection of household waste is arranged. Nobody prescribes how you arrange it, as long as it is arranged. You see the same thing within IT. However, you see here that for transitions and transformations an \acrshort{ea} is enormously needed to support the new world.
\subsubsection{Closing statements}
The government will not disappear, but that trust in the government will. Some government organizations or parts of them can undoubtedly disappear. (note: does not fit with the questions but find it an important statement).
The \gls{ps} needs a cross-sector \acrshort{ea}. It's no longer just about your organization. You have to work together more and more. The \acrshort{ea} can then be further tailored to your own organization.
\section{Interview Independent Software Vendor}
\label{sec:interviewisv}
\subsubsection{Question 1 / Enterprise Architecture}
Enterprise Architecture is developed to bring the business units together under one single architecture firstly. A common architecture brings synergy. It is reusing common components. Develop common language (note: Learning Organisation attribute shared mental model). It will bring us efficiency. Starting with architecture as a steering mechanism (note: engineering \gls{resiliency} attribute Command \& Control) and currently focusing on the internal organisation, the enterprise (note: mostly the first school of thought of \acrshort{ea} \parencite{Lapalme2012}). It is emerging that the current architecture is used as a communication mechanism to the external context (note: first steps into the second school of thought of \acrshort{ea} \parencite{Lapalme2012}). Our \acrshort{ea} is supporting us with the transformation towards a \acrfull{saas} provider. The \acrshort{ea} is used more and more used as a mechanism for explaining. The focus of the \acrshort{ea} is at this moment 80\% on the internal organisation and 20\% on the external context (note: not yet the third school of thought of \acrshort{ea}).
\acrshort{ea} is the responsibility of the \acrfull{coo} but the group of executive management is accountable. This group contains the \acrfull{ceo}, the \acrshort{coo} and the \acrfull{cco}. (note: with placing the responsibility on \acrshort{ea} with the \acrshort{coo} the primary purpose of \acrshort{ea} will be efficiency). The interviewee (\acrshort{ceo}) does not worry about this because in the end everything ends up with the \acrshort{ea}. \acrshort{ea} must be part of the executives. \acrshort{ea} is essential for business operations.
Our \acrshort{ea} supports us to be agile. Our crown jewels (our applications) are a stable core around which we can be flexible and agile to follow external contexts such as new laws and legislation. Think about the \acrfull{api} layer (note: systems \gls{resiliency} attribute Loosely Coupled) that is being built that makes it easier to respond to these changes. Eventually, our \acrshort{ea} must enable us to change to support our customers with their social tasks. We are not there yet. The transformation towards \acrfull{saas} alone takes us multiple years. This is, at this moment, not a problem yet.
The \gls{ps} is even moving slower, and there is not that much competition, but it is changing rapidly. The pace of change is increasing. It can be said that sometimes there is already a permanent state of change. Take the replanning of the municipalities and shifting tasks from the centralised government to the local government. The role of technology gets even more critical, the civilians are getting more empowered, and the participation rate in society increases. The influence of the external contexts does have more and more influence. Only the digital transformation itself is a stressor on the \gls{ps}. It already was there, but we see an increase. At this moment, the policymakers (politics) limit the speed of change.
These are not isolated incidents. An example is the ''Digitaal Stelsel Omgevingswet'', which is again being postponed. This is not sustainable in the near future. If this does not change, the \gls{ps} will get stuck.
\subsubsection{Question 2 / Agility of the public sector}
The current operational model of the \gls{ps} is old and moves slowly because of the regulations, legislation and qualified-majority decision-making. However, when there is a crisis, everything is possible. But only under extraordinary conditions. The \acrshort{ps} should be in a continuous crisis (note: looks like the \gls{antifragile} attribute of insert randomness). After a crisis, lessons learned are not used to improve the public sector (note: attribute part of the learning organisation). There is no feedback loop. The system is not supporting this. Changes to the current systems are slow, complex and large. Because of this, there are not that many suppliers on some solutions. For several solutions, there is only a choice between two (note: the \acrshort{cas} attribute diversity and optionality is not available.). In the worst case, there is only one solution, like with the taxes administration of the Ministery of Finance. The architectures in the \acrshort{ps} cannot support it because it misses alignment with business language. It misses stakeholder specific views in the language of the stakeholders. A good example is the \acrfull{idea}\footnote{\url{https://www.ictu.nl/projecten/idea-beeldtaal-maakt-it-infrastructuur-begrijpelijk}} method of the government. However, they stopped using it. 
Most IT management in the Public Sector is not IT Savvy. It would be better to have IT Savvy Management experienced with policymaking. The IT Systems contain much technical debt. To the extent that the systems with new functionality often use encapsulation. Adjusting IT Systems take much time with many risks. The impact of a new coalition agreement is high. With a coalition agreement, many high-impact system adjustments must be made. The policymakers expect changes to be executed in only a couple of days. In the past, public sector organisations were loosely coupled and were highly cohesive (clear goal). With all those policy changes, organisations even got strangled and cannot be adjusted that easily anymore, like with the taxes administration of the Ministery of Finance as an Example. The taxes administration was specialised in collecting taxes (note: Systems Resilience attribute Loosely Coupled (High Cohesion)). Policymakers also forced them to disbursement (note: Systems Resilience attribute antipattern with result tightly coupled with low cohesion). The same departments, processes and systems were used.
\subsubsection{Question 3 / Dealing with uncertainty}
You cannot define uncertainty on the public sector as a whole. The average size of municipalities is growing because of the reordering of Municipalities. Municipalities that are too small are merged (note: decrease of modularity, self-organisation and diversity). The scaling of municipalities is not always in the best interest of the civilians. It does not always improve the services to the residents of the municipalities. There are cases where a civilian needs to cycle 10km for a passport while it was less in the old situation. The services given are more business-like without a personal touch. If you look at the \gls{ps} for the last 200 years, the \gls{ps} is capable of adjusting when it needs to be adjusted (note: resilient/robust). The \gls{ps} can deal with uncertainty. However, if the way the \gls{ps} deals with uncertainty is the most efficient way is the question. The social cohesion that the civil servants of the \gls{ps} have is enormous. The \gls{ps} can handle uncertainty. The will is intrinsic available. If they get an assignment, they are going for it. If it must be done within four years (the duration of a coalition agreement), they will go for it. Even if the change is too big or complex and the planning is not realistic.
An example of the effect is that of the childcare benefits scandal\footnote{\url{https://en.wikipedia.org/wiki/Dutch_childcare_benefits_scandal}}. Decentralisation of governmental tasks was the cause of this. Because of the absence of \acrshort{ea} and the usage of \acrshort{ea} within the \acrshort{ps} domains, such as social domain, taxes, finance, a.o., these examples are not an incident. \acrshort{ea} can prevent these causes and effects. The fact that the \gls{ps} did not organise \acrshort{ea} is a cause of the incidents. The actual absence is an \acrshort{ea} process that guides the governments. This behaviour is especially shown with the local governments. They are continuously reinventing the wheel (note: No overarching Command \& Control). The \gls{ps} has to go back to the drawing board for every change to develop a new approach.
\subsubsection{Question 4 / Dealing with unexpected events}
The \gls{ps} is handling unexpected events better than uncertainty. The \gls{ps} handles unexpected events better than the political decisions made by coalition agreements. In a crisis situation, the \gls{ps} is capable of working very effectively. Should the \gls{ps} be in an ongoing crisis? No. The \gls{ps} is in need for \gls{antifragile} solutions. Better is to continuously add a small amount of stress to the \gls{ps} system (note: antifragile attribute insert randomness). This is in contrast to sitting back and watching until something happens. It seems that the rules do not apply anymore with an unexpected event. The \gls{ps} has many talents to deal with these situations, but they all seem too busy with their careers, salaries, what should go to which ministry, and others. This is the thing that needs to be solved. Strange because most of the time, the employees in the \gls{ps} enjoy working in a crisis. It makes them feel proud that they accomplished something. There were initiatives to use \acrshort{ea}, and it proved to be supporting the changes. Overarching examples are, for example, the consolidation of 66 datacenters to two private and two public datacenters (note: diversity and optionality), the common desktop standard (project ''goud'') (note: part of the stable part of Seneca's barbell strategy). Re-usability, an ICT dashboard, and many more initiatives were worked on. Later on, these initiatives fell apart, and the ministries picked it up again in their silo. It all was carried by a select group of people in the \gls{ps}. It all fell apart when some of them left the \gls{ps}. If it does not have assignments from the government members, it is dependent on the willingness to cooperate. The dominance of the separate ministries take the overhand, and people fall back in the old habits. To sustain the use of \acrshort{ea} it should not depend on a selective group of people but on the \gls{ps} itself (note: success factor). The mutual differences are gone when there is a common enemy (an unexpected event). At that moment, the solution will overarch the \gls{ps}. Changes following the process have less effect than changes initiated by chaos. The feedback from unexpected events is not fed into the system so that it can be changed (note: learning organisation not in place).
\subsubsection{Question 5 / The risk appetite of the public sector}
For the risk appetite of the \gls{ps}, the government members have an essential role. At this moment, there is no culture of risk-taking. Even worse, taking risks can have serious consequences. Think about, for example, commission ''Elias'' \footnote{\url{https://nl.wikipedia.org/wiki/Parlementair_onderzoek_ICT-projecten_bij_de_overheid}}. Because of this commission, a new department, \acrfull{bit}, was started as part of the Ministry of Home Affairs with the assignment to assess all the IT Projects within the centralised government (note: Engineering Resilience attribute Command \& Control). Because of this, people are not willing to take risks anymore (note: insert randomness, tinkering, naive interventions, monotonicity, fail-fast, and others). Some are busy shielding their bosses and managers for possible errors (note: \gls{antifragile} attribute: (no) skin in the game). At this moment, the \gls{ps} is showing risk avoidance behaviour. The base attitude of the \gls{ps} is that it does not have a risk appetite. Partly because of public opinion. It is all about the use of public funds. Before you know it, there will be negative attention in the media. \acrshort{ea} is mostly used in a prescriptive way (note: Engineering Resilience attribute Command \& Control). The \gls{ps} is not foster a safe environment for experimentation. Even when a good solution is implemented in a time of crisis (unexpected events), punishment will happen afterwards because it did not comply in the way it usually should. The \gls{ps} public sector created an environment in which the \gls{ps} is a fragile ''glass house'' together with a culture of blaming. So the risk appetite is getting less and less.
\subsubsection{Question 6 / The use of diversity and optionality in the public sector}
Optionality does not have a chance in the \gls{ps} because of european tender obligations. The european tenders are mostly about risk reduction. The european tenders contain many legal conditions. But not only legal conditions but also a lot of technical conditions. Everything is defined in a way that you have no options anymore. The conditions are even so that you cannot choose, for example, multiple suppliers so you will have options during the contract periods. The private sector has this already for a long time. There are private companies who have multiple suppliers for a domain. If one supplier is not delivering the quality anymore another supplier is taking over. European tenders did not help us to become more flexible, resilient, and agile. But there are changes. It would be nice to see if the \acrfull{vngr} will be thinking of a broker construction with multiple suppliers. By using this strategy the local governments can choose a supplier by only using bids. It is easier to switch and having options. Another thing that can help optionality is defining right \acrfull{kpi}'s. If you define a KPI in such a way that the performance of a supplier is measured by the ease of transitioning to another supplier it will get easier to switch suppliers. This has a positive influence on executing optionality. But this way of working is not sustained in, for example, the \acrshort{ea}.
\subsubsection{Closing statements}
The digital transformation must be important to everyone and not only to a minister of digital affairs. How do you make sure that business management of the \gls{ps} find it normal to discuss IT, budget, personnel, organisational configuration, and others? If they start thinking like this, they will find out what \acrshort{ea} can do for them. If we know how to close this gap, digital transformation will get the proper attention. We also have to thank ourselves for this because of the use of non-business language.
\section{Interview with consultancy firm/service provider}
\label{sec:interviewconsultancyfirm}
\subsubsection{Question 1 / Enterprise Architecture}
We have, to an extent, an \acrshort{ea} with the necessary elements for the products and services we develop ourselves. We do not have a dedicated enterprise architect. Other types of architects maintain the current \acrshort{ea}. The \acrshort{cto} is accountable for the \acrshort{ea}. In the end, everything rolls up to the \acrshort{ceo}. 

Our \acrshort{ea} is, at this moment, mostly about our products and services and addresses our primary concerns. The concerns are the connections between data, how they should communicate, and the impact on our products and services. With the \acrshort{ea}, we can determine our solution gaps and steer towards procurement of applications and integrations. The integrations are with the sales, finance, HR, and delivery capabilities. We still have two separate worlds in our organisation. These two worlds are the supporting and delivery capabilities of our organisation. Bringing these two worlds together will be on the roadmap for next year so we can work with an integrated \acrshort{ea}. Both worlds come together when we think about our customers. We will realise that when we develop features for our platform, we can connect to the propositions that we offer to our customers. The lack of an \acrshort{ea} slows us down from achieving this. We do not have a business architecture, but we advise our customers on business architecture. We have to close this gap. 

With the current \acrshort{ea} we can make adjustments to our products and services very fast and flexible. Our \acrshort{ea} supports it because it contains the architecture of our products and services and our infrastructure down to the data models that we use for our customers. However, we do not have control over our supporting applications, such as Salesforce. Our products and services are \gls{robust} \& \gls{resilient} and support us to be \gls{agile}. Sometimes we disconnect a server to see what happens (note: insert randomness / SRE / Chaos Engineering). We continuously improve ourselves to get better (note: learning organisation).
\subsubsection{Question 2 / Agility of the public sector}
There is a low degree of agility in the \gls{ps}. This low \gls{agility} is possible because of the lack of IT knowledge and skills in politics and policymakers. If we look at the electoral lists of the central and local elections, we can state that there is a shortage of knowledge, skills and fundamental IT knowledge. There are exceptions, but not that much.

If we compare this to, for example, the Estonian model, we see a world of difference. We are not there (yet). We have to invest more in this. If we think of the \gls{ps} as an aggregate and you go lower in the aggregation, you already see that it is going better. We see that the \gls{ps} has been taking significant steps in the last ten years. Administrative governmental agencies have more responsibilities to operationalise, develop and maintain systems. We have been able to leave a mark in the \acrshort{ps} on the technical quality of systems. It is a good development that the \acrshort{bit} exists. The operating model of the \gls{ps} is extending. Compared to the private sector, the government is at a good average. Maybe this is already a good position for the government. We will see more and more connections and integrations with specific ministries and administrative agencies. The digital transformation is progressing. More and more is shared online and is easy to access. We see this as a positive effect. We can help the \gls{ps} further because of this by bringing in best practices. How should we unlock our data, what can we do with this data, and what is the effect on IT and Governance. \acrshort{ea} can support us in this.
\subsubsection{Question 3 / Dealing with uncertainty}
We see the desire for robustness \& resilience. The \gls{ps} tries to push it back to how it was. As an example, the regulations on electric steps. It is a new development, and it falls outside the scope of current regulations. It is not a scooter or a bicycle, and it does not fit in any other regulations. The reaction of the policymakers is to rule it out. It is not allowed until there is an agreement on new regulations. The Dutch model is to push it back to how it was. The electric steps are allowed if there is an agreement on how to allow them. We see the same behaviour with the  IT capability. Policymaking takes time. It slows down new developments. The choices are made based on robustness, certainty and clarity. The behaviour has everything to do with the risk aversion of the \gls{ps}.

The basic attitude of the \gls{ps} is to avoid mistakes. When a choice has a risk, they do not decide until everything is clear. There is an implicit postponement in this behaviour. The risk of this behaviour is that the \gls{ps} is missing great opportunities. The founding of \acrshort{bit} is an excellent thing to have some certainty, but it is concerning that the whole \gls{ps} is moving towards control and risk mitigation. It removes agility from the system while the government plays a facilitating role in our society. All risks should be avoided, and everything needs to be traceable, making no mistakes. By this approach, the \gls{ps} is probably missing out on options that can make a difference, and it inhibits realising potentials. The \gls{ps} is using \acrshort{ea} as a way to attenuate. 
\subsubsection{Question 4 / Dealing with unexpected events}
If something happens, there is the will to act by setting up something new, reworking systems, and other things. Nevertheless, there still is a considerable delay after the unexpected event happened. After achieving the goal is directly followed up by attenuation. The \gls{ps} does not want to make mistakes, so the \gls{ps} will do as minimal as possible to achieve the goal because of risk mitigation.

If something happens, the \gls{ps} deals with that. However, because of the aversion to risks, the \gls{ps} is not getting the most out of it. If the \gls{ps} exploits the situation, instead of familiar ways, with more radical approaches, the result will be a significant progression, even when there is uncertainty. It is another way of doing so with the risk that something will go wrong. If it goes wrong, there is a risk that the press will magnify it because it is about spending public money. Unfortunately, successes do not get attention.
\subsubsection{Question 5 / The risk appetite of the public sector}
We want to play a role in this world by being more innovative. We allow some governmental bodies to go a little further in experimentation and development, but this is mainly on a project-by-project basis. For these projects, we accepted that it would cost us public money and that making mistakes is allowed. However, generically, the trajectories we see do not have a risk appetite and are even risk-averse. Most of the time, this is good for a reliable government. Use, for risk-taking cases, specific demarcated parts of the \gls{ps}.
\subsubsection{Question 6 / The use of diversity and optionality in the public sector}
We see an improvement on this topic over the past years in the \gls{ps}. We see a growth in knowledge, from an IT perspective, in multiple areas in the \gls{ps}. E.g. architecture, implementation, development, code quality and other qualitative aspects of IT. We see, at this moment, this contribution mainly from the central government and not so much from the local governments. We think that this improvement will continue. We do hope that this improvement will reach the electoral lists. If we look at the systems, we do not see any uniformity. We do see differences in designs and the ways of looking at things.