\chapter{Personal Motivation}
I always want to know how something works and why it works the way it works. This eagerness started at a very early age. I always demolished birthday presents into their parts. I wondered how things worked, and I did not stop with the research on the how and why until I understood it. The search for the how and why is a central theme in my life. And because of this, I never stopped learning. When the why is not known, I never give up researching—not knowing the why has only the meaning that nobody has found the answer yet. 

My essential attitude is that of a mathematician or a scientist. I am very binary and sceptical when something is somewhere in between, and I am not fond of shades of grey. So clear definitions is what I pursue. Everything needs an explanation. When I cannot explain, I do not fall back to religion, infinity, approximately, or even ''it is just what it is''. I only accept that we did not find the answer yet.

This journey started with simple things like a toy car or a doll that had a mechanism of saying things. I still think of my little sister, who found hers back in tiny pieces. Later the subjects of research began to be different. Secondary education drove me to understand chemistry, physics, and biology, and I needed an explanation on the who and why to understand it. This drive was probably the main reason I was not perfect in languages. You cannot put consistent rules on languages.  Languages do not have a clear rationale for the rules, and I often heard it is just the way it is.

Grammar school taught me that I was a natural in research, and I decided to pursue research. Firstly at several Universities, but I failed big time. The Universities at that time gave me a lot of answers that it is the way it is, and most of the lecturers did not appreciate me challenging them on the why. Because of this, I started pursuing a job that could fulfil my eagerness for researching, and I found that with an IT Company in the Netherlands.

The technical (hard) side of IT was, at that time, a match made in heaven. Most of the time, it is just like math, you have a clear answer, and you know why you get that answer. Before I knew it, I was a Senior Consultant and an IT Architect shortly after that. Gaining knowledge is one thing that drives me, but the other thing is sharing that knowledge with others. Gaining and sharing knowledge is the thing that gets me up in the morning. For sharing knowledge, I taught, as a trainer, adults on the why and how of technology subjects for years.

In those years, I advised dozens of companies of the public and private sectors in the Benelux on how they could apply technologies and what problem it solved for them. This period did teach me a lot by seeing other companies and working with different kinds of people. But technology was driving me less and less. Most of the time technology did not change with new releases or new versions. In the base it was still the same mean to reach a business goal. I often solved my customers' problems by not introducing new technology but changing their processes, information architecture, culture, team compositions, or organisational construction. I became more and more interested in how and why organisations could achieve their goals by using IT.

I am very successful in this field of work but I always wondered why I did things in a certain way. This led my back to education and I started a Bachelor in Business \& IT with a University of Applied Sciences. This time it was a big success. Because I already had a lot of experience I was admitted to a University of Applied Sciences only for people who had already experience. They did expect me to challenge the teachers on the why.


