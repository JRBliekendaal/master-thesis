\subsection{Interview ISV}
\label{sub:interviewisv}
% Maarten Hillenaar
\subsubsection{Question 1 / Enterprise Architecture}
Enterprise Architecture is developed to bring the business units together under one single architecture firstly. A common architecture brings synergy. It is reusing common components. Develop common language (note: Learning Organisation attribute shared mental model). It will bring us efficiency. Starting with architecture as a steering mechanism (note: engineering \gls{resiliency} attribute Command \& Control) and currently focusing on the internal organisation, the enterprise (note: mostly the first school of thought of \acrshort{ea} \parencite{Lapalme2012}). It is emerging that the current architecture is used as a communication mechanism to the external context (note: first steps into the second school of thought of \acrshort{ea} \parencite{Lapalme2012}). Our \acrshort{ea} is supporting us with the transformation towards a \acrfull{saas} provider. The \acrshort{ea} is used more and more used as a mechanism for explaining. The focus of the \acrshort{ea} is at this moment 80\% on the internal organisation and 20\% on the external context (note: not yet the third school of thought of \acrshort{ea}).
\acrshort{ea} is the responsibility of the \acrfull{coo} but the group of executive management is accountable. This group contains the \acrfull{ceo}, the \acrshort{coo} and the \acrfull{cco}. (note: with placing the responsibility on \acrshort{ea} with the \acrshort{coo} the primary purpose of \acrshort{ea} will be efficiency). The interviewee (\acrshort{ceo}) does not worry about this because in the end everything ends up with the \acrshort{ea}. \acrshort{ea} must be part of the executives. \acrshort{ea} is essential for business operations.
Our \acrshort{ea} supports us to be agile. Our crown jewels (our applications) are a stable core around which we can be flexible and agile to follow external contexts such as new laws and legislation. Think about the \acrfull{api} layer (note: systems \gls{resiliency} attribute Loosely Coupled) that is being built that makes it easier to respond to these changes. Eventually, our \acrshort{ea} must enable us to change to support our customers with their social tasks. We are not there yet. The transformation towards \acrfull{saas} alone takes us multiple years. This is, at this moment, not a problem yet.
The \gls{ps} is even moving slower, and there is not that much competition, but it is changing rapidly. The pace of change is increasing. It can be said that sometimes there is already a permanent state of change. Take the replanning of the municipalities and shifting tasks from the centralised government to the local government. The role of technology gets even more critical, the civilians are getting more empowered, and the participation rate in society increases. The influence of the external contexts does have more and more influence. Only the digital transformation itself is a stressor on the \gls{ps}. It already was there, but we see an increase. At this moment, the policymakers (politics) limit the speed of change.
These are not isolated incidents. An example is the ''Digitaal Stelsel Omgevingswet'', which is again being postponed. This is not sustainable in the near future. If this does not change, the \gls{ps} will get stuck.
\subsubsection{Question 2 / Agility of the public sector}
The current operational model of the \gls{ps} is old and moves slowly because of the regulations, legislation and qualified-majority decision-making. However, when there is a crisis, everything is possible. But only under extraordinary conditions. The \acrshort{ps} should be in a continuous crisis (note: looks like the \gls{antifragile} attribute of insert randomness). After a crisis, lessons learned are not used to improve the public sector (note: attribute part of the learning organisation). There is no feedback loop. The system is not supporting this. Changes to the current systems are slow, complex and large. Because of this, there are not that many suppliers on some solutions. For several solutions, there is only a choice between two (note: the \acrshort{cas} attribute diversity and optionality is not available.). In the worst case, there is only one solution, like with the taxes administration of the Ministery of Finance. The architectures in the \acrshort{ps} cannot support it because it misses alignment with business language. It misses stakeholder specific views in the language of the stakeholders. A good example is the \acrfull{idea}\footnote{\url{https://www.ictu.nl/projecten/idea-beeldtaal-maakt-it-infrastructuur-begrijpelijk}} method of the government. However, they stopped using it. 
Most IT management in the Public Sector is not IT Savvy. It would be better to have IT Savvy Management experienced with policymaking. The IT Systems contain much technical debt. To the extent that the systems with new functionality often use encapsulation. Adjusting IT Systems take much time with many risks. The impact of a new coalition agreement is high. With a coalition agreement, many high-impact system adjustments must be made. The policymakers expect changes to be executed in only a couple of days. In the past, public sector organisations were loosely coupled and were highly cohesive (clear goal). With all those policy changes, organisations even got strangled and cannot be adjusted that easily anymore, like with the taxes administration of the Ministery of Finance as an Example. The taxes administration was specialised in collecting taxes (note: Systems Resilience attribute Loosely Coupled (High Cohesion)). Policymakers also forced them to disbursement (note: Systems Resilience attribute antipattern with result tightly coupled with low cohesion). The same departments, processes and systems were used.
\subsubsection{Question 3 / Dealing with uncertainty}
You cannot define uncertainty on the public sector as a whole. The average size of municipalities is growing because of the reordering of Municipalities. Municipalities that are too small are merged (note: decrease of modularity, self-organisation and diversity). The scaling of municipalities is not always in the best interest of the civilians. It does not always improve the services to the residents of the municipalities. There are cases where a civilian needs to cycle 10km for a passport while it was less in the old situation. The services given are more business-like without a personal touch. If you look at the \gls{ps} for the last 200 years, the \gls{ps} is capable of adjusting when it needs to be adjusted (note: resilient/robust). The \gls{ps} can deal with uncertainty. However, if the way the \gls{ps} deals with uncertainty is the most efficient way is the question. The social cohesion that the civil servants of the \gls{ps} have is enormous. The \gls{ps} can handle uncertainty. The will is intrinsic available. If they get an assignment, they are going for it. If it must be done within four years (the duration of a coalition agreement), they will go for it. Even if the change is too big or complex and the planning is not realistic.
An example of the effect is that of the childcare benefits scandal\footnote{\url{https://en.wikipedia.org/wiki/Dutch_childcare_benefits_scandal}}. Decentralisation of governmental tasks was the cause of this. Because of the absence of \acrshort{ea} and the usage of \acrshort{ea} within the \acrshort{ps} domains, such as social domain, taxes, finance, a.o., these examples are not an incident. \acrshort{ea} can prevent these causes and effects. The fact that the \gls{ps} did not organise \acrshort{ea} is a cause of the incidents. The actual absence is an \acrshort{ea} process that guides the governments. This behaviour is especially shown with the local governments. They are continuously reinventing the wheel (note: No overarching Command \& Control). The \gls{ps} has to go back to the drawing board for every change to develop a new approach.
\subsubsection{Question 4 / Dealing with unexpected events}
The \gls{ps} is handling unexpected events better than uncertainty. The \gls{ps} handles unexpected events better than the political decisions made by coalition agreements. In a crisis situation, the \gls{ps} is capable of working very effectively. Should the \gls{ps} be in an ongoing crisis? No. The \gls{ps} is in need for \gls{antifragile} solutions. Better is to continuously add a small amount of stress to the \gls{ps} system (note: antifragile attribute insert randomness). This is in contrast to sitting back and watching until something happens. It seems that the rules do not apply anymore with an unexpected event. The \gls{ps} has many talents to deal with these situations, but they all seem too busy with their careers, salaries, what should go to which ministry, and others. This is the thing that needs to be solved. Strange because most of the time, the employees in the \gls{ps} enjoy working in a crisis. It makes them feel proud that they accomplished something. There were initiatives to use \acrshort{ea}, and it proved to be supporting the changes. Overarching examples are, for example, the consolidation of 66 datacenters to two private and two public datacenters (note: diversity and optionality), the common desktop standard (project ''goud'') (note: part of the stable part of Seneca's barbell strategy). Re-usability, an ICT dashboard, and many more initiatives were worked on. Later on, these initiatives fell apart, and the ministries picked it up again in their silo. It all was carried by a select group of people in the \gls{ps}. It all fell apart when some of them left the \gls{ps}. If it does not have assignments from the government members, it is dependent on the willingness to cooperate. The dominance of the separate ministries take the overhand, and people fall back in the old habits. To sustain the use of \acrshort{ea} it should not depend on a selective group of people but on the \gls{ps} itself (note: success factor). The mutual differences are gone when there is a common enemy (an unexpected event). At that moment, the solution will overarch the \gls{ps}. Changes following the process have less effect than changes initiated by chaos. The feedback from unexpected events is not fed into the system so that it can be changed (note: learning organisation not in place).
\subsubsection{Question 5 / The risk appetite of the public sector}
For the risk appetite of the \gls{ps}, the government members have an essential role. At this moment, there is no culture of risk-taking. Even worse, taking risks can have serious consequences. Think about, for example, commission ''Elias'' \footnote{\url{https://nl.wikipedia.org/wiki/Parlementair_onderzoek_ICT-projecten_bij_de_overheid}}. Because of this commission, a new department, \acrfull{bit}, was started as part of the Ministry of Home Affairs with the assignment to assess all the IT Projects within the centralised government (note: Engineering Resilience attribute Command \& Control). Because of this, people are not willing to take risks anymore (note: insert randomness, tinkering, naive interventions, monotonicity, fail-fast, and others). Some are busy shielding their bosses and managers for possible errors (note: \gls{antifragile} attribute: (no) skin in the game). At this moment, the \gls{ps} is showing risk avoidance behaviour. The base attitude of the \gls{ps} is that it does not have a risk appetite. Partly because of public opinion. It is all about the use of public funds. Before you know it, there will be negative attention in the media. \acrshort{ea} is mostly used in a prescriptive way (note: Engineering Resilience attribute Command \& Control). The \gls{ps} is not foster a safe environment for experimentation. Even when a good solution is implemented in a time of crisis (unexpected events), punishment will happen afterwards because it did not comply in the way it usually should. The \gls{ps} public sector created an environment in which the \gls{ps} is a fragile ''glass house'' together with a culture of blaming. So the risk appetite is getting less and less.
\subsubsection{Question 6 / Using diversity and optionality in the public sector}
Optionality does not have a chance in the \gls{ps} because of european tender obligations. The european tenders are mostly about risk reduction. The european tenders contain many legal conditions. But not only legal conditions but also a lot of technical conditions. Everything is defined in a way that you have no options anymore. The conditions are even so that you cannot choose, for example, multiple suppliers so you will have options during the contract periods. The private sector has this already for a long time. There are private companies who have multiple suppliers for a domain. If one supplier is not delivering the quality anymore another supplier is taking over. European tenders did not help us to become more flexible, resilient, and agile. But there are changes. It would be nice to see if the \acrfull{vngr} will be thinking of a broker construction with multiple suppliers. By using this strategy the local governments can choose a supplier by only using bids. It is easier to switch and having options. Another thing that can help optionality is defining right \acrfull{kpi}'s. If you define a KPI in such a way that the performance of a supplier is measured by the ease of transitioning to another supplier it will get easier to switch suppliers. This has a positive influence on executing optionality. But this way of working is not sustained in, for example, the \acrshort{ea}.
\subsubsection{Closing statements}
The digital transformation must be important to everyone and not only to a minister of digital affairs. How do you make sure that business management of the \gls{ps} find it normal to discuss IT, budget, personnel, organisational configuration, and others? If they start thinking like this, they will find out what \acrshort{ea} can do for them. If we know how to close this gap, digital transformation will get the proper attention. We also have to thank ourselves for this because of the use of non-business language.