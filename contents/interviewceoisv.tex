\subsection{Interiew CEO of ISV}
\label{sub:interviewceoisv}
% Maarten Hillenaar
\subsubsection{Question 1 / Enterprise Architecture}
Enterprise Architecture is developed to firstly bring the business units together under one single architecture. A common architecture brings synergy.Reusing common components. Develop common language (note: Learning Organisation attribute shared mental model). It will bring us efficiency.Starting with architecture as a steering mechanism (note: engineering \gls{resiliency} attribute Command \& Control). Currently focus on the internal organisation, the enterprise (note: mostly the first school of thought of \acrshort{ea} \parencite{Lapalme2012}). It is emerging that the current architecture can be used as a communication mechanism to the external context (note: first steps into the second school of thought of \acrshort{ea} \parencite{Lapalme2012}). Our \acrshort{ea} is supporting us with the transformation towards a \acrfull{saas} provider. The \acrshort{ea} is used more and more used as a mechanism for explanation. The focus of the \acrshort{ea} is at this moment 80\% on the internal organisation and 20\% on the external context (note: not yet the third school of thought of \acrshort{ea}).
\acrshort{ea} is the responsibility of the \acrfull{coo} but the group of executive management is accountable. This group contains the \acrfull{ceo}, the \acrshort{coo} and the \acrfull{cco}. (note: with placing the responsibility on \acrshort{ea} with the \acrshort{coo} the primary purpose of \acrshort{ea} will be efficiency). The interviewee (\acrshort{ceo}) does not worry about this because in the end everything ends up with the \acrshort{ea}. \acrshort{ea} must be part of the executives. \acrshort{ea} is essential for business operations.
Our \acrshort{ea} supports us to be agile. Our crown jewels (our applications) is a stable core around which we can be flexible and agile to follow our external context such as new laws and legislation. Think about the \acrfull{api} layer (note: systems \gls{resiliency} attribute Loosely Coupled) that is being build that makes it easier to respond on these changes. Eventually our \acrshort{ea} must enable us to change to support our customers with their social tasks. We are not there yet. The transformation towards \acrfull{saas} alone takes us multiple years. This is at this moment not a problem yet. The \gls{ps} is even moving slower, and there is not that much competition, but it is changing rapidly. The pace of change is increasing. You could say that sometimes there is already a permanent state of change. Take the replanning of the municipalities and shifting tasks from the centralised government to the local government. The role of technology gets more important, the civilians are getting more empowered and the participation rate in society if increasing. The influence of the external context does have more and more influence. Only the digital transformation itself is a stressor on our customers. It already was there but we see an increase in it. At this moment the speed of change is limited by the policymakers (politics). Example is the ''Digitaal Stelsel Omgevingeweg'' that is being postponed a a lot. This is not an isolated incident. This is not sustainable in the near future. If this does not change the public sector will get stuck.
\subsubsection{Question 2 / Agility of the public sector}
The current operational model of the \gls{ps} is old moves slow because of the regulations, legislation and the qualified-majority decision-making. But while there is a crisis everything is possible. But only under very special conditions. The \acrshort{ps} should be in a continuous crisis (note: looks like the \gls{antifragile} attribute of insert randomness). After a crisis the lessons learned are not used to improve the public sector (note: attribute part of the learning organisation). There is not feedback loop. The system is not supporting this. Changes to the current systems are slow, complex and large. Because of this there are not that many suppliers on some solutions. For several solutions there is only a choice between two (note: the \acrshort{cas} attribute diversity and optionality is not available.). In the worst case there is only one solutions like with the Department of Taxes. The architectures in the \acrshort{ps} cannot support it because it misses alignment with business language. It misses stakeholder specific views in the language of the stakeholders. A good example is the \acrfull{idea}\footnote{\url{https://www.ictu.nl/projecten/idea-beeldtaal-maakt-it-infrastructuur-begrijpelijk}} method of the government. But they stopped using it. 
Most of IT management in the Public Sector is not IT Savvy. It would be better to have IT Savvy Management with experience with policy making. The IT Systems contain a lot of technical debt. To extent the systems with new functionality often encapsulation is used. The impact of a new coalition agreement is high. Adjusting those IT Systems take a lot of time with a lot of risk. With a coalition agreement a lot of high impact system adjustments must be made. The policy makers expect this in only a couple of days. In the past public sector organisations where loosely coupled and had a clear goal. With all those policy changes organisations even got strangled and have a hard time to adjust themselves. Example Department of Taxes. They where specialised in collecting taxes (note: Systems Resilience attribute Loosely Coupled (High Cohesion)). Policy making forced the department of Taxes to also do disbursement. The same departments where used.

\subsubsection{Question 3 / Dealing with uncertainty}



\subsubsection{Question 4 / Dealing with unexpected events}

\subsubsection{Question 5 / The risk appetite of the public sector}

\subsubsection{Question 6 / Using diversity and optionality in the public sector}

\subsubsection{Closing statements}


The public sector should be in a continuous state of crisis.

