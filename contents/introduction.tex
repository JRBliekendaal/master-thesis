\chapter{Introduction}
\label{ch:introduction}
blah\\

In this thesis, the researcher defines how and with which Enterprise Architecture concepts Enterprise Architecture can be used to steer an \acrfull{isv} towards being \gls{antifragile} in the Public Sector Market.

\section{Context}
\label{sec:context}
The researcher is working for an independent software vendor (ISV) specialising in delivering software and services to the local governments in The Netherlands, such as the municipalities, the provinces, and the regional water authorities. Because of the digital transformation, the pace of change is increasing rapidly \needsref. The increasing pace is not only seen in the private sector but also in the public sector. 

\section{Structure of the thesis}
\label{sec:structure}
In chapter \ref{ch:introduction} the context of the research is set, the core concepts of Enterprise Architecture and Antifragility are introduced together with the contextual concepts of \acrshort{isv} and the Public Sector Market. In chapter \ref{ch:theoreticalbackground} the theoretical background is given on the research. Chapter \ref{ch:research-methodology} explains the used methodology for the research.

\section{Introduction of the Public Sector Market}
\label{sec:intropublicsector}
\lipsum[1]

\section{Introduction of Indepenent Software Vendor}
\label{sec:introisv}
\lipsum[1]

\section{Introduction of the concept Enteprise Architecture}
\label{introea}
\lipsum[1]

\section{Introduction of the concept of Antifragility}
\label{sec:introantifragility}
\lipsum[1]

\section{Problem statement}
\label{sec:problemstatement}

\section{Research questions}
\label{sec:research-questions}

\subsection{Main research question}
\label{sub:main-research-question}
What are, for an \acrlong{isv}, the success factors of enterprise architecture for \glspl{antifragile} in the public sector market?

\subsection{Sub-questions}
\label{sub:sub-questions}

\begin{enumerate}
	\item{Sub-question 1}
	\item{Sub-question 2}
\end{enumerate}
