\chapter{Introduction}
\label{ch:introduction}
Speed of change\\
Only constant is change\\

\acrfull{vuca}\\
The challenge\\

In this thesis, the researcher defines how and with which \acrfull{ea} concepts \acrshort{ea} can be used to steer an \acrfull{isv} towards being \gls{antifragile} in the Public Sector Market.

\section{Context}
\label{sec:context}
The researcher is working as a Chief Architect for an \acrshort{isv} specialised in delivering software and services to the local governments in The Netherlands, such as the municipalities, the provinces, and the regional water authorities. The local governments embraced the digital transformation, and because of this the pace of change is increasing rapidly \needsref. 

\section{Structure of the thesis}
\label{sec:structure}
In chapter \ref{ch:introduction} the context of the research is set, the core concepts of \acrshort{ea} and \glspl{antifragile} are introduced together with the contextual concepts of \acrshort{isv} and the Public Sector Market. In chapter \ref{ch:theoreticalbackground} the theoretical background is given on the research. Chapter \ref{ch:research-methodology} explains the used methodology for the research.

\section{Introduction of the Public Sector Market}
\label{sec:intropublicsector}

\section{Introduction of Indepenent Software Vendor}
\label{sec:introisv}

\section{Introduction of the concept Enteprise Architecture}
\label{introea}

\section{Introduction of the concept of Antifragility}
\label{sec:introantifragility}

\section{Problem statement}
\label{sec:problemstatement}

\section{Research questions}
\label{sec:research-questions}

\subsection{Main research question}
\label{sub:main-research-question}
\begin{center}

\framebox{
\begin{minipage}{0.9\linewidth}
What are, for an \acrlong{isv}, the success factors of \acrlong{ea} for \gls{antifragility} in the public sector market?
\end{minipage}
}
\end{center}

\subsection{Sub-questions}
\label{sub:sub-questions}

\begin{enumerate}
	\item{Sub-question 1}
	\item{Sub-question 2}
\end{enumerate}
