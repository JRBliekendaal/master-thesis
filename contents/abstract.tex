\thispagestyle{plain}
\phantomsection
\addcontentsline{toc}{chapter}{Abstract}
\begin{center}
	\Large
	\textbf{Towards an Antifragile Public Sector}
	
	\vspace{0.4cm}
	\large
	Introducing \Gls{antifragility} in the Dutch Public Sector with \gls{ea}
	
	\vspace{0.4cm}
	\textbf{René Bliekendaal}
	
	\vspace{0.9cm}
	\textbf{Abstract}
\end{center}

\small \noindent The Greek philosopher Heraclitus once said that one constant since the beginning of time is change. His central claim is summed up in the phrase Panta Rhei ("life is flux"), recognising life's essential, underlying essence as change. Nothing in life is permanent, nor can it be, because the very nature of existence is change. The Dutch public sector deals with many changes in its environment. Changes follow one another at lightning speed. In recent years, the external environment placed new and increasingly compelling demands on the functioning of public organisations. A responsive and adaptive government is needed to deal with this. We need to create public organisations that can cope with or even seize opportunities in a dynamic, complex, unpredictable environment. The public sector started an improvement program for information provisioning. On multiple occasions, the improvement program mentions using Enterprise Architecture. It was proposed to use \gls{antifragile} from Nassim Taleb to deal with disruptive change. However, how can the Dutch \gls{ps} achieve \gls{antifragility} with support of \gls{ea}? Hence, the research question 'What are success factors of \gls{ea} and \gls{antifragile} that positively influence the contribution of \gls{ea} in achieving \gls{antifragility} in the Dutch \gls{ps}?' We can conclude that there are fourteen potential success factors. For this conclusion, we performed literature research. We used semi-structured interviews with four Dutch \gls{ps} CxOs for validation and data collection, and we validated the results with an expert group of ten Dutch \gls{ps} Enterprise Architects. We used triangulation to determine the success factors by ranking.