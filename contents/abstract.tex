\thispagestyle{plain}
%\phantomsection
%\addcontentsline{toc}{chapter}{Abstract}
\begin{center}
	\Large
	\textbf{Towards an Antifragile Public Sector}
	
	\vspace{0.4cm}
	\large
	Introducing \Gls{antifragility} in the Dutch Public Sector with \gls{ea}
	
	\vspace{0.4cm}
	\textbf{René Bliekendaal}
	
	\vspace{0.9cm}
	\textbf{Abstract}
\end{center}

\small \noindent The Greek philosopher Heraclitus once said that one constant since the beginning of time is change. Nothing in life is permanent, nor can it be, because the very nature of existence is change. The Dutch \gls{ps} deals with many changes in its environment. Changes follow one another at lightning speed. The Dutch \gls{ps} uses \gls{ea} to accompany change. In recent years, the environment placed new and increasingly compelling demands on the functioning of public organisations. A responsive and adaptive government is needed to deal with this. We must create public organisations that can cope with or even seize opportunities in a dynamic, complex, unpredictable environment. The Dutch \gls{ps} proposed to use \gls{antifragile} to deal with disruptive change. However, how can the Dutch \gls{ps} achieve \gls{antifragility} with the support of \gls{ea}? Hence, the research question of this thesis is: 'What are the success factors of \gls{ea} and \gls{antifragile} that positively influence the contribution of \gls{ea} in achieving \gls{antifragility} in the Dutch \gls{ps}?' We can conclude that there are fourteen potential success factors. We used triangulation to determine these success factors. We performed literature research and used semi-structured interviews with four CxOs of the Dutch \gls{ps} for validation and data collection. An expert group of ten Dutch \gls{ps} enterprise architects validated the final results.