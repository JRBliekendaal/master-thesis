\chapter*{Executive Summary}
\label{executivesummary}
The Greek philosopher Heraclitus once said that one constant since the beginning of time is change. His central claim is summed up in the phrase Panta Rhei ("life is flux"), recognising life's essential, underlying essence as change. Nothing in life is permanent, nor can it be, because the very nature of existence is change. The Dutch public sector deals with many changes in its environment. Changes follow one another at lightning speed. These are changes such as new technologies, social developments and political priorities. In recent years, the external environment placed new and increasingly compelling demands on the functioning of public organisations. The \gls{ps} finds it challenging to adapt to the expected speed of change. ''There is a need to invest for an even a better government that can respond adequately and flexibly to unforeseen circumstances.'' was plead to 'Schippers' \parencite{Secretarissen-generaal2018}. A responsive and adaptive government is needed to deal with this. We need to create public organisations that can cope with or even seize opportunities in a dynamic difficult, unpredictable environment. 

This continuous change confronts policy-makers with high demands on their steering skills. The public sector started an improvement program for information provisioning to deal with the increasingly compelling demands on the functioning of public organisations. On multiple occasions, the improvement program mentions using Architecture in different appearances (\gls{ea}, \acrfull{nora} and \acrfull{ear}) supporting the improvements.

The Dutch \gls{ps} wants to change to be more adaptive and responsive. It was proposed to use \gls{antifragile} from \textcite{Taleb2012} to deal with disruptive change. \gls{ea} is defined as a tool by the Dutch \gls{ps} to support with the implementation of changes. However, how can the Dutch \gls{ps} achieve \gls{antifragility} with support of \gls{ea}? What are \gls{antifragile} success factors relevant to the Dutch \gls{ps}, and what are \gls{ea} success factors in achieving it? Hence, our research question:

\vspace{\baselineskip}
\noindent \emph{'What are success factors of \gls{ea} and \gls{antifragile} that positively influence the contribution of \gls{ea} in achieving \gls{antifragility} in the Dutch \gls{ps}?'}
\vspace{\baselineskip}

We can conclude --- based on our used data sets --- that there are fourteen \glspl{attribute} that are potential success factors. The first seven success factors are found in literature, confirmed by interviews, and agreed upon by an expert group. The last seven are found in two out of three research tools and are less probable and need more research. Nevertheless, two success factors in the last seven can make the difference for the Dutch \gls{ps} as key success factors. These two were not found in literature and can be unique for the Dutch \gls{ps}. We recommended starting with the first seven, possibly with the two possible key success factors not found in the literature.
\begin{table}[H]
	\centering
	\small
	\begin{tabular}{@{}cll@{}}
		\toprule
		\textbf{\#} & \textbf{Attribute} & \textbf{Category} \\%
		\midrule
		1 & \Gls{optionality} & \Gls{antifragile} \\%
		2 & \Gls{failfast} & \Gls{antifragile} \\%
		3 & \Gls{resourcestoinvest} & \Gls{antifragile} \\%
		4 & \Gls{systeminenvironment} & \gls{ea} \\%
		5 & \Gls{environmentallearning} & \gls{ea} \\%
		6 & \Gls{intraorganisationalcoherency} & \gls{ea} \\%
		7 & \Gls{systeminenvironmentcoevolutionlearning} & \gls{ea} \\%
		\hdashline %
		8 & \Gls{nonmonotonicity} & \Gls{antifragile}  \\%
		9 & \Gls{selforganisation} & \Gls{antifragile}  \\%
		10 & \Gls{senecabarbell} & \Gls{antifragile}  \\%
		11 & \Gls{safeworkingenvironment}\textsuperscript{*} & \Gls{antifragile}  \\%
		12 & \Gls{holisticsystemicstance} & \gls{ea}  \\%
		13 & \Gls{organisationallearning} & \gls{ea}  \\%
		14 & \Gls{adapttobusinesslanguage}\textsuperscript{*} & \gls{ea}  \\%
		\bottomrule
		\multicolumn{2}{l}{* Not found in literature}
	\end{tabular}%
	\caption*{Potential success factors}
	\label{executive:potentialsuccess}%
\end{table}%
The topic of \gls{antifragile} is still relatively young, and as far as we have been able to find, it has not been used in practice yet in the context of the (Dutch) \gls{ps}. Little information was therefore available to perform a quantitative analysis. We did choose to use the qualitative research method. The challenge of this method was the validation of results. How could we ensure that we have done everything to remove possible subjectivity? With using the triangulation method, we have minimised possible subjectivity by using multiple research tools. 

We performed a literature study. We distilled a list of possible success factors on \gls{antifragile} and \gls{ea}.  We used semi-structured interviews to have the possibility to capture more information than a structured interview. We selected interviewees from the public sector with a role as \gls{cxo} to get the business perspective of the Dutch \gls{ps}. We validated our findings while at the same time we collected new data. The result after analysis was a selection of fourteen possible success factors. Our last validation step was the use of an expert group. We used a different perspective for the expert group members than for the interviewees. We decide to use the \gls{ea} perspective of the Dutch \gls{ps}. We used a group support system for the expert group session for brainstorming and rating possible success factors. After the expert group analysis, the results were a set of fifteen validated possible success factors.

We combined the literature study results, interviews, and expert group. We analysed the possible success factors on the occurrences over the three tools and ranked the possible success factors. We selected the success factors with three occurrences as potential success factors (first seven). The possible success factors were also selected (the last seven), and we dropped the ones with only occurrence.

