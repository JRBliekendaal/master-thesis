\chapter{Blocks of text that can be used}

\section{Validation through an artefact}
Because there is not much known on the applicability of antifragile on Enterprise Architecture, the success factors need to be validated to be true. To validate, the researcher will create an artefact. The Delphi Research Method is used to validate the artefact. By validating the artefact, the researcher can ensure that the success factors are valid with some degree of certainty.\\

\section{CAS System Theory/Complexity sciences}
Quote from AMS011:\\
''The whole is different from the sum of its parts and their interactions'' [61] (p.77) Though emergence, the whole cannog be reduced to the original parts, the whole is considered a new entity or unit. The whole is ''qualitativly different from their parts ... The cannot be meaningfully compared-they are different'' [61] (system holism)

\section{Relevant Laws}
\begin{itemize}
	\item{Second Law of Thermodynamics}
	\item{Conways Law}
	\item{Metcalfe's Law}
\end{itemize}

\section{Discussion on Frameworks like TOGAF}
When the third of the three different schools of Enterprise Architecture, Enterprise Ecological Adaption \parencite{Lapalme2012}, is more suitable for a antifragile enteprise architecture and the residuality theory of \textcite[p. 809]{OReilly2021} is advocating Post-Structuralism, then what is still the added value for frameworks like TOGAF, FEAF, ....
\begin{remark}
reference is not oreilly 2020 but 2021!!!
\end{remark}

\section{public sector stressors}
Floods, municipal redistricting, redistribution of main tasks between the local and the central government but also change in the information architecture of the Public Sector Market or a new centr