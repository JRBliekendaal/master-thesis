\subsection{Interview local government}
\label{sub:interviewlg}
\subsubsection{Question 1 / Enterprise Architecture}
There is somewhat of an \acrshort{ea}, but we are not using it that broadly. As an organisation, we do not have a real \acrshort{ea}. Our organisation is best compared to that of a consultancy firm. Our core task is lobbying advocacy. We guide the things we do, and then again, it concerns the things we do for municipalities or on behalf of municipalities. We have a multi-year vision. We use guiding principles for the things we do.
Nevertheless, there are processes in the making for portfolio management. What do we do, what don't we do, and how do they relate to each other. There is not one responsibility on the \acrshort{ea}. It is a stepped responsibility that lies with committees and the services board. When it comes to IT, the responsibility lies with the Directory of Information Society. \acrshort{ea} is used for assay the request for subsidy. When there is a request for a subsidy that is not in line with the goals, it is not requested by the (European) government.
The case of how \acrshort{ea} contributes to the agility of the public sector is complicated. The public administrators are not architects and vice versa. \acrshort{ea} is hard to understand. \acrshort{ea} does not provide answers for the problems of today. Ultimately \acrshort{ea} should deliver this so the change can start tomorrow. It is essential to clarify the problems that public administrators are having. Both the facts as well the underlying causes. \acrshort{ea} should clarify the differences. Make it clear where we need to go and map out a path to get there, based on little steps that ultimately lead to the goal. Moreover, in the language public administrators understand. The architect must use the language of the stakeholders to make \acrshort{ea} successful (note: success factor).
\subsubsection{Question 2 / Agility of the public sector}
The \gls{ps} is more about risk aversion. Legality is about holding on to what is known. So, it is exactly known what the municipalities do, and we know precisely what the Land Registry does. All the subsystems of the \gls{ps} have a defined assignment. Moreover, it would be best if it stayed between the lines. Think, for example, about purpose limitation. Purpose Limitation will hold it back when \gls{ps} wants to be agile. The \gls{ps} cannot experiment that easily with rules like these. It will put experimentation at the edge.
The operating model of the \gls{ps} does not offer the freedom to do so that easily. The \gls{ps} can not experiment, discover and then say this was a nice experiment; let us go further. It is impossible to take decisions on the whole, such as within the social domain. If you have the right to confiscate a car, you cannot decide that there need to be a taxi to drive kids to school. It is a responsibility of a different part of the \gls{ps}. This all has to do with \gls{houseofthorbecke}\footnote{\url{https://www.denederlandsegrondwet.nl/id/vieqcpdzf0gw/bestuurlijke_indeling}} together with the current legislation and regulations. It does not mean that the \gls{ps} does not want to be more agile. It is not about separate parts of the government anymore. There is more and more collaboration based on federation. The \gls{ps} wants to be more agile. It is often a subject of conversation. Nevertheless, it gets stuck in the administrative decision-making processes.
\subsubsection{Question 3 / Dealing with uncertainty}
The reflex on \gls{uncertainty} of the \gls{ps} is that the \gls{ps} gets very insecure from \gls{uncertainty}. So the \gls{ps} does not know how to deal with \gls{uncertainty}. The common reflex is to push the uncertainty back to robust/resilient, so it is under control again. Robust \& resilient is back to its previous state but then sturdier, more robust (note: Risk avoidance). However, the \gls{ps} claims that they can deal with it. See, for example, the energy transition. The \gls{ps} defined the framework for this transition. It can contribute to the economy with many new jobs and a new knowledge model. We did see that also in the past with road and waterway engineering. Nevertheless, then they forget about the mechanisms needed to accomplish it. So the \gls{ps} does want to deal with \gls{uncertainty}, but the \gls{ps} is not creating the right conditions or the freedom of acting to be able to do so.
The available \acrshort{ea}'s within the \gls{ps} do not help either. It does not contribute to accept \gls{uncertainty}. At the most, our new vision on Information, Common Ground\footnote{\url{https://commonground.nl/}}, is contributing to this. An important principle in that vision is the ''community'' principle that could help with this \gls{uncertainty} (note: shared mental models). This principle states that municipalities, chain partners, market parties and the \acrshort{vngr} work together as a community in realisation. It is the certainty that it is uncertain. There are always new issues, and organising collaboration will help us to better deal with this \gls{uncertainty}, especially in the \gls{ps}.
\subsubsection{Question 4 / Dealing with unexpected events}
We, fortunately, live in a country where the \gls{ps} is staffed with good people who understand what citizens need or what is needed in a disaster area. So help is available pretty quickly. For example the fires at Moerdijk\footnote{\url{https://nl.wikipedia.org/wiki/Brand_Moerdijk_5_januari_2011}}, near Rotterdam. Or the plane crash of Turkish Airlines\footnote{\url{https://en.wikipedia.org/wiki/Turkish_Airlines_Flight_1951}} at Schiphol. Before we knew it, the fires were already distinguished. However, the \gls{ps} is not able to ascertain if it is an incident or a structural problem. If the problem takes too long, we see the reflex to control it again, so we fall back to our past habits. Adopting newly learned patterns is hard with the current legislation and regulations. Maybe we must be in a continuous crisis, so we have the freedom to do what we have to do. It is easier to be more agile in a crisis. With the local governments, there is only one who can decide what to do in a crisis, and this is the mayor as part of the triangle\footnote{\url{https://nl.wikipedia.org/wiki/Driehoek_(overheid)}}. 
It seems that the local governments have two organisation \& operational models. A model for running the municipality in a normal state and one in a state of crisis. (note: Seneca's Barbell?)
\subsubsection{Question 5 / The risk appetite of the public sector}
Drive and urge for innovation and change, which could perhaps be another interpretation of the word risk. People in the \gls{ps} want to find out if it can be done differently, but it is very dependent on the person. So actually, that depends on people in the \gls{ps}. Not on the \gls{ps} as a whole, based on intrinsic motivation to make things better for citizens and entrepreneurs. We have even set up our organisation in such a way that we can support this. Nowadays we have a department for research and innovation. But after something new is thought of it will go to the department to create it and finally to to a department to maintain it. Sometimes we are limited because it influences the standing legislation and regulations.
\subsubsection{Question 6 / Using diversity and optionality in the public sector}
The \gls{ps} is not supporting diversity and optionality, but on the other hand the \gls{ps} is based on the absolute premise that all organisations are autonomous. (note: with a clear goal per organisation so no options). The implementation of the policies is extremely diverse. For example, there are various approaches known for people who are unemployed. One municipality forces people to perform work before they receive benefits, while another municipality supports them to be financially healthy again. Both achieve the obligation to provide care to a citizen so that the citizen has an income again. Diversity and optionality are less important. The local authorities simply have to implement the policies. It is all set down in the law and regulations. There is quite a diversity in how municipalities organize things. One municipality collects the household waste itself, while another has outsourced it. In both cases, the collection of household waste is arranged. Nobody prescribes how you arrange it, as long as it is arranged. You see the same thing within IT. However, you see here that for transitions and transformations an \acrshort{ea} is enormously needed to support the new world.
\subsubsection{Closing statements}
The government will not disappear, but that trust in the government will. Some government organizations or parts of them can undoubtedly disappear. (note: does not fit with the questions but find it an important statement).
The \gls{ps} needs a cross-sector \acrshort{ea}. It's no longer just about your organization. You have to work together more and more. The \acrshort{ea} can then be further tailored to your own organization.