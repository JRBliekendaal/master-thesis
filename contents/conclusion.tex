\chapter{Conclusion and discussions}
\label{ch:conclusionanddiscussions}
blah
\section{Conclusion}
\label{sec:conclusion}
We conducted literature research to answer the sub-questions. What is the Dutch \gls{ps}? What is \gls{antifragile}, and what are possible success factors for \gls{antifragility}? What is \acrlong{ea}, and what are the possible success factors of \acrlong{ea}? We presented the answers to these questions in the background chapter (\cref{ch:theoreticalbackground}) of this thesis. The literature research provided a list of \textit{twenty-three} generic \glspl{attribute} of \gls{antifragile} and \textit{six} generic \gls{attribute} of \acrlong{ea} that can positively influence achieving \gls{antifragility} with \acrlong{ea}.

Following this, we conducted interviews with \glspl{cxo} from the Dutch public sector (\cref{ch:interviews}) and an expert group composed of (Enterprise) Architects from the Dutch \gls{ps} (\cref{ch:expertgroup}) to determine which of those \glspl{attribute} are applicable to the Dutch \gls{ps}. We performed a \acrlong{qda} on the data set of the interviews, and we used a Group Support System for the expert group. The Group Support System provided us with the necessary tools and helped us capture the findings. 

We combined the literature study results, interviews, and expert group for \gls{triangulation} (\cref{sub:triangulation}). We analysed the \glspl{attribute} on the occurrence of the \glspl{attribute} in all three research tools (\cref{app:combinedfindings}). If an \gls{attribute} occurred in all three research tools, we decided that this \gls{attribute} is \textit{most likely} a success factor for the Dutch \gls{ps}. When an \gls{attribute} occurred in two out of three research tools, we decided that this \gls{attribute} is \textit{likely} a success factor for the Dutch \gls{ps} but it needs more research to determine this. \Glspl{attribute} that occurred in only one of the three research tools are \textit{least likely} to be a success factor for the Dutch \gls{ps}. We know that we can only say this is true for the used data sets.

We can conclude that there are three \gls{antifragile} and four \acrlong{ea} \glspl{attribute} \textit{most likely} to be a success factor for the Dutch \gls{ps} (\cref{tab:identifiedsuccessfactors}). The three \gls{antifragile} \glspl{attribute} are \textit{\gls{optionality}}, \textit{\gls{failfast}}, and \textit{\gls{resourcestoinvest}}. The four \acrlong{ea} \glspl{attribute} are \textit{\gls{systeminenvironment}}, \textit{\gls{environmentallearning}}, \textit{\gls{intraorganisationalcoherency}}, and \textit{\gls{systeminenvironmentcoevolutionlearning}}.

\begin{table}[H]
	\centering
	\begin{tabular}{@{}ll@{}}
		\toprule
	\textbf{Attribute} & \textbf{Category} \\%
		\midrule
		\Gls{optionality} & \Gls{antifragile} \\%
		\Gls{failfast} & \Gls{antifragile} \\%
		\Gls{resourcestoinvest} & \Gls{antifragile} \\%
		\Gls{systeminenvironment} & \acrlong{ea} \\%
		\Gls{environmentallearning} & \acrlong{ea} \\%
		\Gls{intraorganisationalcoherency} & \acrlong{ea} \\%
		\Gls{systeminenvironmentcoevolutionlearning} & \acrlong{ea} \\%
		\bottomrule
	\end{tabular}%
	\caption[Most likely success factors]{Most likely success factors}
	\label{tab:identifiedsuccessfactors}%
\end{table}%
We discovered seven \textit{likely} success factors (\cref{tab:identifiedpossiblefactors}). These are \glspl{attribute} found in two out of three research tools. Four of those \textit{likely} success factors are \gls{antifragile} \glspl{attribute}. For \acrlong{ea}, we found three \textit{likely} success factors. The four \textit{likely} success factors of \gls{antifragile} are \textit{\gls{nonmonotonicity}}, \textit{\gls{selforganisation}}, \textit{\gls{senecabarbell}}, and \textit{\gls{safeworkingenvironment}}. The three \textit{likely} success factors of \acrlong{ea} are \textit{\gls{holisticsystemicstance}}, \textit{\gls{organisationallearning}}, and \textit{\gls{adapttobusinesslanguage}}.
\begin{table}[H]
	\centering
	\begin{tabular}{@{}ll@{}}
		\toprule
		\textbf{Attribute} & \textbf{Category} \\%
		\midrule
		\Gls{nonmonotonicity} & \Gls{antifragile}  \\%
		\Gls{selforganisation} & \Gls{antifragile}  \\%
		\Gls{senecabarbell} & \Gls{antifragile}  \\%
		\Gls{safeworkingenvironment} & \Gls{antifragile}  \\%
		\Gls{holisticsystemicstance} & \acrlong{ea}  \\%
		\Gls{organisationallearning} & \acrlong{ea}  \\%
		\Gls{adapttobusinesslanguage} & \acrlong{ea}  \\%
		\bottomrule
	\end{tabular}%
	\caption[Likely success factors]{Likely success factors}
	\label{tab:identifiedpossiblefactors}%
\end{table}%
An important observation is that the \textit{likely} success factors \textit{\gls{safeworkingenvironment}} and \textit{\gls{adapttobusinesslanguage}} do not originate from the literature research. These two \textit{likely} success factors could make a difference for the Dutch \gls{ps} as key differentiators. These \glspl{attribute} are not part of the existing \acrlong{bok} on \acrlong{ea} and \gls{antifragile}.

Implementing \gls{antifragility} in the Dutch \gls{ps} must incorporate at least the \glspl{attribute} \gls{optionality}, \gls{failfast}, and \gls{resourcestoinvest} to be successful. \acrlong{ea} can support this transformation successfully by incorporating the \gls{systeminenvironment}, \gls{environmentallearning}, \gls{intraorganisationalcoherency}, and \gls{systeminenvironmentcoevolutionlearning} attributes in their Enterprise Architecture practices.

\section{Research limitations}
\label{sec:limitations}
This research is subject to several limitations. The study is of a qualitative method. After aliterature research, we gathered most of the results through interviews and an expert group. The sample size used for the interviews and the expert group was small. We only interviewed four CxO's of different public sector organisations to get different viewpoints. It was hard to find the participants who fulfilled the profile of knowing Enterprise Architecture. We had a high attendance during the expert group session. Ten of eleven participants joined the expert group session. We carefully selected people with a profile of experience with Enterprise Architecture from different public sector organisations. We tried to balance the various organisations like the central government, the local governments and suppliers. However, this selection also narrowed our options. We have deliberately chosen to work with experts for an excellent qualitative result. But this is not large enough to have a real impact. Everything was about interpretation. We tried to overcome this with multiple methods with triangulation with numerous strategies and the composition of the interview and expert group participants. We did not use a blind expert group to foster dialogue and discussion between the participants to get a shared mental model. Nevertheless, it is still possible that the results can differ if we replicate the research with other participants. The results are only trustworthy for the collected data sets. 

Another limitation is the absence of literature on the research subjects, particularly in \acrlong{ea}, \gls{antifragility}, with the Dutch \gls{ps}. Literature and research are scarce on this combination of subjects. The result of this research can be rebutted when more information comes available. But at least now there is information available, and I hope this will support further research.

The interpretation of data sets is also a limitation. The coding of the transcriptions is my interpretation of what is said. An interviewee or an expert group participant rarely uses the same names and phrases as the codes. I tried to overcome these limitations in multiple ways. One of the methods was creating summaries of the interview transcriptions in my own words and validating these with the interviewees. For the expert group, we used an open group with the possibility of discussions and dialogue. If something was not evident, the participants explained it themselves.

The last limitation is that of the chosen boundary of the research. We focussed only on the central and local governments with suppliers. The \gls{ps} contains more, e.g. semi-governmental organisations, healthcare, education, public transportation, and others. When other \gls{ps} organisations are put into the scope of this research, there is a possibility of different results. We see the same limitation by narrowing it down to only one part of the \gls{ps}, e.g., the local government. Also, in this case, the results are only trustworthy for the data we collected during the research.

\section{Discussions}
\label{sec:discussions}

\subsection{The relevance of this research}
\label{sub:relevanceofresearch}
The relevance of the research is always a discussion. I, as a researcher, may find it relevant but does my target audience also thinks it is relevant? Until now, everyone is enthusiastic about the research and the study results. As an example, the results of a rating with the expert group (\cref{tab:relevance}). I asked them to what extent do you find this research relevant, with a rating between one and ten. One for least relevant and ten for most relevant. The results surprised my promotor, co-promotor and me. The expert group participants find the research relevant. They rated it with an 8,2 with low variability.
\begin{table}[H]
	\centering
	\begin{tabular}{p{.55\textwidth}ccc}
		\toprule
		\textbf{Question} & \textbf{Rating} & \textbf{Variability} & \textbf{Abstains} \\
		\midrule
		To what extent do you find the research relevant? & 8,2 & 23\% & 0 \\%
		To what extent do you think that the research can be used by yourself? & 7,7 & 10\% & 0 \\%
		To what extent do you think that the research can be used in the public sector? & 7,2 & 32\% & 1 \\%
		To what extent do you think that the research can be used by your organisation? & 6,6 & 33\% & 0 \\%
		\bottomrule
	\end{tabular}%
	\caption[What is the relevance of this research?]{What is the relevance of this research?}
	\label{tab:relevance}%
\end{table}%

\subsection{Is the Dutch public sector different than that from other countries?}
\label{sub:differencepublicsector}
I did not perform wide-scale research on \gls{ps} worldwide. But in general, the Dutch \gls{ps} is relatively unique. Yes, it has the same three-layer structure as most democratic countries. However, The Netherlands is a decentralised unitary state, while most comparable countries have a centralised unitary state. This is one of the main reasons that the Estonian model\footnote{\url{https://www.cnbc.com/2019/02/08/how-estonia-became-a-digital-society.html}} does not work one-on-one with The Netherlands. We see this in practice with the fact that we have \underline{n} possible implementation of the law with the municipalities where \underline{n} is the number of municipalities. What about the results of this research. Can it be scaled towards other countries? I did scope this research on the Dutch \gls{ps} on purpose. I do have access to the Dutch \gls{ps} resources, and I do not have that with other countries. This research is a master's thesis and not a PhD. There is not enough time to do that research thoroughly. So I do not know. Nevertheless, it can be an excellent topic for research.

\subsection{Differences between the central and local Dutch governments}
\label{sub:differenceslocalvscentral}
While performing the research, I realised that you could not just say the \gls{ps} as a generalisation of everything in the \gls{ps}. E.g. I noticed differences in the way of working and culture between the central government and the local governments. The local governments have more possibilities to experiment and do things differently than the central government. This difference makes the local governments also more \gls{antifragile}. As \textcite{Taleb2012} tells us, \gls{diversity} and \gls{optionality} makes us \gls{antifragile}. The more \gls{fragile} systems in a system, the more \gls{antifragile} the system is. But most of the time the local governments do not have the \gls{attribute} \gls{resourcestoinvest} as a system property. They are continuously under stress by performing the laws with less budget every year. Most interesting for further research is to repeat the same research but only for one type of government and see the differences.

\subsection{Blaming culture in the Public Sector}
\label{sub:blamingculture}
While conducting interviews, I noticed that there is a lot of blaming going on in the \gls{ps}, especially with the central governments. One little misstep can already have major repercussions because of public accountability, even if that misstep solves a problem or delivers a positive effect. People are getting afraid of deviating from prescribed paths because of this. This behaviour is an antipattern in becoming \gls{antifragile} which stimulates e.g. \gls{failfast}, \gls{insertrandomness}, and \gls{insertlowlevelstress}. This is the main reason that the \gls{safeworkingenvironment} \gls{attribute} was found in the interviews. In the end \gls{safeworkingenvironment} was rated as likely. It was not discovered with literature research. This finding can be a key differentiator for the Dutch \gls{ps} to become more \gls{antifragile}.

\subsection{Adapt to business language}
\label{sub:adapttobusinesslanguage}
Interviewees and expert group participants said that \acrlong{ea} is not seen as of value by the policymakers. As long as \acrlong{ea} does not speak the \textit{natural} language of the policymakers, \acrlong{ea} will be confronted with decisions instead of being involved with decision making. One interviewee explained it well. This statement ended up as a new \gls{attribute} as \gls{adapttobusinesslanguage}. It is categorised as a \textit{likely} success factor because it was not a result of the literature research. I think that this \gls{attribute} is extremely important and should be solved first. Otherwise, \acrlong{ea} cannot ''respond adequately and flexibly to unforeseen circumstances.'' \parencite{Secretarissen-generaal2018}. But I also think this problem is not unique to the Dutch \gls{ps} or the \gls{ps} in general. From my experience, this topic is already on the agenda of many researchers for years. One of the expert group participants even said we need to be more business-savvy but beware that the other side of the table should become more IT-savvy. I think the \gls{attribute} \gls{adapttobusinesslanguage} should be a pre-condition—a pre-condition for \acrlong{ea} to support the public sector in being more \gls{antifragile}. Maybe there is not one solution that will fit all. It could be that research on a particular organisation, system or type will give us an answer on how to deal with this on a smaller scale.

\subsection{Enterprise Architecture in general}
\label{sub:eainthepublicsector}
The more I descended into the depths of \acrlong{ea}, the more I became aware that researchers and practitioners agreed to disagree. Agree to disagree on what \acrlong{ea} is. Let alone how we should practice it. The result is that every organisation and enterprise architect uses \acrlong{ea} differently. Even worse, with their definition of it. Although I have been working as an (Enterprise) Architect for 25 years, I was never as aware of this as I am now. But if there is no consensus on what \acrlong{ea} is, is it possible to determine the success factors? At least they agree that there are multiple definitions, approaches, strategies and schools of thought. So the best I could do for this research was to make the most likely choice for an \gls{antifragile} system.

\subsection{How recent is this research (wrong word recent)}

think about covid, think about ukraine, and others at this moment.


\subsection{Was reinstating the attribute optionality a good approach?}
\label{sub:reinstating optionality}


\subsection{Definition of System Thinking vs Emergence of Botjes}
\label{sub:systemsthinkingdiscussion}

\section{Recommendations}
\label{sec:reccomandations}

\begin{itemize}
	\item{Research not on the public sector but only parts of the public sector}
	\item{seneca's barbell, scope of change (not everything has to be antifragile.. stable core emerging only for righthandside) start small fail fast}
\end{itemize}

\section{Retrospective}
\label{sec:retrospective}

\subsection{Quality of Research}

\subsection{Process of Research}

\subsubsection{Research organisation}

Hans as the red line

co promotor

Support from organisation
Directors but also direct colleagues



\subsection{Researcher}

\begin{center}
	\textit{''I have found out how little I actually know''}
\end{center}

\begin{itemize}
	\item{The added value of a Co-Promotor}
\end{itemize}