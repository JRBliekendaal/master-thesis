\chapter{Conclusion and discussions}
\label{ch:conclusionanddiscussions}
We followed the steps of the research, and we answered our sub-questions. We did our literature research and interviews. We validated the findings with an expert group and finished it with analysis. There is only one thing left to do: interpret the results and give a conclusion to this research on particular our main research question. After our conclusion, we will discuss possible limitations, discussions and recommendations. We will end this chapter with a retrospective on the research process, the organisation and the researcher.

\section{Conclusion}
\label{sec:conclusion}
We conducted literature research to answer the sub-questions. What is the Dutch \gls{ps}? What is \gls{antifragile}, and what are possible success factors for \gls{antifragility}? What is \acrlong{ea}, and what are the possible success factors of \acrlong{ea}? We presented the answers to these questions in the background chapter (\cref{ch:theoreticalbackground}) of this thesis. The literature research provided a list of \textit{twenty-three} generic \glspl{attribute} of \gls{antifragile} and \textit{six} generic \gls{attribute} of \acrlong{ea} that can positively influence achieving \gls{antifragility} with \acrlong{ea}.

Following this, we conducted interviews with \glspl{cxo} from the Dutch public sector (\cref{ch:interviews}) and an expert group session composed of (Enterprise) Architects from the Dutch \gls{ps} (\cref{ch:expertgroup}) to determine which of those \glspl{attribute} are applicable to the Dutch \gls{ps}. We performed a \acrlong{qda} on the data set of the interviews, and we used a Group Support System for the expert group. The Group Support System provided us with the necessary tools and helped us capture the findings. 

We combined the literature study results, interviews, and expert group for \gls{triangulation} (\cref{sub:triangulation}). We analysed the \glspl{attribute} on the occurrence of the \glspl{attribute} in all three research tools (\cref{app:combinedfindings}). If an \gls{attribute} occurred in all three research tools, we decided that this \gls{attribute} is \textit{most likely} a success factor for the Dutch \gls{ps}. When an \gls{attribute} occurred in two out of three research tools, we decided that this \gls{attribute} is \textit{likely} a success factor for the Dutch \gls{ps} but it needs more research to determine this. \Glspl{attribute} that occurred in only one of the three research tools are \textit{least likely} to be a success factor for the Dutch \gls{ps}. We know that we can only say this is true for the used data sets.

We can conclude that there are three \gls{antifragile} and four \acrlong{ea} \glspl{attribute} \textit{most likely} to be a success factor for the Dutch \gls{ps} (\cref{tab:identifiedsuccessfactors}). The three \gls{antifragile} \glspl{attribute} are \textit{\gls{optionality}}, \textit{\gls{failfast}}, and \textit{\gls{resourcestoinvest}}. The four \acrlong{ea} \glspl{attribute} are \textit{\gls{systeminenvironment}}, \textit{\gls{environmentallearning}}, \textit{\gls{intraorganisationalcoherency}}, and \textit{\gls{systeminenvironmentcoevolutionlearning}}.

\begin{table}[H]
	\centering
	\begin{tabular}{@{}ll@{}}
		\toprule
	\textbf{Attribute} & \textbf{Category} \\%
		\midrule
		\Gls{optionality} & \Gls{antifragile} \\%
		\Gls{failfast} & \Gls{antifragile} \\%
		\Gls{resourcestoinvest} & \Gls{antifragile} \\%
		\Gls{systeminenvironment} & \acrlong{ea} \\%
		\Gls{environmentallearning} & \acrlong{ea} \\%
		\Gls{intraorganisationalcoherency} & \acrlong{ea} \\%
		\Gls{systeminenvironmentcoevolutionlearning} & \acrlong{ea} \\%
		\bottomrule
	\end{tabular}%
	\caption[Most likely success factors]{Most likely success factors}
	\label{tab:identifiedsuccessfactors}%
\end{table}%
We discovered seven \textit{likely} success factors (\cref{tab:identifiedpossiblefactors}). These are \glspl{attribute} found in two out of three research tools. Four of those \textit{likely} success factors are \gls{antifragile} \glspl{attribute}. For \acrlong{ea}, we found three \textit{likely} success factors. The four \textit{likely} success factors of \gls{antifragile} are \textit{\gls{nonmonotonicity}}, \textit{\gls{selforganisation}}, \textit{\gls{senecabarbell}}, and \textit{\gls{safeworkingenvironment}}. The three \textit{likely} success factors of \acrlong{ea} are \textit{\gls{holisticsystemicstance}}, \textit{\gls{organisationallearning}}, and \textit{\gls{adapttobusinesslanguage}}.
\begin{table}[H]
	\centering
	\begin{tabular}{@{}ll@{}}
		\toprule
		\textbf{Attribute} & \textbf{Category} \\%
		\midrule
		\Gls{nonmonotonicity} & \Gls{antifragile}  \\%
		\Gls{selforganisation} & \Gls{antifragile}  \\%
		\Gls{senecabarbell} & \Gls{antifragile}  \\%
		\Gls{safeworkingenvironment} & \Gls{antifragile}  \\%
		\Gls{holisticsystemicstance} & \acrlong{ea}  \\%
		\Gls{organisationallearning} & \acrlong{ea}  \\%
		\Gls{adapttobusinesslanguage} & \acrlong{ea}  \\%
		\bottomrule
	\end{tabular}%
	\caption[Likely success factors]{Likely success factors}
	\label{tab:identifiedpossiblefactors}%
\end{table}%
An important observation is that the \textit{likely} success factors \textit{\gls{safeworkingenvironment}} and \textit{\gls{adapttobusinesslanguage}} do not originate from the literature research. These two \textit{likely} success factors could make a difference for the Dutch \gls{ps} as key differentiators. These \glspl{attribute} are not part of the existing \acrlong{bok} on \acrlong{ea} and \gls{antifragile}.

Implementing \gls{antifragility} in the Dutch \gls{ps} must incorporate at least the \glspl{attribute} \textit{\gls{optionality}}, \textit{\gls{failfast}}, and \textit{\gls{resourcestoinvest}} to be successful. \acrlong{ea} can support this transformation successfully by incorporating the \textit{\gls{systeminenvironment}}, \textit{\gls{environmentallearning}}, \textit{\gls{intraorganisationalcoherency}}, and \textit{\gls{systeminenvironmentcoevolutionlearning}} attributes in their \acrlong{ea} practices.

\section{Research limitations}
\label{sec:limitations}
This research is subject to several limitations. The study is of a qualitative method. After aliterature research, we gathered most of the results through interviews and an expert group. The sample size used for the interviews and the expert group was small. We only interviewed four CxO's of different public sector organisations to get different viewpoints. It was hard to find the participants who fulfilled the profile of knowing Enterprise Architecture. We had a high attendance during the expert group session. Ten of eleven participants joined the expert group session. We carefully selected people with a profile of experience with Enterprise Architecture from different public sector organisations. We tried to balance the various organisations like the central government, the local governments and suppliers. However, this selection also narrowed our options. We have deliberately chosen to work with experts for an excellent qualitative result. But this is not large enough to have a real impact. Everything was about interpretation. We tried to overcome this with multiple methods with triangulation with numerous strategies and the composition of the interview and expert group participants. We did not use a blind expert group to foster dialogue and discussion between the participants to get a shared mental model. Nevertheless, it is still possible that the results can differ if we replicate the research with other participants. The results are only trustworthy for the collected data sets. 

Another limitation is the absence of literature on the research subjects, particularly in \acrlong{ea}, \gls{antifragility}, with the Dutch \gls{ps}. Literature and research are scarce on this combination of subjects. The result of this research can be rebutted when more information comes available. But at least now there is information available, and I hope this will support further research.

The interpretation of data sets is also a limitation. The coding of the transcriptions is my interpretation of what is said. An interviewee or an expert group participant rarely uses the same names and phrases as the codes. I tried to overcome these limitations in multiple ways. One of the methods was creating summaries of the interview transcriptions in my own words and validating these with the interviewees. For the expert group, we used an open group with the possibility of discussions and dialogue. If something was not evident, the participants explained it themselves.

The last limitation is that of the chosen boundary of the research. We focussed only on the central and local governments with suppliers. The \gls{ps} contains more, e.g. semi-governmental organisations, healthcare, education, public transportation, and others. When other \gls{ps} organisations are put into the scope of this research, there is a possibility of different results. We see the same limitation by narrowing it down to only one part of the \gls{ps}, e.g., the local government. Also, in this case, the results are only trustworthy for the data we collected during the research.

\section{Discussions}
\label{sec:discussions}

\subsection{The relevance of this research}
\label{sub:relevanceofresearch}
The relevance of the research is always a discussion. I, as a researcher, may find it relevant but does my target audience also thinks it is relevant? Until now, everyone is enthusiastic about the research and the study results. As an example, the results of a rating with the expert group (\cref{tab:relevance}). I asked them to what extent do you find this research relevant, with a rating between one and ten. One for least relevant and ten for most relevant. The results surprised my promotor, co-promotor and me. The expert group participants find the research relevant. They rated it with an 8,2 with low variability.
\begin{table}[H]
	\centering
	\begin{tabular}{p{.55\textwidth}ccc}
		\toprule
		\textbf{Question} & \textbf{Rating} & \textbf{Variability} & \textbf{Abstains} \\
		\midrule
		To what extent do you find the research relevant? & 8,2 & 23\% & 0 \\%
		To what extent do you think that the research can be used by yourself? & 7,7 & 10\% & 0 \\%
		To what extent do you think that the research can be used in the public sector? & 7,2 & 32\% & 1 \\%
		To what extent do you think that the research can be used by your organisation? & 6,6 & 33\% & 0 \\%
		\bottomrule
	\end{tabular}%
	\caption[What is the relevance of this research?]{What is the relevance of this research?}
	\label{tab:relevance}%
\end{table}%

\subsection{Is antifragile and Enterprise Architecture in the (Dutch) public sector a timely topic?}
\label{sub:timelytopic}
What about the timeliness of the topic of \gls{antifragility} and \acrlong{ea}? At this moment, many things are happening in the environment of the (Dutch) \gls{ps} that they cannot control. We are (hopefully) at the end of the COVID-19 pandemic, and the subsequent unexpected events are happening already. Unforeseen circumstances for which the \gls{ps} cannot prepare themselves. Think about the refugee crisis caused by the war between Russia and Ukraine and the energy crisis because of this war with tremendous consequences for not only Russia and Ukraine. Think about the possible shortage of oil and gas because of this. If we need the capability to respond adequately and be flexible for unforeseen circumstances, it is now. With this, I do not say that \gls{antifragile} with \acrlong{ea} is the silver bullet for the \gls{ps}. But it will undoubtedly support the \gls{ps} in being more responsive and more adaptive. What if we had embedded the system \gls{attribute} \gls{optionality} for our gas and oil supply. Would we still have issues on this matter?

\subsection{Is the Dutch public sector different than that from other countries?}
\label{sub:differencepublicsector}
I did not perform wide-scale research on \gls{ps} worldwide. But in general, the Dutch \gls{ps} is relatively unique. Yes, it has the same three-layer structure as most democratic countries. However, The Netherlands is a decentralised unitary state, while most comparable countries have a centralised unitary state. This is one of the main reasons that the Estonian model\footnote{\url{https://www.cnbc.com/2019/02/08/how-estonia-became-a-digital-society.html}} does not work one-on-one with The Netherlands. We see this in practice with the fact that we have \underline{n} possible implementation of the law with the municipalities where \underline{n} is the number of municipalities. What about the results of this research. Can it be scaled towards other countries? I did scope this research on the Dutch \gls{ps} on purpose. I do have access to the Dutch \gls{ps} resources, and I do not have that with other countries. This research is a master's thesis and not a PhD. There is not enough time to do that research thoroughly. So I do not know. Nevertheless, it can be an excellent topic for research.

\subsection{Differences between the central and local Dutch governments}
\label{sub:differenceslocalvscentral}
While performing the research, I realised that you could not just say the \gls{ps} as a generalisation of everything in the \gls{ps}. E.g. I noticed differences in the way of working and culture between the central government and the local governments. The local governments have more possibilities to experiment and do things differently than the central government. This difference makes the local governments also more \gls{antifragile}. As \textcite{Taleb2012} tells us, \gls{diversity} and \gls{optionality} makes us \gls{antifragile}. The more \gls{fragile} systems in a system, the more \gls{antifragile} the system is. But most of the time the local governments do not have the \gls{attribute} \gls{resourcestoinvest} as a system property. They are continuously under stress by performing the laws with less budget every year. Most interesting for further research is to repeat the same research but only for one type of government and see the differences.

\subsection{Blaming culture in the Public Sector}
\label{sub:blamingculture}
While conducting interviews, I noticed that there is a lot of blaming going on in the \gls{ps}, especially with the central governments. One little misstep can already have major repercussions because of public accountability, even if that misstep solves a problem or delivers a positive effect. People are getting afraid of deviating from prescribed paths because of this. This behaviour is an antipattern in becoming \gls{antifragile} which stimulates e.g. \gls{failfast}, \gls{insertrandomness}, and \gls{insertlowlevelstress}. This is the main reason that the \gls{safeworkingenvironment} \gls{attribute} was found in the interviews. In the end \gls{safeworkingenvironment} was rated as likely. It was not discovered with literature research. This finding can be a key differentiator for the Dutch \gls{ps} to become more \gls{antifragile}.

\subsection{Adapt to business language}
\label{sub:adapttobusinesslanguage}
Interviewees and expert group participants said that \acrlong{ea} is not seen as of value by the policymakers. As long as \acrlong{ea} does not speak the \textit{natural} language of the policymakers, \acrlong{ea} will be confronted with decisions instead of being involved with decision making. One interviewee explained it well. This statement ended up as a new \gls{attribute} as \gls{adapttobusinesslanguage}. It is categorised as a \textit{likely} success factor because it was not a result of the literature research. I think that this \gls{attribute} is extremely important and should be solved first. Otherwise, \acrlong{ea} cannot ''respond adequately and flexibly to unforeseen circumstances.'' \parencite{Secretarissen-generaal2018}. But I also think this problem is not unique to the Dutch \gls{ps} or the \gls{ps} in general. From my experience, this topic is already on the agenda of many researchers for years. One of the expert group participants even said we need to be more business-savvy but beware that the other side of the table should become more IT-savvy. I think the \gls{attribute} \gls{adapttobusinesslanguage} should be a pre-condition—a pre-condition for \acrlong{ea} to support the public sector in being more \gls{antifragile}. Maybe there is not one solution that will fit all. It could be that research on a particular organisation, system or type will give us an answer on how to deal with this on a smaller scale.

\subsection{Enterprise Architecture in general}
\label{sub:eainthepublicsector}
The more I descended into the depths of \acrlong{ea}, the more I became aware that researchers and practitioners agreed to disagree. Agree to disagree on what \acrlong{ea} is. Let alone how we should practice it. The result is that every organisation and enterprise architect uses \acrlong{ea} differently. Even worse, with their definition of it. Although I have been working as an (Enterprise) Architect for 25 years, I was never as aware of this as I am now. But if there is no consensus on what \acrlong{ea} is, is it possible to determine the success factors? At least they agree that there are multiple definitions, approaches, strategies and schools of thought. So the best I could do for this research was to make the most likely choice for an \gls{antifragile} system.

\subsection{Was reinstating the attribute optionality a good approach?}
\label{sub:reinstating optionality}
The \acrlong{eaal} of \textcite{Botjes2021} merged \gls{optionality} into \gls{diversity} because of an overlap between both \glspl{attribute} (\cref{sub:attributesofantifragile}). We decided to reinstate this attribute and added it to the attributes that could be a possible success factor for the (Dutch) \gls{ps}. Despite the overlap in definitions, there is also a very subtle difference between \gls{diversity} and \gls{optionality}. We had the foresight that \gls{optionality} could be important for the Dutch \gls{ps}. We were right. The Dutch \gls{ps} is already diverse, but does not have a lot optionality. The interviews and the expert group confirmed this. \Gls{diversity} did not end up in the list of \glspl{attribute} that can be a possible success factor, but \gls{optionality} did. It even ended up in the list of most likely success factors.

\subsection{Used definitions by Botjes (2021)}
\label{sub:systemsthinkingdiscussion}
We used the \acrlong{eaal} of \textcite{Botjes2021} as our primary model for the \glspl{attribute} of \gls{antifragile}. To understand the \acrlong{eaal} and use it as intended, we adopted the definitions given by \textcite{Botjes2021}. Two definitions are, in my opinion, not correct. These definitions are those of \textit{\gls{failfast}} and \textit{\gls{systemsthinking}}.

\Textcite[Table II]{Botjes2021} defined \gls{failfast} as ''The combined attributes in this group enable the possibility to execute the strategy \gls{failfast}.'' What \textcite{Botjes2021} probably wanted to clarify for the \acrlong{eaal} is that \gls{failfast} is implemented when all the \glspl{attribute} of Complex Adaptive System resilience are adopted. If this is the case, his description is not a definition but a finding or something of this kind. Probably a better description would be something as: 'The combined attributes of Complex Adaptive System resilience enable the possibility to execute the strategy \gls{failfast}. \Gls{failfast} is the principle of freely experimenting and learning while trying to reach the desired result. By quickly finding the failures, you can learn and optimise solutions instantly to reach your goal.'

\Textcite{Botjes2021} did the same with the definition of \gls{systemsthinking}.The definition as given by \textcite{Botjes2021} is: ''Systems thinking is the Fifth Discipline that integrates the other four. Systems thinking also needs the disciplines of building shared vision, mental models, team learning, and personal mastery to realize its potential.'' \textcite{Botjes2021} probably wanted to explain that \gls{systemsthinking} is implemented by the other four \glspl{attribute} of the learning organisation.  \textcite{Botjes2021} makes use of \textcite{Senge1994} to define learning organisation for which \gls{systemsthinking} is part of. \textcite{Senge1994} defined systems thinking differently as ''Systems thinking is a discipline for seeing wholes and a framework for seeing interrelationships rather than things, for seeing patterns of change rather than static snapshots.''

A better description for the definition could be: 'Systems thinking is a discipline for seeing wholes and a framework for seeing interrelationships rather than things, for seeing patterns of change rather than static snapshots \parencite{Senge1994}. Systems thinking is \textcite{Senge1994} Fifth Discipline that integrates building shared vision, mental models, team learning, and personal mastery to realise its potential.'

\section{Recommendations}
\label{sec:reccomandations}
The previous sections already gave a lot of recommendations for further research or action. Nevertheless, I still have one left.

\subsection{How to start with antifragile in the (Dutch) public sector}
\label{sub:howtostart}
Should everything in the Dutch \gls{ps} be \gls{antifragile}? No, this is not the approach that is suggested by \textcite{Taleb2012}. A system consisting out of many small \gls{fragile} sub-systems is also \gls{antifragile}. Also think of the \gls{antifragile} system \gls{attribute} of \gls{senecabarbell}. There is a negative asymmetry if you have more to lose than to benefit from events, you are fragile. With success, you have a positive asymmetry. You gain more than you lose, and so you are \gls{antifragile}. I think it is evident that we really like positive asymmetry. \textcite{Taleb2012} advocates a strategy of going for two extremes that balance each other out. He uses \gls{senecabarbell} to describe a dual attitude of playing it safe in some areas (robust to negative \gls{blackswan}) and taking a lot of small risks in others (open to positive \gls{blackswan}), hence achieving \gls{antifragility}.

When I apply this to the \gls{ps}, we need a stable core with an extreme risk aversion, while on the other hand, we need parts of the public sector that are on the other side of the \gls{senecabarbell} to take high risks. The payoff will be the largest (you gain more than lose). So, where do we need the most responsiveness and adaptivity to make us more flexible to unforeseen circumstances? Start with small experimentations with parts of the Dutch \gls{ps} where it is needed the most. But make the parts of the Dutch \gls{ps} that do not need to be \gls{antifragile} as \gls{robust} as possible. What about the support from \acrlong{ea}? We have seen that the public sector primarily works with intentional deterministic architecture. This approach is a great method to make your system extremely robust. However, this approach is less suitable for an \gls{antifragile} system. A more emerging method for \acrlong{ea} should be more appropriate. Start small with a \gls{looselycoupled} system, \gls{failfast} and apply \gls{nonmonotonicity} to advance.

\section{Retrospective}
\label{sec:retrospective}
My research was on an emerging topic, \gls{antifragility}. Having an entire research organisation helps tremendously in making this research a success. 

I had a co-promotor besides only a promotor. The co-promotor did research on antifragility in the past. With an emerging topic without that many qualitative research papers, it is great to have someone to discuss the subject. Besides the fact that a co-promoter has a lot of content knowledge, a co-promoter is often easier to reach for operational problems. 

For day-to-day matters, I had contact with my co-promotor over Signal. Every week we had an online meeting for an hour to talk things through in detail. My promotor joined every fourth weekly meeting to guard the right processes and govern the quality aspects. When we had an issue we could not solve, the promotor was always available to support us in those matters. One of the success factors of this research was having a co-promotor with a lot of knowledge on the topic and the time to spare.

But this was not the whole research organisation. Before researching, I asked people from my organisation and the \gls{ps} to support me. Glad I did this. It saved my research a couple of times. It was hard to find the right people for interviews. One of the people in my research organisation used his network to solve this. Also, finding particular information that you need to substantiate your findings can be very difficult, especially if you are looking for information fom the Dutch \gls{ps}. All information is public, but it is a lot of information. Sometimes it is just a needle in a haystack. Having the former \acrshort{cio} of the Dutch government in your research organisation is a blessing.

Was my research without any real issues? No, there were lots of problems. It is tough to balance your life when you have responsibilities with your family and employer, follow masterclasses, do assignments and research, and write your thesis all at the same time. It is hard, but sometimes you must make choices—the same with me. I had to choose in December 2021 to pause my research and writing my thesis temporarily. It was too much to handle. But I managed to get back in the saddle and restarted the research in February 2022.

I always was used to knowing the end station of things. My promotor taught me that you do not know where you will end up with scientific research, so you have to enjoy the ride and not only the destination. A problem with not knowing exactly where you will end up is that your timelines are very fluid. That fluid that I had to take two weeks off to work dedicated on the research and thesis, and even that was not enough. I was thrilled that my organisation backed me. They gave me the possibility to finish the research during business hours. It was great to see that many colleagues took it into account, replanned meetings, and supported me by taking on work and other things. 

If you ask me what I have learned from this journey, I will not answer that I know more about \gls{antifragility} than when I started. That one is obvious. I will  not tell what I would do differently next time I do extensive research like this. Every situation is different (life is flux) and I learned from \textcite{Taleb2012} that I can see this research as a rare event. You cannot predict what will happen. So how can I prepare myself better then just having a good plan. But what I really learned and what I want to share is that while performing scientific research I discovered the following:\\

\vspace{\baselineskip}
\begin{center}
\noindent \textit{\textbf{I discovered that I have found out how little I actually know.}}
\end{center}