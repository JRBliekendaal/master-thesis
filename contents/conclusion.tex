\chapter{Conclusion and discussions}
\label{ch:conclusionanddiscussions}

\parencite{Dietz2008} and \parencite{Dietz2013}\\
\parencite{Digitaleoverheid2021}\\
\parencite{Knops2021}\\

\section{Conclusion}
\label{sec:conclusion}


blah subquestions.. blah main research question... explaining...

{\small\tabcolsep=3pt  % hold it local
\begin{longtable}{@{}p{0.2\textwidth}p{0.15\textwidth}p{0.6\textwidth}@{}}
	\textbf{Attribute} & \textbf{Category} & \textbf{Definition} \\%
	\midrule%
	\endhead%
	\hline
	\endfoot%
	\caption[Most likely success factors]{Most likely success factors}
	\label{tab:identifiedsuccessfactors}
	\endlastfoot%
	\Gls{optionality} & \Gls{antifragile} & \Glsdesc*{optionality} \\%
	\Gls{failfast} & \Gls{antifragile} & \Glsdesc*{failfast} \\%
	\Gls{resourcestoinvest} & \Gls{antifragile} & \Glsdesc*{resourcestoinvest} \\%
	\Gls{systeminenvironment} & \acrlong{ea} & \Glsdesc*{systeminenvironment} \\%
	\Gls{environmentallearning} & \acrlong{ea} & \Glsdesc*{environmentallearning} \\%
	\Gls{intraorganisationalcoherency} & \acrlong{ea} & \Glsdesc*{intraorganisationalcoherency} \\%
	\Gls{systeminenvironmentcoevolutionlearning} & \acrlong{ea} & \Glsdesc*{systeminenvironmentcoevolutionlearning} \\%
		\bottomrule%
\end{longtable}%
}
{\small\tabcolsep=3pt  % hold it local
\begin{longtable}{@{}p{0.2\textwidth}p{0.15\textwidth}p{0.6\textwidth}@{}}
	\textbf{Attribute} & \textbf{Category} & \textbf{Definition} \\%
	\midrule%
	\endhead%
	\hline
	\endfoot%
	\caption[Likely success factors]{Likely success factors}
	\label{tab:identifiedpossiblefactors}
	\endlastfoot%
	\Gls{nonmonotonicity} & \Gls{antifragile} & \Glsdesc*{nonmonotonicity} \\%
	\Gls{selforganisation} & \Gls{antifragile} & \Glsdesc*{selforganisation} \\%
	\Gls{senecabarbell} & \Gls{antifragile} & \Glsdesc*{senecabarbell} \\%
	\Gls{safeworkingenvironment} & \Gls{antifragile} & \Glsdesc*{safeworkingenvironment} \\%
	\Gls{holisticsystemicstance} & \acrlong{ea} & \Glsdesc*{holisticsystemicstance} \\%
	\Gls{organisationallearning} & \acrlong{ea} & \Glsdesc*{organisationallearning} \\%
	\Gls{adapttobusinesslanguage} & \acrlong{ea} & \Glsdesc*{adapttobusinesslanguage} \\%
	\bottomrule%
\end{longtable}%
}
\begin{remark}
	Do not forget. \Gls{adapttobusinesslanguage} and \gls{safeworkingenvironment} are special. These were not found in literature. So these can be differentiators especially for the Dutch public sector!!!!! This needs more research!
\end{remark}




\section{Discussions}
\label{sec:discussions}

\begin{itemize}
	\item{Discuss the definition of System Thinking vs Emergence}
	\item{Discuss Blaming Culture Public Sector}
	\item{Discuss Speaking the language of the Business with EA}
	\item{Discuss is the public sector really different from that of the private sector?}
	\item{Discuss is the Dutch public sector different than that from other countries?}
\end{itemize}

\subsection{Is the public sector in The Netherlands unique?}
\label{sub:discussionpublicsector}
Is the public sector in The Netherlands the same as in the rest of the world? This needs further research and needs to be confirmed so that the outcome of this research is universally applicable. Maybe the outcome can be generalised. Further research should demonstrate this.

\subsection{Is the public sector different then the private sector?}
\label{sub:discussionpublicvsprivate}


\subsection{Size of Expert Group}
\label{sub:discussionsizeofeg}
Is the size of the delphi group large enough to determine....
\subsection{The composition of the Expert Gropu}
\label{sub:compositionofeg}
Is the composition of the Delphi Group a good reflection of the Public Sector Market?

\subsection{enterprise archiecture}

agree to disagree. less know than expected. different viewpoints. different schools, opinions, practices, etc.. LIterature study did not help and made it even worse.

\subsection{Causal Loop Diagram}

\section{Recommendations}
\label{sec:reccomandations}

\begin{itemize}
	\item{Research not on the public sector but only parts of the public sector}
\end{itemize}